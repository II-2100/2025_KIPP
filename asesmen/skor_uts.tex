% Options for packages loaded elsewhere
% Options for packages loaded elsewhere
\PassOptionsToPackage{unicode}{hyperref}
\PassOptionsToPackage{hyphens}{url}
\PassOptionsToPackage{dvipsnames,svgnames,x11names}{xcolor}
%
\documentclass[
  letterpaper,
  DIV=11,
  numbers=noendperiod]{scrreprt}
\usepackage{xcolor}
\usepackage{amsmath,amssymb}
\setcounter{secnumdepth}{-\maxdimen} % remove section numbering
\usepackage{iftex}
\ifPDFTeX
  \usepackage[T1]{fontenc}
  \usepackage[utf8]{inputenc}
  \usepackage{textcomp} % provide euro and other symbols
\else % if luatex or xetex
  \usepackage{unicode-math} % this also loads fontspec
  \defaultfontfeatures{Scale=MatchLowercase}
  \defaultfontfeatures[\rmfamily]{Ligatures=TeX,Scale=1}
\fi
\usepackage{lmodern}
\ifPDFTeX\else
  % xetex/luatex font selection
\fi
% Use upquote if available, for straight quotes in verbatim environments
\IfFileExists{upquote.sty}{\usepackage{upquote}}{}
\IfFileExists{microtype.sty}{% use microtype if available
  \usepackage[]{microtype}
  \UseMicrotypeSet[protrusion]{basicmath} % disable protrusion for tt fonts
}{}
\makeatletter
\@ifundefined{KOMAClassName}{% if non-KOMA class
  \IfFileExists{parskip.sty}{%
    \usepackage{parskip}
  }{% else
    \setlength{\parindent}{0pt}
    \setlength{\parskip}{6pt plus 2pt minus 1pt}}
}{% if KOMA class
  \KOMAoptions{parskip=half}}
\makeatother
% Make \paragraph and \subparagraph free-standing
\makeatletter
\ifx\paragraph\undefined\else
  \let\oldparagraph\paragraph
  \renewcommand{\paragraph}{
    \@ifstar
      \xxxParagraphStar
      \xxxParagraphNoStar
  }
  \newcommand{\xxxParagraphStar}[1]{\oldparagraph*{#1}\mbox{}}
  \newcommand{\xxxParagraphNoStar}[1]{\oldparagraph{#1}\mbox{}}
\fi
\ifx\subparagraph\undefined\else
  \let\oldsubparagraph\subparagraph
  \renewcommand{\subparagraph}{
    \@ifstar
      \xxxSubParagraphStar
      \xxxSubParagraphNoStar
  }
  \newcommand{\xxxSubParagraphStar}[1]{\oldsubparagraph*{#1}\mbox{}}
  \newcommand{\xxxSubParagraphNoStar}[1]{\oldsubparagraph{#1}\mbox{}}
\fi
\makeatother


\usepackage{longtable,booktabs,array}
\usepackage{calc} % for calculating minipage widths
% Correct order of tables after \paragraph or \subparagraph
\usepackage{etoolbox}
\makeatletter
\patchcmd\longtable{\par}{\if@noskipsec\mbox{}\fi\par}{}{}
\makeatother
% Allow footnotes in longtable head/foot
\IfFileExists{footnotehyper.sty}{\usepackage{footnotehyper}}{\usepackage{footnote}}
\makesavenoteenv{longtable}
\usepackage{graphicx}
\makeatletter
\newsavebox\pandoc@box
\newcommand*\pandocbounded[1]{% scales image to fit in text height/width
  \sbox\pandoc@box{#1}%
  \Gscale@div\@tempa{\textheight}{\dimexpr\ht\pandoc@box+\dp\pandoc@box\relax}%
  \Gscale@div\@tempb{\linewidth}{\wd\pandoc@box}%
  \ifdim\@tempb\p@<\@tempa\p@\let\@tempa\@tempb\fi% select the smaller of both
  \ifdim\@tempa\p@<\p@\scalebox{\@tempa}{\usebox\pandoc@box}%
  \else\usebox{\pandoc@box}%
  \fi%
}
% Set default figure placement to htbp
\def\fps@figure{htbp}
\makeatother





\setlength{\emergencystretch}{3em} % prevent overfull lines

\providecommand{\tightlist}{%
  \setlength{\itemsep}{0pt}\setlength{\parskip}{0pt}}



 


\KOMAoption{captions}{tableheading}
\makeatletter
\@ifpackageloaded{caption}{}{\usepackage{caption}}
\AtBeginDocument{%
\ifdefined\contentsname
  \renewcommand*\contentsname{Table of contents}
\else
  \newcommand\contentsname{Table of contents}
\fi
\ifdefined\listfigurename
  \renewcommand*\listfigurename{List of Figures}
\else
  \newcommand\listfigurename{List of Figures}
\fi
\ifdefined\listtablename
  \renewcommand*\listtablename{List of Tables}
\else
  \newcommand\listtablename{List of Tables}
\fi
\ifdefined\figurename
  \renewcommand*\figurename{Figure}
\else
  \newcommand\figurename{Figure}
\fi
\ifdefined\tablename
  \renewcommand*\tablename{Table}
\else
  \newcommand\tablename{Table}
\fi
}
\@ifpackageloaded{float}{}{\usepackage{float}}
\floatstyle{ruled}
\@ifundefined{c@chapter}{\newfloat{codelisting}{h}{lop}}{\newfloat{codelisting}{h}{lop}[chapter]}
\floatname{codelisting}{Listing}
\newcommand*\listoflistings{\listof{codelisting}{List of Listings}}
\makeatother
\makeatletter
\usepackage{pdflscape}
\makeatother
\makeatletter
\makeatother
\makeatletter
\@ifpackageloaded{caption}{}{\usepackage{caption}}
\@ifpackageloaded{subcaption}{}{\usepackage{subcaption}}
\makeatother
\usepackage{bookmark}
\IfFileExists{xurl.sty}{\usepackage{xurl}}{} % add URL line breaks if available
\urlstyle{same}
\hypersetup{
  pdftitle={Asesement UTS Berbasis Rubrik},
  colorlinks=true,
  linkcolor={blue},
  filecolor={Maroon},
  citecolor={Blue},
  urlcolor={Blue},
  pdfcreator={LaTeX via pandoc}}


\title{Asesement UTS Berbasis Rubrik}
\author{}
\date{}
\begin{document}
\maketitle


Berikut adalah panduan untuk menilai TUGAS UTS Matakuliah II-2100 bagi
seorang mahasiswa individu. Tugas dibuat dalam bentuk laman web dengan
URL TUGAS UTS yang diberikan. Menggunakan panduan ini, penilai dapat
melakukan asesmen terhadap tugas orang lain, maupun self asesmen pada
laporannya sendiri

\chapter{Instruksi:}\label{instruksi}

\begin{enumerate}
\def\labelenumi{\arabic{enumi}.}
\tightlist
\item
  Pelajari Panduan di
  `https://ii-2100.github.io/2025\_KIPP/asesmen.html' Dimana terdapat
  lima TUGAS: UTS-1, UTS-2, UTS-3, UTS-4, UTS-5
\item
  Pelajari Rubrik setiap TUGAS yang ada dalam dokumen ini
\item
  Temukan Tugas Mahasiswa di URL TUGAS UTS yang diberikan. Halaman
  pertama (index.html) adalah UTS-1. Dari portal ini penilia dapat pergi
  ke UTS-2 dan seteursnya.
\item
  Untuk Setiap TUGAS gunakan Rubrik yang sesuai untuk menilai Tugas
\item
  Laporkan Hasil Pengukuran menggunakan Template di bawah, beserta saran
  perbaikan.
\end{enumerate}

\chapter{\texorpdfstring{\textbf{Bentuk-bentuk
Asesmen:}}{Bentuk-bentuk Asesmen:}}\label{bentuk-bentuk-asesmen}

Untuk mengukur CPMK tersebut, asesmen akan menggunakan beberapa bentuk
penilaian, yaitu:

\begin{enumerate}
\def\labelenumi{\arabic{enumi}.}
\tightlist
\item
  \textbf{Kuis Materi Topik Setiap Minggu (Q-1 s/d Q-14):} Mengukur
  pemahaman konseptual dari 14 topik perkuliahan yang disampaikan setiap
  minggu.
\item
  \textbf{Ujian Tengah Semester (UTS-1 s/d UTS-5):} Fokus pada proyek
  demonstrasi komunikasi personal

  \begin{enumerate}
  \def\labelenumii{\arabic{enumii}.}
  \tightlist
  \item
    UTS-1 All About Me, berisikan pesan yang memperkenalkan sosok diri
    kita
  \item
    UTS-2 Song for you, berisikan pesan berbentuk puisi, lago, dan/atau
    viodeo clip{[}
  \item
    UTS-3 My Stories for You, berisikan kisah inspiratif dan menarik
    yang Anda ingin bagikan dengan pribadi lain
  \item
    UTS-4 My Shape, berisikan laporan siapa Anda berdasar hasil sebuah
    lembar kerja
  \item
    UTS-5 My Personal Review, berisikan telahan pesan personal
    berdasarkan rubrik
  \end{enumerate}
\item
  \textbf{Ujian Akhir Semester (UAS-1 s/d UAS-5):} Fokus pada proyek
  demonstrasi komunikasi inspiratif publik yang mengaplikasikan konsep
  interpersonal.

  \begin{enumerate}
  \def\labelenumii{\arabic{enumii}.}
  \tightlist
  \item
    UAS-1 My Concepts,
  \item
    UAS-2 My Opinions,
  \item
    UAS-3 My Innovations, berisikan desain suatu produk atau layanan
    yang membangun kapasitas dan efektivitas
  \item
    UAS-4 My Knowledge, berisikan pengetahuan dan pembelajaran bagi
    masyarakat atas suatu topik dalam kuliah ini
  \item
    UAS-5 My Professional Reviews, berisikan telaahan pesan publikl
    berdasarkan rubrik.
  \end{enumerate}
\end{enumerate}

\begin{longtable}[]{@{}
  >{\raggedright\arraybackslash}p{(\linewidth - 12\tabcolsep) * \real{0.1429}}
  >{\raggedright\arraybackslash}p{(\linewidth - 12\tabcolsep) * \real{0.1429}}
  >{\raggedright\arraybackslash}p{(\linewidth - 12\tabcolsep) * \real{0.1429}}
  >{\raggedright\arraybackslash}p{(\linewidth - 12\tabcolsep) * \real{0.1429}}
  >{\raggedright\arraybackslash}p{(\linewidth - 12\tabcolsep) * \real{0.1429}}
  >{\raggedright\arraybackslash}p{(\linewidth - 12\tabcolsep) * \real{0.1429}}
  >{\raggedright\arraybackslash}p{(\linewidth - 12\tabcolsep) * \real{0.1429}}@{}}
\caption{Tabel daftar bentuk asesmen, jadwal soft deadline, CPMK yang
diukur serta bobot peniliaian dalam skala
100.}\label{tbl-asesmen}\tabularnewline
\toprule\noalign{}
\begin{minipage}[b]{\linewidth}\raggedright
Jenis
\end{minipage} & \begin{minipage}[b]{\linewidth}\raggedright
Asesmen
\end{minipage} & \begin{minipage}[b]{\linewidth}\raggedright
Soft Deadline Minggu ke
\end{minipage} & \begin{minipage}[b]{\linewidth}\raggedright
CPMK-1
\end{minipage} & \begin{minipage}[b]{\linewidth}\raggedright
CPMK-2
\end{minipage} & \begin{minipage}[b]{\linewidth}\raggedright
CPMK-3
\end{minipage} & \begin{minipage}[b]{\linewidth}\raggedright
CPMK-4
\end{minipage} \\
\midrule\noalign{}
\endfirsthead
\toprule\noalign{}
\begin{minipage}[b]{\linewidth}\raggedright
Jenis
\end{minipage} & \begin{minipage}[b]{\linewidth}\raggedright
Asesmen
\end{minipage} & \begin{minipage}[b]{\linewidth}\raggedright
Soft Deadline Minggu ke
\end{minipage} & \begin{minipage}[b]{\linewidth}\raggedright
CPMK-1
\end{minipage} & \begin{minipage}[b]{\linewidth}\raggedright
CPMK-2
\end{minipage} & \begin{minipage}[b]{\linewidth}\raggedright
CPMK-3
\end{minipage} & \begin{minipage}[b]{\linewidth}\raggedright
CPMK-4
\end{minipage} \\
\midrule\noalign{}
\endhead
\bottomrule\noalign{}
\endlastfoot
Kuiz & Q1-Q7 & & 14 & & & \\
Kuiz & Q8-Q14 & & & & 14 & \\
UTS-1 & All About Me & 4 & & 6 & & \\
UTS-2 & My Song for You & 5 & & 7 & & \\
UTS-3 & My Stories for You & 6 & & 7 & & \\
UTS-4 & My Shape & 7 & & 6 & & \\
UTS-5 & My Personal Review & 8 & 10 & & & \\
UAS-1 & My Concepts & 12 & & & & 6 \\
UAS-2 & My Opinions & 13 & & & & 6 \\
UAS-3 & My Innovations & 14 & & & & 7 \\
UAS-4 & My Knowledge & 15 & & & & 7 \\
UAS-5 & My Professional Review & 16 & & & 10 & \\
& & & 24 & 26 & 24 & 26 \\
\end{longtable}

\begin{landscape}

\chapter{Rubrik UTS-1 All About Me}\label{rubrik-uts-1-all-about-me}

\begin{longtable}[]{@{}lllllll@{}}

\caption{\label{tbl-rubric_uts-1}Rubrik All About Me}

\tabularnewline

\toprule\noalign{}
& Kriteria & 5 - Sangat Baik & 4 - Baik & 3 - Cukup & 2 - Kurang & 1 -
Buruk \\
\midrule\noalign{}
\endhead
\bottomrule\noalign{}
\endlastfoot
0 & Orisinalitas & Narasi menghadirkan sudut pandang sangat unik ... &
Gagasan cukup orisinal dengan sedikit klise. & Beberapa unsur orisinal
namun banyak tema umum. & Prediktabel dan orisinalitas rendah. & Klise
tanpa unsur baru. \\
1 & Keterlibatan & Sangat menarik dari awal hingga akhir, menjaga... &
Umumnya menarik dengan beberapa momen kuat. & Cukup menarik; sesekali
kehilangan atensi. & Sulit mempertahankan atensi; konten kurang men... &
Tidak menarik dan tidak memikat audiens. \\
2 & Humor & Humor tepat waktu, relevan, dan efektif; serin... & Humor
baik; beberapa momen lucu. & Humor cukup; sebagian berhasil, sebagian
tidak. & Humor terasa dipaksakan/tidak tepat; jarang be... & Humor tidak
efektif atau tidak ada. \\
3 & Wawasan (Insight) & Memberi pemahaman mendalam tentang daya
tarik;... & Pesan/insight jelas meski tidak sangat mendalam. & Ada pesan
umum namun dampak terbatas. & Berusaha memberi pesan, tetapi dangkal
atau ti... & Tanpa insight bermakna tentang daya tarik inte... \\

\end{longtable}

\chapter{Rubrik UTS-2 Songs for You}\label{rubrik-uts-2-songs-for-you}

\begin{longtable}[]{@{}lllllll@{}}

\caption{\label{tbl-rubric_uts-2}Rubrik Songs For You}

\tabularnewline

\toprule\noalign{}
& Kriteria & 5 - Sangat Baik & 4 - Baik & 3 - Cukup & 2 - Kurang & 1 -
Buruk \\
\midrule\noalign{}
\endhead
\bottomrule\noalign{}
\endlastfoot
0 & Orisinalitas & Ikatan digambarkan dengan cara sangat unik dan... &
Cukup orisinal, minim klise. & Ada unsur orisinal namun banyak pola
umum. & Prediktabel; sedikit unsur baru. & Klise tanpa kebaruan. \\
1 & Keterlibatan & Sangat memikat dari awal hingga akhir. & Menarik di
sebagian besar bagian. & Cukup menarik; sesekali datar. & Kurang
memikat; banyak bagian lemah. & Tidak memikat sama sekali. \\
2 & Humor & Konsisten efektif, relevan, dan tepat waktu. & Umumnya baik;
beberapa momen berhasil. & Cukup; sebagian berhasil. & Dipaksakan/tidak
relevan; jarang berhasil. & Tidak efektif atau tidak ada. \\
3 & Inspirasi & Sangat menginspirasi; kesan mendalam tentang k... &
Cukup menginspirasi; ada momen kuat. & Ada unsur inspiratif; dampak
terbatas. & Berusaha menginspirasi namun dangkal. & Tidak
menginspirasi. \\

\end{longtable}

\chapter{Rubrik UTS-3 My Story For
You}\label{rubrik-uts-3-my-story-for-you}

\begin{longtable}[]{@{}lllllll@{}}

\caption{\label{tbl-rubric_uts-3}Rubrik My Story for You}

\tabularnewline

\toprule\noalign{}
& Kriteria & 5 - Sangat Baik & 4 - Baik & 3 - Cukup & 2 - Kurang & 1 -
Buruk \\
\midrule\noalign{}
\endhead
\bottomrule\noalign{}
\endlastfoot
0 & Orisinalitas & Pengembangan cerita sangat unik dan segar. & Lanjutan
cukup orisinal, minim klise. & Ada unsur baru namun banyak pola umum. &
Prediktabel; sedikit kebaruan. & Tidak ada pengembangan baru. \\
1 & Keterlibatan & Sangat memikat dan konsisten menjaga atensi. &
Menarik dengan beberapa jeda kecil. & Cukup menarik; ritme naik-turun. &
Kurang menarik; mudah kehilangan atensi. & Tidak menarik. \\
2 & Pengembangan Narasi & Sambung rapi dengan bagian awal; menunjukkan
p... & Terkait baik dengan bagian awal; beberapa aspe... & Melanjutkan
cerita, namun ada ketidakselarasan... & Hubungan longgar dengan bagian
awal; pengemban... & Terputus dari cerita awal; tanpa perkembangan
... \\
3 & Inspirasi & Sangat menginspirasi tentang kekuatan ikatan. & Cukup
menginspirasi; ada momen kuat. & Ada unsur inspiratif; resonansi
terbatas. & Berusaha menginspirasi tetapi dangkal. & Tidak
menginspirasi. \\

\end{longtable}

\chapter{Rubrik UTS-4 My SHAPEe}\label{rubrik-uts-4-my-shapee}

\begin{longtable}[]{@{}lllllll@{}}

\caption{\label{tbl-rubric_uts-4}Rubrik MySHAPE}

\tabularnewline

\toprule\noalign{}
& Kriteria & 5 - Sangat Baik & 4 - Baik & 3 - Cukup & 2 - Kurang & 1 -
Buruk \\
\midrule\noalign{}
\endhead
\bottomrule\noalign{}
\endlastfoot
0 & Orisinalitas & Pengembangan cerita sangat unik dan segar. & Lanjutan
cukup orisinal, minim klise. & Ada unsur baru namun banyak pola umum. &
Prediktabel; sedikit kebaruan. & Tidak ada pengembangan baru. \\
1 & Keterlibatan & Sangat memikat dan konsisten menjaga atensi. &
Menarik dengan beberapa jeda kecil. & Cukup menarik; ritme naik-turun. &
Kurang menarik; mudah kehilangan atensi. & Tidak menarik. \\
2 & Pengembangan Narasi & Sambung rapi dengan bagian awal; menunjukkan
p... & Terkait baik dengan bagian awal; beberapa aspe... & Melanjutkan
cerita, namun ada ketidakselarasan... & Hubungan longgar dengan bagian
awal; pengemban... & Terputus dari cerita awal; tanpa perkembangan
... \\
3 & Inspirasi & Sangat menginspirasi tentang kekuatan ikatan. & Cukup
menginspirasi; ada momen kuat. & Ada unsur inspiratif; resonansi
terbatas. & Berusaha menginspirasi tetapi dangkal. & Tidak
menginspirasi. \\

\end{longtable}

\chapter{Rubrik UTS-5 My Personal
Reviews}\label{rubrik-uts-5-my-personal-reviews}

\begin{longtable}[]{@{}lllllll@{}}

\caption{\label{tbl-rubric_uts-5}Rubrik My Personal Reviews}

\tabularnewline

\toprule\noalign{}
& Criterion & Level1 & Level2 & Level3 & Level4 & Level5 \\
\midrule\noalign{}
\endhead
\bottomrule\noalign{}
\endlastfoot
0 & Pemahaman Konsep Interpersonal & Tidak paham & Kurang & Cukup &
Paham & Sangat paham \& komprehensif \\
1 & Analisis Kritis Pesan & Tidak kritis & Kurang kritis & Cukup &
Kritis & Sangat kritis \& tajam \\
2 & Argumentasi (Logos) & Tidak logis & Kurang koheren & Cukup & Logis &
Sangat logis \& meyakinkan \\
3 & Etos \& Empati & Tidak tampak & Kurang & Cukup & Baik & Sangat baik
\& berimbang \\
4 & Rekomendasi Perbaikan & Tidak ada & Umum & Cukup & Konkret & Sangat
konkret \& aplikatif \\

\end{longtable}

\end{landscape}

\chapter{FORMAT LAPORAN}\label{format-laporan}

** LAPORAN PENGUKURAN BERDASARKAN RUBRIK DARI TUGAS UTS**

\section{Identifikasi}\label{identifikasi}

\begin{enumerate}
\def\labelenumi{\arabic{enumi}.}
\tightlist
\item
  Nama Mahasiswa dan NIM Penyusun TUGAS
\item
  Nama Penilai:
\end{enumerate}

\section{Tinjauan Umum}\label{tinjauan-umum}

Isi dengan tinjauan secara umum Karya yang di nilai

\section{Tinjauan Spesifik}\label{tinjauan-spesifik}

Isi dengan narasi penilaian secara khusus per UTS lalu beri hasil
detail. beri juga saran perbaikan.

\subsection{SKOR}\label{skor}

Hitung skor setiap TUGAS lalu hitung kontribusi nilai tersebut pada skor
CPMK, menurut Table~\ref{tbl-asesmen}.

\begin{longtable}[]{@{}llllll@{}}

\caption{\label{tbl-skor-akhir}Daftar Nilai}

\tabularnewline

\toprule\noalign{}
UTS & Skor & CPMK-1 & CPMK-2 & CPMK-3 & CPMK-4 \\
\midrule\noalign{}
\endhead
\bottomrule\noalign{}
\endlastfoot
UTS-1 & & & & & \\
UTS-2 & & & & & \\
UTS-3 & & & & & \\
UTS-4 & & & & & \\
UTS-5 & & & & & \\
Total & & & & & \\

\end{longtable}




\end{document}
