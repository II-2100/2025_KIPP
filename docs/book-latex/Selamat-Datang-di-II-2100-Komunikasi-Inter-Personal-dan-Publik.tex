% Options for packages loaded elsewhere
% Options for packages loaded elsewhere
\PassOptionsToPackage{unicode}{hyperref}
\PassOptionsToPackage{hyphens}{url}
\PassOptionsToPackage{dvipsnames,svgnames,x11names}{xcolor}
%
\documentclass[
  letterpaper,
  DIV=11,
  numbers=noendperiod]{scrreprt}
\usepackage{xcolor}
\usepackage{amsmath,amssymb}
\setcounter{secnumdepth}{5}
\usepackage{iftex}
\ifPDFTeX
  \usepackage[T1]{fontenc}
  \usepackage[utf8]{inputenc}
  \usepackage{textcomp} % provide euro and other symbols
\else % if luatex or xetex
  \usepackage{unicode-math} % this also loads fontspec
  \defaultfontfeatures{Scale=MatchLowercase}
  \defaultfontfeatures[\rmfamily]{Ligatures=TeX,Scale=1}
\fi
\usepackage{lmodern}
\ifPDFTeX\else
  % xetex/luatex font selection
\fi
% Use upquote if available, for straight quotes in verbatim environments
\IfFileExists{upquote.sty}{\usepackage{upquote}}{}
\IfFileExists{microtype.sty}{% use microtype if available
  \usepackage[]{microtype}
  \UseMicrotypeSet[protrusion]{basicmath} % disable protrusion for tt fonts
}{}
\makeatletter
\@ifundefined{KOMAClassName}{% if non-KOMA class
  \IfFileExists{parskip.sty}{%
    \usepackage{parskip}
  }{% else
    \setlength{\parindent}{0pt}
    \setlength{\parskip}{6pt plus 2pt minus 1pt}}
}{% if KOMA class
  \KOMAoptions{parskip=half}}
\makeatother
% Make \paragraph and \subparagraph free-standing
\makeatletter
\ifx\paragraph\undefined\else
  \let\oldparagraph\paragraph
  \renewcommand{\paragraph}{
    \@ifstar
      \xxxParagraphStar
      \xxxParagraphNoStar
  }
  \newcommand{\xxxParagraphStar}[1]{\oldparagraph*{#1}\mbox{}}
  \newcommand{\xxxParagraphNoStar}[1]{\oldparagraph{#1}\mbox{}}
\fi
\ifx\subparagraph\undefined\else
  \let\oldsubparagraph\subparagraph
  \renewcommand{\subparagraph}{
    \@ifstar
      \xxxSubParagraphStar
      \xxxSubParagraphNoStar
  }
  \newcommand{\xxxSubParagraphStar}[1]{\oldsubparagraph*{#1}\mbox{}}
  \newcommand{\xxxSubParagraphNoStar}[1]{\oldsubparagraph{#1}\mbox{}}
\fi
\makeatother


\usepackage{longtable,booktabs,array}
\usepackage{calc} % for calculating minipage widths
% Correct order of tables after \paragraph or \subparagraph
\usepackage{etoolbox}
\makeatletter
\patchcmd\longtable{\par}{\if@noskipsec\mbox{}\fi\par}{}{}
\makeatother
% Allow footnotes in longtable head/foot
\IfFileExists{footnotehyper.sty}{\usepackage{footnotehyper}}{\usepackage{footnote}}
\makesavenoteenv{longtable}
\usepackage{graphicx}
\makeatletter
\newsavebox\pandoc@box
\newcommand*\pandocbounded[1]{% scales image to fit in text height/width
  \sbox\pandoc@box{#1}%
  \Gscale@div\@tempa{\textheight}{\dimexpr\ht\pandoc@box+\dp\pandoc@box\relax}%
  \Gscale@div\@tempb{\linewidth}{\wd\pandoc@box}%
  \ifdim\@tempb\p@<\@tempa\p@\let\@tempa\@tempb\fi% select the smaller of both
  \ifdim\@tempa\p@<\p@\scalebox{\@tempa}{\usebox\pandoc@box}%
  \else\usebox{\pandoc@box}%
  \fi%
}
% Set default figure placement to htbp
\def\fps@figure{htbp}
\makeatother





\setlength{\emergencystretch}{3em} % prevent overfull lines

\providecommand{\tightlist}{%
  \setlength{\itemsep}{0pt}\setlength{\parskip}{0pt}}



 


\KOMAoption{captions}{tableheading}
\makeatletter
\@ifpackageloaded{tcolorbox}{}{\usepackage[skins,breakable]{tcolorbox}}
\@ifpackageloaded{fontawesome5}{}{\usepackage{fontawesome5}}
\definecolor{quarto-callout-color}{HTML}{909090}
\definecolor{quarto-callout-note-color}{HTML}{0758E5}
\definecolor{quarto-callout-important-color}{HTML}{CC1914}
\definecolor{quarto-callout-warning-color}{HTML}{EB9113}
\definecolor{quarto-callout-tip-color}{HTML}{00A047}
\definecolor{quarto-callout-caution-color}{HTML}{FC5300}
\definecolor{quarto-callout-color-frame}{HTML}{acacac}
\definecolor{quarto-callout-note-color-frame}{HTML}{4582ec}
\definecolor{quarto-callout-important-color-frame}{HTML}{d9534f}
\definecolor{quarto-callout-warning-color-frame}{HTML}{f0ad4e}
\definecolor{quarto-callout-tip-color-frame}{HTML}{02b875}
\definecolor{quarto-callout-caution-color-frame}{HTML}{fd7e14}
\makeatother
\makeatletter
\@ifpackageloaded{bookmark}{}{\usepackage{bookmark}}
\makeatother
\makeatletter
\@ifpackageloaded{caption}{}{\usepackage{caption}}
\AtBeginDocument{%
\ifdefined\contentsname
  \renewcommand*\contentsname{Table of contents}
\else
  \newcommand\contentsname{Table of contents}
\fi
\ifdefined\listfigurename
  \renewcommand*\listfigurename{List of Figures}
\else
  \newcommand\listfigurename{List of Figures}
\fi
\ifdefined\listtablename
  \renewcommand*\listtablename{List of Tables}
\else
  \newcommand\listtablename{List of Tables}
\fi
\ifdefined\figurename
  \renewcommand*\figurename{Figure}
\else
  \newcommand\figurename{Figure}
\fi
\ifdefined\tablename
  \renewcommand*\tablename{Table}
\else
  \newcommand\tablename{Table}
\fi
}
\@ifpackageloaded{float}{}{\usepackage{float}}
\floatstyle{ruled}
\@ifundefined{c@chapter}{\newfloat{codelisting}{h}{lop}}{\newfloat{codelisting}{h}{lop}[chapter]}
\floatname{codelisting}{Listing}
\newcommand*\listoflistings{\listof{codelisting}{List of Listings}}
\makeatother
\makeatletter
\makeatother
\makeatletter
\@ifpackageloaded{caption}{}{\usepackage{caption}}
\@ifpackageloaded{subcaption}{}{\usepackage{subcaption}}
\makeatother
\usepackage{bookmark}
\IfFileExists{xurl.sty}{\usepackage{xurl}}{} % add URL line breaks if available
\urlstyle{same}
\hypersetup{
  pdftitle={Selamat Datang di II-2100 Komunikasi Inter Personal dan Publik},
  pdfauthor={Armein Z. R. Langi},
  colorlinks=true,
  linkcolor={blue},
  filecolor={Maroon},
  citecolor={Blue},
  urlcolor={Blue},
  pdfcreator={LaTeX via pandoc}}


\title{Selamat Datang di II-2100 Komunikasi Inter Personal dan Publik}
\author{Armein Z. R. Langi}
\date{2025-09-02}
\begin{document}
\maketitle

\renewcommand*\contentsname{Table of contents}
{
\hypersetup{linkcolor=}
\setcounter{tocdepth}{2}
\tableofcontents
}

\bookmarksetup{startatroot}

\chapter*{Selamat Datang}\label{selamat-datang}
\addcontentsline{toc}{chapter}{Selamat Datang}

\markboth{Selamat Datang}{Selamat Datang}

\section*{\texorpdfstring{\textbf{SILABUS MATA
KULIAH}}{SILABUS MATA KULIAH}}\label{silabus-mata-kuliah}
\addcontentsline{toc}{section}{\textbf{SILABUS MATA KULIAH}}

\markright{\textbf{SILABUS MATA KULIAH}}

\textbf{Nama Mata Kuliah:}~Komunikasi Interpersonal dan Publik~
\textbf{Kode Mata Kuliah:}~II2100~ \textbf{SKS:}~2~
\textbf{Penyelenggara:}~Sistem dan Teknologi Informasi / STEI~
\textbf{Semester:}~Ganjil 2025/2026~\textbf{Dosen Pengampu:}~Armein Z.
R. Langi

Kuliah ONLINE Gabungan K01 dan K03 Selasa, 15:00-17:00
https://meet.google.com/afj-qgdo-wxr

II-2100 Komunikasi Interpersonal dan Publik Tuesday, 3:00 -- 5:00pm Time
zone: Asia/Jakarta Google Meet joining info Video call link:
https://meet.google.com/afj-qgdo-wxr

\textbf{Deskripsi Singkat Mata Kuliah:}~Mata kuliah ini memperkenalkan
mahasiswa pada konsep-konsep dasar komunikasi interpersonal dan publik,
meliputi komunikasi diri, proses mendengar, penggunaan bahasa verbal dan
nonverbal, serta strategi pengelolaan konflik. Selain itu, mata kuliah
ini juga membekali mahasiswa dengan kemampuan untuk mendemonstrasikan
keterampilan komunikasi interpersonal yang efektif, menjelaskan konsep
dan mendemonstrasikan keterampilan berbicara di depan umum, serta
memahami peran teknologi dalam komunikasi.

\textbf{Capaian Pembelajaran Mata Kuliah (CPMK):}~Setelah menyelesaikan
mata kuliah ini, mahasiswa diharapkan mampu:

1.~CPMK-1 \textbf{Menjelaskan konsep-konsep komunikasi interpersonal}.

2.~CPMK-2 \textbf{Mendemonstrasikan komunikasi interpersonal}~secara
efektif.

3.~CPMK-3 \textbf{Menjelaskan konsep-konsep berbicara di depan publik}.

4.~CPMK-4 \textbf{Mendemonstrasikan kemampuan berbicara di depan
publik}~dengan percaya diri.

\textbf{Metode Pembelajaran:}

•~Ceramah

•~Diskusi

•~Tugas individu dan kelompok

•~Presentasi

Manfaatkan AI seperti NotebokLM secara maksimal untuk membantu
pembelajaran.

\textbf{Modalitas Pembelajaran:}

•~Visual sinkron (kuliah tatap muka atau daring langsung)

•~Asinkron (video rekaman, materi bacaan, forum diskusi online)

\textbf{Metode Penilaian:}

•~Ujian Tengah Semester (UTS) 30\%

•~Ujian Akhir Semester (UAS) 30\%

•~Tugas (meliputi partisipasi diskusi, kuis, latihan, presentasi, esai)
40\%

\textbf{Referensi Utama:}

1.~Berko, Roy, Joan E. Aitken, and Andrew Wolvin.~\emph{ICOMM:
Interpersonal Concepts and Competencies: Foundations of Interpersonal
Communication}. Rowman \& Littlefield Publishers, Inc., 2010.

2.~Floyd, Kory.~\emph{Interpersonal Communication}. Second Edition.
McGraw-Hill, 2011.

3.~Adler, Ronald B., Lawrence B. Rosenfeld, and Russell F.
Proctor.~\emph{Interplay: The Process of Interpersonal Communication}.
14th Edition. Oxford University Press, 2018.

4.~Esenwein, J. Berg.~\emph{The Art of Public Speaking}. {[}Sumber tanpa
tanggal, digunakan sebagai referensi pendukung untuk Public Speaking{]}.

\section*{Belajar adalah Perjalanan}\label{belajar-adalah-perjalanan}
\addcontentsline{toc}{section}{Belajar adalah Perjalanan}

\markright{Belajar adalah Perjalanan}

Ada empat tahap perjalanan

\begin{itemize}
\item
  P1: Persiapan: Tahu sedang di mana, mau ke mana dan mengapa
\item
  P2: Presentasi: Penyajian pengetahuan cara mencapai tujuan: Jalan dan
  Kendaraan
\item
  P3: Praktek: latihan berkendaraan serta merencana membangun rute
  perjalanan (itenerary) hingga tiba di tujuan
\item
  P4: Perform. Setelah menetapkan rute yang hendak ditempuh beserta
  kendaraannya, itenerary ini dijalankan
\end{itemize}

\bookmarksetup{startatroot}

\chapter*{RENCANA PEMBELAJARAN SEMESTER
(RPS)}\label{rencana-pembelajaran-semester-rps}
\addcontentsline{toc}{chapter}{RENCANA PEMBELAJARAN SEMESTER (RPS)}

\markboth{RENCANA PEMBELAJARAN SEMESTER (RPS)}{RENCANA PEMBELAJARAN
SEMESTER (RPS)}

\textbf{Minggu 1: Fondasi Komunikasi Interpersonal}

•~\textbf{Bahan Kajian:}~Konsep komunikasi interpersonal.

•~\textbf{Penjelasan:}~Definisi komunikasi sebagai proses yang kompleks
dan fundamental. Model komunikasi (linear, interaksional,
transaksional), termasuk komponen-komponennya seperti partisipan,
konteks, dan tujuan. Pengenalan terhadap kompetensi komunikasi dan
pentingnya pemahaman bersama atas makna.

•~\textbf{Metode:}~Ceramah, diskusi kelompok tentang pengalaman
komunikasi.

•~\textbf{Tugas:}~Refleksi diri awal mengenai kompetensi komunikasi.

\textbf{Minggu 2: Komunikasi Diri dan Persepsi}

•~\textbf{Bahan Kajian:}~Konsep komunikasi diri.

•~\textbf{Penjelasan:}~\textbf{Komunikasi intrapersonal
(self-talk)}~sebagai dasar perencanaan, evaluasi, dan
visualisasi.~\textbf{Konsep diri (self-concept)}, bagaimana ia
berkembang, karakteristiknya, dan pengaruh~\textbf{self-fulfilling
prophecy}~terhadap komunikasi. Pembentukan skema mental untuk memproses
informasi. Peran persepsi, termasuk pengaruh otak kanan dan kiri dalam
mendengarkan dan belajar.

•~\textbf{Metode:}~Ceramah, latihan self-talk, diskusi tentang bias
persepsi.

•~\textbf{Tugas:}~Jurnal self-talk dan analisis diri.

\textbf{Minggu 3-4: Bahasa Verbal: Makna dan Pengaruh}

•~\textbf{Bahan Kajian:}~Bahasa lisan.

•~\textbf{Penjelasan:}~Bahasa bersifat simbolis, diatur oleh aturan
(fonologi, sintaksis, semantik, pragmatik), dan
subjektif.~\textbf{Hipotesis Sapir-Whorf}~yang menyatakan bahasa
membentuk pandangan dunia kita. Pengaruh bahasa terhadap identitas,
afiliasi, kekuasaan, kesopanan, seksisme, rasisme, presisi, dan tanggung
jawab. Variasi bahasa seperti dialek dan contohnya seperti Spanglish.
Pentingnya memilih kata yang tepat untuk menghindari kesalahpahaman.

•~\textbf{Metode:}~Ceramah, analisis studi kasus penggunaan bahasa,
latihan penulisan yang presisi.

•~\textbf{Tugas:}~Analisis penggunaan bahasa dalam media atau percakapan
sehari-hari.

\textbf{Minggu 5-6: Komunikasi Nonverbal: Pesan Tanpa Kata}

•~\textbf{Bahan Kajian:}~Komunikasi nonverbal.

•~\textbf{Penjelasan:}~Definisi dan karakteristik komunikasi nonverbal.
Dasar komunikasi nonverbal (program neurologis, pengaruh budaya).
Hubungan antara isyarat verbal dan nonverbal (mengganti, melengkapi).
Bentuk-bentuk komunikasi nonverbal seperti~\textbf{chronemics
(penggunaan waktu)}, termasuk budaya monokronik dan polikronik. Isyarat
nonverbal lain seperti sentuhan, bau, estetika, dan rasa. Perilaku
nonverbal immediacy.

•~\textbf{Metode:}~Ceramah, demonstrasi perilaku nonverbal, analisis
video.

•~\textbf{Tugas:}~Observasi dan analisis komunikasi nonverbal dalam
berbagai konteks.

\textbf{Minggu 7: Mendengarkan Aktif dan Empati}

•~\textbf{Bahan Kajian:}~Mendengar.

•~\textbf{Penjelasan:}~Mendengarkan sebagai keterampilan yang dipelajari
dan komponen penting dari komunikasi. Gaya mendengarkan yang berbeda
(people-oriented, action-oriented, relational, analytical, task,
critical). Proses mendengarkan yang sadar (mindful listening) dan
berbagai respons mendengarkan (stonewalling, backchanneling,
paraphrasing, empathizing, supporting, analyzing, advising).

•~\textbf{Metode:}~Ceramah, latihan mendengarkan aktif (paraphrasing),
role-playing.

•~\textbf{Tugas:}~Latihan mendengarkan dalam interaksi sehari-hari dan
menulis refleksi.

\textbf{Minggu 8: UJIAN TENGAH SEMESTER (UTS)}

•~\textbf{Cakupan Materi:}~Minggu 1-7.

\textbf{Minggu 9: Dinamika Hubungan Interpersonal}

•~\textbf{Bahan Kajian:}~Hubungan---Defined.

•~\textbf{Penjelasan:}~Definisi hubungan dan mengapa kita membentuknya.
Model tahap perkembangan hubungan (Knapp's models: coming together,
coming apart, maintenance).~\textbf{Dialektika relasional}~(misalnya,
otonomi vs.~koneksi) sebagai ketegangan alami dalam hubungan. Peran
ekspektasi dan kepuasan dalam hubungan.

•~\textbf{Metode:}~Ceramah, diskusi kelompok tentang dinamika hubungan.

•~\textbf{Tugas:}~Studi kasus dinamika hubungan.

\textbf{Minggu 10: Komunikasi dalam Hubungan Dekat (Keluarga dan
Pertemanan)}

•~\textbf{Bahan Kajian:}~Definisi keluarga.

•~\textbf{Penjelasan:}~Komunikasi dalam pertemanan (jenis pertemanan,
perbedaan gender, media sosial dalam pertemanan). Komunikasi dalam
keluarga (definisi keluarga, menciptakan keluarga melalui komunikasi,
pola komunikasi keluarga seperti orientasi percakapan dan konformitas).
Ekspresi cinta melalui ``love languages''. Hubungan romantis dan peran
komunikasi digital.

•~\textbf{Metode:}~Ceramah, berbagi pengalaman, diskusi kelompok tentang
pola komunikasi keluarga.

•~\textbf{Tugas:}~Analisis pola komunikasi dalam hubungan dekat.

\textbf{Minggu 11: Mengelola Konflik dalam Komunikasi}

•~\textbf{Bahan Kajian:}~Mengelola konflik.

•~\textbf{Penjelasan:}~Konsep dan jenis resolusi konflik (win-lose,
lose-lose, win-win). Strategi komunikasi asertif.~\textbf{Iklim
komunikasi (communication climate)}~dan perannya dalam konflik.
Mengembangkan iklim yang suportif vs.~defensif, termasuk faktor-faktor
penyebabnya. Pola demand/withdraw dalam konflik.

•~\textbf{Metode:}~Ceramah, simulasi konflik dan resolusi, latihan
membangun iklim komunikasi suportif.

•~\textbf{Tugas:}~Perencanaan strategi resolusi konflik untuk skenario
tertentu.

\textbf{Minggu 12: Fondasi Berbicara di Depan Umum}

•~\textbf{Bahan Kajian:}~Presentasi untuk khalayak umum atau luas.

•~\textbf{Penjelasan:}~Sejarah bidang komunikasi yang berfokus pada
berbicara di depan umum. Mengembangkan keberanian dan rasa percaya diri
dalam berbicara di depan umum melalui persiapan yang memadai. Pentingnya
memahami audiens dan menghindari monoton. Struktur pidato dasar
(menyatakan fakta, berargumen, seruan untuk bertindak). Prinsip ``short
speeches''.

•~\textbf{Metode:}~Ceramah, latihan singkat presentasi individu, diskusi
mengatasi kecemasan berbicara.

•~\textbf{Tugas:}~Persiapan outline pidato persuasif.

\textbf{Minggu 13: Seni Presentasi dan Bercerita (Storytelling)}

•~\textbf{Bahan Kajian:}~Presentasi untuk khalayak umum atau luas.

•~\textbf{Penjelasan:}~Membangun pidato yang menarik, memanfaatkan
elemen storytelling dalam presentasi untuk menyampaikan ide kompleks dan
membangun koneksi. Keterampilan presentasi yang efektif, termasuk
penggunaan vokal dan isyarat nonverbal dalam menyampaikan pesan di depan
publik.

•~\textbf{Metode:}~Ceramah, analisis video presentasi yang efektif,
latihan presentasi dengan fokus storytelling.

•~\textbf{Tugas:}~Menyampaikan pidato singkat di depan kelas dengan
elemen storytelling.

\textbf{Minggu 14: Teknologi Komunikasi dan Dampaknya}

•~\textbf{Bahan Kajian:}~Teknologi komunikasi.

•~\textbf{Penjelasan:}~Peran teknologi dalam komunikasi interpersonal
dan publik.~\textbf{Komunikasi bermediasi komputer (CMC)}~dan
implikasinya (online education, jejaring sosial, texting, email, IMing).
Dampak media sosial terhadap kualitas hubungan dan manajemen kesan
online. Konsep~\textbf{multimodality}~(kemampuan menggunakan berbagai
saluran komunikasi). Tantangan seperti kecanduan texting, cyberbullying,
dan phubbing.

•~\textbf{Metode:}~Ceramah, diskusi tentang etika komunikasi digital,
analisis studi kasus.

•~\textbf{Tugas:}~Esai reflektif tentang dampak teknologi pada
komunikasi personal.

\textbf{Minggu 15: Review dan Aplikasi Keterampilan}

•~\textbf{Bahan Kajian:}~Integrasi seluruh bahan kajian.

•~\textbf{Penjelasan:}~Sesi ini didedikasikan untuk meninjau kembali
semua konsep dan keterampilan yang telah dipelajari. Mahasiswa akan
mendapatkan kesempatan untuk mempraktikkan keterampilan komunikasi
interpersonal dan publik secara terintegrasi melalui simulasi atau
presentasi akhir.

•~\textbf{Metode:}~Diskusi terbuka, latihan simulasi, presentasi akhir
(jika dialokasikan).

•~\textbf{Tugas:}~Persiapan UAS, menyelesaikan tugas akhir (jika ada).

\textbf{Minggu 16: UJIAN AKHIR SEMESTER (UAS)}

•~\textbf{Cakupan Materi:}~Minggu 1-14, dengan penekanan pada materi
setelah UTS.

\bookmarksetup{startatroot}

\chapter*{Asesmen Pencapaian CPMK}\label{asesmen-pencapaian-cpmk}
\addcontentsline{toc}{chapter}{Asesmen Pencapaian CPMK}

\markboth{Asesmen Pencapaian CPMK}{Asesmen Pencapaian CPMK}

\begin{tcolorbox}[enhanced jigsaw, breakable, colback=white, leftrule=.75mm, opacitybacktitle=0.6, colframe=quarto-callout-warning-color-frame, title=\textcolor{quarto-callout-warning-color}{\faExclamationTriangle}\hspace{0.5em}{Warning}, coltitle=black, left=2mm, arc=.35mm, bottomrule=.15mm, toprule=.15mm, bottomtitle=1mm, colbacktitle=quarto-callout-warning-color!10!white, toptitle=1mm, titlerule=0mm, rightrule=.15mm, opacityback=0]

Petunjuk asesmen ini masih dalam pengerjaan, sehingga perlu dikunjungi
secara berkala untuk mendapatkan petunjuk terkini.

\end{tcolorbox}

Pencapaian tujuan komunikasi akan semakin berhasil bila kita memiliki
daya tarik. Sebaliknya juga berlaklu bahwa komunikasi bermaksud
memperkuat daya tarik.

Dalam kuliah ini, daya tarik personal timbul apabila kita
mengkomunikasikan sosok yang interesting, cerdas, dan berpengaruh. Cara
mengkomunikasikannya adalah dengan menyampaikan pengalaman hidup kita,
pemahaman kita, pendapat kita, dan karya kitga. Tugas-tugas ini
dimaksudkan untuk berlatih mengekspresikan daya tarik.

\section*{Konsep Penguasaan
Kompetensi}\label{konsep-penguasaan-kompetensi}
\addcontentsline{toc}{section}{Konsep Penguasaan Kompetensi}

\markright{Konsep Penguasaan Kompetensi}

Berikut adalah konsep asesmen untuk mata kuliah II 2100 Komunikasi
Interpersonal dan Publik, yang dirancang untuk mengukur pencapaian
Capaian Pembelajaran Mata Kuliah (CPMK) secara komprehensif, dilengkapi
dengan rubrik lima level pencapaian.

Mata kuliah II 2100 ``Komunikasi Interpersonal dan Publik'' bertujuan
untuk membekali mahasiswa dengan kemampuan

\begin{enumerate}
\def\labelenumi{\arabic{enumi}.}
\item
  \textbf{menjelaskan konsep-konsep komunikasi interpersonal (CPMK-1)},
\item
  \textbf{mendemonstrasikan komunikasi interpersonal secara efektif
  (CPMK-2)},
\item
  \textbf{menjelaskan konsep-konsep berbicara di depan publik (CPMK-3)},
  dan
\item
  \textbf{mendemonstrasikan kemampuan berbicara di depan publik dengan
  percaya diri (CPMK-4)}.
\end{enumerate}

Filosofi dasar perkuliahan ini adalah bahwa komunikasi pribadi berperan
penting dalam \textbf{membentuk konsep diri yang otentik,
mengekspresikan diri untuk daya tarik pribadi, serta membangun dan
mengembangkan relasi yang berhasil dengan \emph{significant other}}.
Relasi pribadi ini kemudian menjadi \textbf{dasar untuk berkomunikasi
secara publik}, yang menuntut \textbf{otentisitas, kepercayaan,
ketulusan, kejujuran, kompetensi, dan logika gagasan} (yang dikenal
sebagai \emph{pathos, ethos}, dan \emph{logos}) pada pendengar.

Di era Artificial Intelligence (AI), mahasiswa dianjurkan memanfaatkan
teknologi secara maksimal.

\section*{Bukti Kompetensi}\label{bukti-kompetensi}
\addcontentsline{toc}{section}{Bukti Kompetensi}

\markright{Bukti Kompetensi}

\textbf{Bentuk-bentuk Asesmen:} Untuk mengukur CPMK tersebut, asesmen
akan menggunakan beberapa bentuk penilaian, yaitu:

\begin{enumerate}
\def\labelenumi{\arabic{enumi}.}
\tightlist
\item
  \textbf{Kuis Materi Topik Setiap Minggu (Q-1 s/d Q-14):} Mengukur
  pemahaman konseptual dari 14 topik perkuliahan yang disampaikan setiap
  minggu.
\item
  \textbf{Ujian Tengah Semester (UTS-1 s/d UTS-5):} Fokus pada proyek
  demonstrasi komunikasi personal

  \begin{enumerate}
  \def\labelenumii{\arabic{enumii}.}
  \tightlist
  \item
    UTS-1 All About Me, berisikan pesan yang memperkenalkan sosok diri
    kita
  \item
    UTS-2 Song for you, berisikan pesan berbentuk puisi, lago, dan/atau
    viodeo clip{[}
  \item
    UTS-3 My Stories for You, berisikan kisah inspiratif dan menarik
    yang Anda ingin bagikan dengan pribadi lain
  \item
    UTS-4 My Shape, berisikan laporan siapa Anda berdasar hasil sebuah
    lembar kerja
  \item
    UTS-5 My Personal Review, berisikan telahan pesan personal
    berdasarkan rubrik
  \end{enumerate}
\item
  \textbf{Ujian Akhir Semester (UAS-1 s/d UAS-5):} Fokus pada proyek
  demonstrasi komunikasi inspiratif publik yang mengaplikasikan konsep
  interpersonal.

  \begin{enumerate}
  \def\labelenumii{\arabic{enumii}.}
  \tightlist
  \item
    UAS-1 My Concepts,
  \item
    UAS-2 My Opinions,
  \item
    UAS-3 My Innovations, berisikan desain suatu produk atau layanan
    yang membangun kapasitas dan efektivitas
  \item
    UAS-4 My Knowledge, berisikan pengetahuan dan pembelajaran bagi
    masyarakat atas suatu topik dalam kuliah ini
  \item
    UAS-5 My Professional Reviews, berisikan telaahan pesan publikl
    berdasarkan rubrik.
  \end{enumerate}
\end{enumerate}

\begin{longtable}[]{@{}
  >{\raggedright\arraybackslash}p{(\linewidth - 12\tabcolsep) * \real{0.1429}}
  >{\raggedright\arraybackslash}p{(\linewidth - 12\tabcolsep) * \real{0.1429}}
  >{\raggedright\arraybackslash}p{(\linewidth - 12\tabcolsep) * \real{0.1429}}
  >{\raggedright\arraybackslash}p{(\linewidth - 12\tabcolsep) * \real{0.1429}}
  >{\raggedright\arraybackslash}p{(\linewidth - 12\tabcolsep) * \real{0.1429}}
  >{\raggedright\arraybackslash}p{(\linewidth - 12\tabcolsep) * \real{0.1429}}
  >{\raggedright\arraybackslash}p{(\linewidth - 12\tabcolsep) * \real{0.1429}}@{}}
\caption{Tabel daftar bentuk asesmen, jadwal soft deadline, CPMK yang
diukur serta bobot peniliaian dalam skala
100.}\label{tbl-asesmen}\tabularnewline
\toprule\noalign{}
\begin{minipage}[b]{\linewidth}\raggedright
Jenis
\end{minipage} & \begin{minipage}[b]{\linewidth}\raggedright
Asesmen
\end{minipage} & \begin{minipage}[b]{\linewidth}\raggedright
Soft Deadline Minggu ke
\end{minipage} & \begin{minipage}[b]{\linewidth}\raggedright
CPMK-1
\end{minipage} & \begin{minipage}[b]{\linewidth}\raggedright
CPMK-2
\end{minipage} & \begin{minipage}[b]{\linewidth}\raggedright
CPMK-3
\end{minipage} & \begin{minipage}[b]{\linewidth}\raggedright
CPMK-4
\end{minipage} \\
\midrule\noalign{}
\endfirsthead
\toprule\noalign{}
\begin{minipage}[b]{\linewidth}\raggedright
Jenis
\end{minipage} & \begin{minipage}[b]{\linewidth}\raggedright
Asesmen
\end{minipage} & \begin{minipage}[b]{\linewidth}\raggedright
Soft Deadline Minggu ke
\end{minipage} & \begin{minipage}[b]{\linewidth}\raggedright
CPMK-1
\end{minipage} & \begin{minipage}[b]{\linewidth}\raggedright
CPMK-2
\end{minipage} & \begin{minipage}[b]{\linewidth}\raggedright
CPMK-3
\end{minipage} & \begin{minipage}[b]{\linewidth}\raggedright
CPMK-4
\end{minipage} \\
\midrule\noalign{}
\endhead
\bottomrule\noalign{}
\endlastfoot
Kuiz & Q1-Q7 & & 14 & & & \\
Kuiz & Q8-Q14 & & & & 14 & \\
UTS-1 & All About Me & 4 & & 6 & & \\
UTS-2 & My Song for You & 5 & & 7 & & \\
UTS-3 & My Stories for You & 6 & & 7 & & \\
UTS-4 & My Shape & 7 & & 6 & & \\
UTS-5 & My Personal Review & 8 & 10 & & & \\
UAS-1 & My Concepts & 12 & & & & 6 \\
UAS-2 & My Opinions & 13 & & & & 6 \\
UAS-3 & My Innovations & 14 & & & & 7 \\
UAS-4 & My Knowledge & 15 & & & & 7 \\
UAS-5 & My Professional Review & 16 & & & 10 & \\
& & & 24 & 26 & 24 & 26 \\
\end{longtable}

\section*{Bukti Pencapaian Beserta Rubrik
Terkait}\label{bukti-pencapaian-beserta-rubrik-terkait}
\addcontentsline{toc}{section}{Bukti Pencapaian Beserta Rubrik Terkait}

\markright{Bukti Pencapaian Beserta Rubrik Terkait}

\subsection*{Kuis Materi Topik Setiap Minggu (Q-1 s/d
Q-7)}\label{kuis-materi-topik-setiap-minggu-q-1-sd-q-7}
\addcontentsline{toc}{subsection}{Kuis Materi Topik Setiap Minggu (Q-1
s/d Q-7)}

Pada materi setiap minggu, terdapat Kuiz yang perlu dikerjakan melalui
Situs Microsoft Form.

\subsection*{Kuis Materi Topik Setiap Minggu (Q-8 s/d
Q-14)}\label{kuis-materi-topik-setiap-minggu-q-8-sd-q-14}
\addcontentsline{toc}{subsection}{Kuis Materi Topik Setiap Minggu (Q-8
s/d Q-14)}

Pada materi setiap minggu, terdapat Kuiz yang perlu dikerjakan melalui
Situs Microsoft Form.

\subsection*{UTS-1 All About Me}\label{uts-1-all-about-me}
\addcontentsline{toc}{subsection}{UTS-1 All About Me}

Berisikan pesan yang memperkenalkan sosok diri kita. Meskipun komunikasi
bhersifat pribadi, hendaknya informasi yang sensitif atau yang bersifat
rahasia bank (seperti tanggal lahir dan nama ibu kandung) tidak
ditayangkan

\href{./asesmen/Rubrik_II-2100_CSV/UTS-1.csv}{Rubrik UTS-1: Daya Tarik
(Attractiveness)}

\subsection*{UTS-2 Songs for you}\label{uts-2-songs-for-you}
\addcontentsline{toc}{subsection}{UTS-2 Songs for you}

Berisikan pesan berbentuk puisi, lagu, dan/atau video clip untuk ia atau
mereka yang kita anggap relasi intim. Penggunakan AI akan mempercepat
proses pembuatannya, namun otensitas dan signature pribadi sangat
penting untuk keberhasilan pesan ini.

\href{./asesmen/Rubrik_II-2100_CSV/UTS-2.csv}{Rubrik UTS-2: Song For
You}

\subsection*{UTS-3 My Stories for You,}\label{uts-3-my-stories-for-you}
\addcontentsline{toc}{subsection}{UTS-3 My Stories for You,}

Berisikan kisah inspiratif dan menarik yang Anda ingin bagikan dengan
pribadi lain.

\href{./asesmen/Rubrik_II-2100_CSV/UTS-4.csv}{Rubrik UTS-3: My Story for
You}

\subsection*{UTS-4 My Shape,}\label{uts-4-my-shape}
\addcontentsline{toc}{subsection}{UTS-4 My Shape,}

Berisikan laporan siapa Anda berdasar asesmen VIA, Piagam Diri dan
Identitas Naratif, hasil sebuah lembar kerja
\href{asesmen/UTS-4-My-SHAPE/my_shape_short_2.pdf}{My SHAPE}

\href{./asesmen/Rubrik_II-2100_CSV/UTS-4.csv}{Rubrik UTS-4: My SHAPE}

\subsection*{UTS-5 My Personal Review,}\label{uts-5-my-personal-review}
\addcontentsline{toc}{subsection}{UTS-5 My Personal Review,}

Berisikan telahan pesan personal UTS-1 s/d UTS-4 berdasarkan rubrik
masing-masing. Anda diminta melakukan Self Assesment dan Peer Assessment
menggunakan rubrik yang ada.

Self Assessment menggunakan AI, sedangkan Peer Assessment dilakukan
manual

\begin{tcolorbox}[enhanced jigsaw, breakable, colback=white, leftrule=.75mm, opacitybacktitle=0.6, colframe=quarto-callout-warning-color-frame, title=\textcolor{quarto-callout-warning-color}{\faExclamationTriangle}\hspace{0.5em}{Penugasan Peer Assignmen Baru Ada Nanti Saat UTS}, coltitle=black, left=2mm, arc=.35mm, bottomrule=.15mm, toprule=.15mm, bottomtitle=1mm, colbacktitle=quarto-callout-warning-color!10!white, toptitle=1mm, titlerule=0mm, rightrule=.15mm, opacityback=0]

Daftar URL untuk Peer Assessment baru akan diperoleh saat UTS. Jadi
meskipun Anda bisa mencoba sebelum UTS, pastikan yang anda laporkan
sesuai penugasan saat UTS.

\end{tcolorbox}

Lakukan:

\begin{enumerate}
\def\labelenumi{\arabic{enumi}.}
\tightlist
\item
  Self-Assessment menggunakan AI (Skor 2)
\item
  Peer-Assessment 1 Tanpa AI (Skor 4)
\item
  Peer-Assessment 2 Tanpa AI (Skor 4)
\item
  Peer-Assessment 3 Tanpa AI (Skor 4, bonus)
\end{enumerate}

Skor Penilaian diisi ke dalam file excel
\href{./asesmen/UTS-5_Skor.xlsx}{Lembar Skor}. Isi skor keempat asesmen
ini dalam file yang sama, masing-masing menggunakan satu baris, dengan
urutan di atas. Berarti dimulai dengan baris berisi data Anda. Hasil ini
disimpan di repositori anda di folder UTS-5, link kan ke portal anda
pada bagian UTS-5, LALU LAPORKAN LINK FILE TERSEBU KE MS FORM UTS-1
\url{https://forms.cloud.microsoft/r/tF4pm04jUT}

\subsection*{UAS-1 My Concepts,}\label{uas-1-my-concepts}
\addcontentsline{toc}{subsection}{UAS-1 My Concepts,}

Berisikan suatu penjelasan atas suatu fenomena yang penting, dengan
tujuan untuk mengkomunikasikan keistimewaan pengetahuan yang kita
miliki. Penjelasan singkat apa itu konsep dapat di lihat di sini:
\url{https://youtu.be/DxK-_0AzHiw?si=5lpAur3wWtiOyUDV}

\subsection*{UAS-2 My Opinions,}\label{uas-2-my-opinions}
\addcontentsline{toc}{subsection}{UAS-2 My Opinions,}

Berisikan suatu pandangan mengenai bagaimana sikap, keputusan, atau
tindakan terbaik yang harus diambil dalam menghadapiu suatu kondisi atau
perkembangan

\subsection*{UAS-3 My Innovations}\label{uas-3-my-innovations}
\addcontentsline{toc}{subsection}{UAS-3 My Innovations}

Berisikan desain suatu produk atau layanan yang membangun kapasitas dan
efektivitas berkomunikasi KIPP

\subsection*{UAS-4 My Knowledge,}\label{uas-4-my-knowledge}
\addcontentsline{toc}{subsection}{UAS-4 My Knowledge,}

Berisikan pengetahuan dan pembelajaran bagi masyarakat atas suatu topik
dalam kuliah ini.

\subsection*{UAS-5 My Professional
Reviews}\label{uas-5-my-professional-reviews}
\addcontentsline{toc}{subsection}{UAS-5 My Professional Reviews}

Berisikan telaahan pesan publik berdasarkan rubrik. \#\# Teknis
Pelaksanaan, Pengumpulan, dan Penilaian

\subsection*{Repositori Github}\label{repositori-github}
\addcontentsline{toc}{subsection}{Repositori Github}

Dianjurkan setiap peserta memanfaatkan Github untuk meng-host portfolio
karya laporan tugas. Untuk itu setiap peserta dapat mem-fork repositori
\href{https://github.com/II-2100/all-about-me}{contoh} dan menggunakan
nya sebagai template \url{https://II-2100.github.io/all-about-me}.
Meskpun demikian, peserta dapat menggunakan platform lain untuk
menyimpan hasil tugas nya.

Pada Table~\ref{tbl-asesmen} di cantumkan juga soft-deadline, yaitu
jadwal pengumpulan tugas. Hard deadline UTS dan UAS masing-masing ada
pada UTS-5 dan UAS-10., berarti semua tugas sudah harus tersubmit
sebelum hard-deadline

Link ke halaman utama dari tugas Anda hendaknya dilaporkan di
\url{https://forms.office.com/r/pbUdPnGN3h}. Sebelum dilaporkan,
pastikan

\section*{Hubungan dengan CPMK}\label{hubungan-dengan-cpmk}
\addcontentsline{toc}{section}{Hubungan dengan CPMK}

\markright{Hubungan dengan CPMK}

Singkatnya: \textbf{skor pada rubrik = bukti ketercapaian indikator
tugas}, sedangkan \textbf{CPMK = capaian akhir pembelajaran mata
kuliah}. Jadi, skor rubrik tiap tugas Anda dipetakan ke CPMK yang
relevan lalu diagregasi (dengan bobot) untuk menilai apakah CPMK ``Tidak
tercapai'' s.d. ``Melampaui''.

Berikut cara praktisnya (disesuaikan dengan laman kuliah II-2100):

\subsection*{1) Peta Tugas → CPMK}\label{peta-tugas-cpmk}
\addcontentsline{toc}{subsection}{1) Peta Tugas → CPMK}

Di RPS/Asesmen situs kuliah, CPMK yang dinilai adalah:

\begin{itemize}
\tightlist
\item
  \textbf{CPMK-1}: Menjelaskan konsep komunikasi interpersonal.
\item
  \textbf{CPMK-2}: Mendemonstrasikan komunikasi interpersonal efektif.
\item
  \textbf{CPMK-3}: Menjelaskan konsep public speaking.
\item
  \textbf{CPMK-4}: Mendemonstrasikan public speaking percaya diri.
  ({[}ii-2100.github.io{]}{[}1{]})
\end{itemize}

Bagian ``Asesmen Terkait'' menunjukkan evidence untuk tiap CPMK (kuis,
UTS, UAS). Intinya:

\begin{itemize}
\tightlist
\item
  \textbf{CPMK-1} dinilai terutama lewat pengembangan dan evaluasi
  konsep pada UAS, plus kuis topik. ({[}ii-2100.github.io{]}{[}2{]})
\item
  \textbf{CPMK-2} dinilai lewat proyek UTS (pesan/presentasi personal)
  dan sebagian UAS (aplikasi interpersonal).
  ({[}ii-2100.github.io{]}{[}2{]})
\item
  \textbf{CPMK-3} dinilai lewat materi/kuis public speaking dan bagian
  konsep/opini di UAS. ({[}ii-2100.github.io{]}{[}2{]})
\item
  \textbf{CPMK-4} dinilai lewat UTS \& UAS bagian performa
  (presentasi/video). ({[}ii-2100.github.io{]}{[}2{]})
\end{itemize}

Dengan versi tugas Anda (tanpa ``Personal/Professional Review''), contoh
peta sederhananya:

\begin{itemize}
\tightlist
\item
  \textbf{UTS-1 (Attractiveness)} → CPMK-2, CPMK-4
\item
  \textbf{UTS-2 (Bonding)} → CPMK-2
\item
  \textbf{UTS-3 (Bonding -- lanj.)} → CPMK-2
\item
  \textbf{UTS-4 (Authentic Self Discovery)} → CPMK-2 (dan CPMK-4 jika
  ada performa/presentasi)
\item
  \textbf{UAS-1 (Concept Explanation)} → CPMK-1, CPMK-3
\item
  \textbf{UAS-2 (Opinion Essay)} → CPMK-3 (argumentasi/retorika), bisa
  mendukung CPMK-1 minor
\item
  \textbf{UAS-3 (Presentation)} → CPMK-4 (dan CPMK-3 untuk
  struktur/retorika)
\item
  \textbf{UAS-4 (Innovation)} → CPMK-4 (presentasi/pitch), dapat
  menyokong CPMK-1/3 jika memuat konsep \& argumen
\end{itemize}

Catatan: Di halaman ``Asesmen Pencapaian CPMK'' juga ada tabel yang
menandai \textbf{tiap jenis asesmen dan CPMK yang diukur} (Table 1).
Gunakan itu sebagai rujukan saat menetapkan bobot kontribusi tugas ke
CPMK. ({[}ii-2100.github.io{]}{[}2{]})

\subsection*{2) Dari Skor Rubrik → Skor CPMK (langkah
perhitungan)}\label{dari-skor-rubrik-skor-cpmk-langkah-perhitungan}
\addcontentsline{toc}{subsection}{2) Dari Skor Rubrik → Skor CPMK
(langkah perhitungan)}

\textbf{Langkah A --- Hitung skor tugas} Ambil \textbf{rata-rata skor
rubric 1--5} per tugas (atau boboti per kriteria kalau diinginkan).

\textbf{Langkah B --- Normalisasi (opsional tapi rapi)}

\begin{itemize}
\tightlist
\item
  Ke \textbf{0--100}:
  \texttt{Nilai\%\ =\ (SkorRata2\ −\ 1)\ /\ 4\ ×\ 100}.
\end{itemize}

\textbf{Langkah C --- Agregasi ke CPMK}

\begin{itemize}
\tightlist
\item
  Tetapkan \textbf{bobot kontribusi} setiap tugas ke CPMK yang relevan
  (mis. UAS-3 mungkin menyumbang 40\% ke CPMK-4; UTS-1 20\%;
  dst---sesuaikan kebijakan Anda).
\item
  \textbf{Skor CPMK-k = Σ (bobot\_tugas → CPMK-k ×
  skor\_tugas\_terkait)}.
\end{itemize}

\textbf{Langkah D --- Pemetaan ke Level CPMK (0--4)} Situs menegaskan
level CPMK: \textbf{0 Tidak tercapai, 1 Kurang, 2 Hampir, 3 Tercapai, 4
Melampaui} (dengan deskripsi rinci per CPMK). Anda bisa memetakan nilai
akhir CPMK (0--100) ke level, misalnya:

\begin{itemize}
\tightlist
\item
  \textless40 → \textbf{0}, 40--54.9 → \textbf{1}, 55--69.9 →
  \textbf{2}, 70--84.9 → \textbf{3}, ≥85 → \textbf{4}. Atau
  \textbf{langsung} dari skala 1--5:
  \texttt{Level\ ≈\ round(rata-rata\ rubrik)\ −\ 1} (dibatasi 0--4).
  Pilih satu metode dan konsisten. ({[}ii-2100.github.io{]}{[}2{]})
\end{itemize}

\subsection*{\texorpdfstring{3) Hubungan dengan \textbf{nilai mata
kuliah}
(UTS/UAS/Tugas)}{3) Hubungan dengan nilai mata kuliah (UTS/UAS/Tugas)}}\label{hubungan-dengan-nilai-mata-kuliah-utsuastugas}
\addcontentsline{toc}{subsection}{3) Hubungan dengan \textbf{nilai mata
kuliah} (UTS/UAS/Tugas)}

Ini soal \textbf{grade} keseluruhan, terpisah dari level CPMK tetapi
memakai evidence yang sama. Di RPS: \textbf{UTS 30\% + UAS 30\% + Tugas
40\%} → Nilai akhir. CPMK memakai bukti yang sama namun dilaporkan per
capaian (0--4) sesuai deskripsi CPMK. ({[}ii-2100.github.io{]}{[}1{]})

\begin{center}\rule{0.5\linewidth}{0.5pt}\end{center}

\subsection*{Contoh mini (ilustratif)}\label{contoh-mini-ilustratif}
\addcontentsline{toc}{subsection}{Contoh mini (ilustratif)}

\begin{itemize}
\tightlist
\item
  UAS-3 (Presentation) rata-rata rubrik = \textbf{4.2/5} →
  \textbf{80\%}.
\item
  Misal bobot ke \textbf{CPMK-4} = 40\%; UTS-1 ke CPMK-4 = 20\% (skor
  3.8 → 70\%); UTS-4 ke CPMK-4 = 40\% (skor 4.6 → 90\%).
\item
  \textbf{CPMK-4} = 0.4×80 + 0.2×70 + 0.4×90 = \textbf{84\%} →
  \textbf{Level 3 (Tercapai)}.
\end{itemize}

\begin{center}\rule{0.5\linewidth}{0.5pt}\end{center}

\begin{tcolorbox}[enhanced jigsaw, breakable, colback=white, leftrule=.75mm, opacitybacktitle=0.6, colframe=quarto-callout-important-color-frame, title=\textcolor{quarto-callout-important-color}{\faExclamation}\hspace{0.5em}{Important}, coltitle=black, left=2mm, arc=.35mm, bottomrule=.15mm, toprule=.15mm, bottomtitle=1mm, colbacktitle=quarto-callout-important-color!10!white, toptitle=1mm, titlerule=0mm, rightrule=.15mm, opacityback=0]

Petunjuk:

\begin{enumerate}
\def\labelenumi{\arabic{enumi}.}
\item
  Baca petunjuk ini
\item
  Fork repositori \href{https://github.com/II-2100/all-about-me.git}{All
  About Me} atau \url{https://github.com/II-2100/all-about-me.git}
\item
  simpan hasil laporan tugas anda pada repositori tersebut
\item
  publish bagian yang hendak di asses melalui portal page
\item
  submit URL portal page ke URL Submisi:
  \url{https://forms.office.com/r/pbUdPnGN3h}
\item
  Self Assessment. Karya Anda akan dinilai menggunakan
  \href{asesmen/skor_uts.pdf}{promt ChatGPT}. Pastikan Anda sudah
  mencoba sendiri . Berikut cara saya melakukan review: menggunakan
  chatGPT, saya mengattach \href{asesmen/skor_uts.pdf}{promt ChatGPT},
  disertai perintah :``self assess uts-1 sanpai uts-5 dari URL
  `https://ii-2100.github.io/all-about-me/'\,''
\end{enumerate}

ChatGPT melakukan self-assessment UTS-1 s.d. UTS-5 langsung dari laman
yang Anda berikan dan menilai memakai rubrik tugas UTS (skala 1--5 per
kriteria). Rekap skor siap diunduh sebagai CSV:
\href{sandbox:/mnt/data/UTS_self_assessment.csv}{Download CSV
ringkasan}.

\end{tcolorbox}

\bookmarksetup{startatroot}

\chapter{Kuliah 1: Komunikasi
Interpersonal}\label{kuliah-1-komunikasi-interpersonal}

\href{https://youtube.com/playlist?list=PL_m-BplfO92EHtF458s1yWRf6u0lGZQaZ&si=hAc9s5o7a4aHXJFO}{Video
Clips}

\href{https://forms.office.com/r/WCHBJJiwXc}{Quiz}

\section{Ringkasan Materi}\label{ringkasan-materi}

Materi ini membahas berbagai aspek komunikasi interpersonal, dimulai
dari tujuan dan pemicunya, hingga karakteristik kompetensi, peran media
sosial, dan strategi pengembangan diri. Komunikasi interpersonal
dipandang sebagai tuntutan kehidupan, jalan keluar dari masalah, dan
cara untuk mengembangkan kualitas hidup.

\url{https://youtu.be/hGrXqna6gJg?si=H3Zr-1hsT3g0Pgpt} \#\#\# Sesi 1:
Model Komunikasi Antar Pribadi

\begin{itemize}
\tightlist
\item
  \textbf{Tujuan Komunikasi:} Mendapatkan pengertian, persetujuan, dan
  kemufakatan.
\item
  \textbf{Pemicu dan Pemicu Perubahan:} Komunikasi memicu dan dipicu
  oleh perubahan dalam empat aspek:
\end{itemize}

\begin{enumerate}
\def\labelenumi{\arabic{enumi}.}
\tightlist
\item
  \textbf{Situasi:} Peristiwa eksternal.
\item
  \textbf{Kondisi:} Kondisi internal yang dipengaruhi situasi.
\item
  \textbf{Prospek:} Prediksi masa depan (positif akan dijaga, negatif
  akan memicu aksi).
\item
  \textbf{Komunikasi:} Aksi yang paling mudah dilakukan.
\end{enumerate}

\begin{itemize}
\tightlist
\item
  \textbf{Pihak yang Terlibat:} Aku (pihak pertama), Engkau (pihak
  kedua), Dia/Mereka, dan Kita. Komunikasi bertujuan menghasilkan nilai
  yang menguntungkan bagi keempat pihak.
\item
  \textbf{Repertoar Komunikasi:} Bahan persediaan ucapan atau lakon yang
  perlu disiapkan untuk berbagai kasus, seperti menolong tetangga
  kecelakaan, menghadapi polisi, menghibur kakak yang di-PHK, wawancara
  kerja, menagih uang kos, atau menyelesaikan tugas kuliah.
\item
  \textbf{Peran Komunikasi Interpersonal:} Tuntutan kehidupan, jalan
  keluar dari masalah, dan pengembangan kualitas hidup.
\end{itemize}

\subsection{Sesi 2: Karakteristik Kompetensi Komunikasi
Interpersonal}\label{sesi-2-karakteristik-kompetensi-komunikasi-interpersonal}

\url{https://youtu.be/A4ewFqIeUEE?si=-QA3wXBa9lGLSEpX}

\begin{itemize}
\tightlist
\item
  \textbf{Model Komunikasi Transaksional:} Pertukaran pesan interaktif
  dan adaptif antara dua belah pihak. Pesan dikemas oleh A, dikirim
  melalui saluran, dan di-``unboxing'' oleh B, seringkali dibalas oleh
  B.
\item
  \textbf{Tujuan Komunikasi:Personal:} Membangun relasi, menggunakan
  bahasa unik yang diciptakan relasi dekat untuk efisiensi.
\item
  \textbf{Impersonal:} Tujuan praktis (misalnya melamar pekerjaan),
  menggunakan bahasa yang dimengerti umum.
\item
  \textbf{Perbedaan Personal/Impersonal dengan
  Private/Publik:Personal/Impersonal:} Terkait dengan relasi.
\item
  \textbf{Private:} Tidak ada yang tahu kecuali komunikator.
\item
  \textbf{Publik:} Semua orang tahu.
\item
  \textbf{Repertoar sebagai Desain Pesan:} Dengan modal model-model ini,
  kita bisa merancang repertoar (persediaan bahan berbicara) untuk
  memecahkan masalah dan memenuhi kebutuhan, termasuk merancang pesan
  dan panggung delivery-nya.
\end{itemize}

\subsection{Sesi 3: Bisnis Konten Komunikasi
Interpersonal}\label{sesi-3-bisnis-konten-komunikasi-interpersonal}

\url{https://youtu.be/bRsTpwSsgc8?si=IW9dqY3APcyMdf4K}

\begin{enumerate}
\def\labelenumi{\arabic{enumi}.}
\tightlist
\item
  \textbf{Karakteristik Kompetensi Berkomunikasi:Koleksi Repertoar:}
  Banyak bahan, pesan, dan keterampilan menyampaikan (misalnya koleksi
  lagu dan cara menyanyikannya).
\item
  \textbf{Adaptabilitas:} Mampu merespons reaksi audiens.
\item
  \textbf{Keterampilan (Skill):} Mampu tampil dengan terampil.
\item
  \textbf{Empati:} Mampu mengambil sudut pandang pihak lain.
\item
  \textbf{Kompleksitas Kognitif:} Kecerdasan berpikir meliputi:
\end{enumerate}

\begin{itemize}
\tightlist
\item
  \textbf{Perceiving:} Berpersepsi.
\item
  \textbf{Thinking:} Mengolah informasi.
\item
  \textbf{Knowing:} Mengingat sesuatu.
\item
  \textbf{Remembering:} Mengingat (mirip knowing).
\item
  \textbf{Judging:} Membuat keputusan.
\item
  \textbf{Problem Solving:} Memecahkan masalah.
\end{itemize}

\begin{enumerate}
\def\labelenumi{\arabic{enumi}.}
\tightlist
\item
  \textbf{Refleksi Diri (Self-assessment):} Untuk mengetahui seberapa
  kompeten kita berkomunikasi.
\end{enumerate}

\subsection{Sesi 4: Teknologi Digital dan Komunikasi
Interpersonal}\label{sesi-4-teknologi-digital-dan-komunikasi-interpersonal}

\url{https://youtu.be/YewAjHFz_v4?si=7XAtZJhX2g1ExlRV}

\begin{itemize}
\tightlist
\item
  \textbf{Peran Media Sosial:} Memfasilitasi dan memperkuat komunikasi
  interpersonal. Mengatasi kelemahan individu dengan ``social power''
  (kekuatan sosial, analogi paduan suara).
\item
  \textbf{Kekuatan Media Sosial:} Masukan dari banyak orang, interaksi,
  sharing konten, kolaborasi.
\item
  \textbf{Analogi Komunikasi Online vs.~Offline:Interaksi Langsung
  (Offline):} Seperti tenis (harus balas real-time).
\item
  \textbf{Media Sosial (Online):} Seperti catur (bisa menunda jawaban
  sampai siap).
\item
  \textbf{Kaya Nuansa vs.~Ramping (Lean):Kaya Nuansa:} Komunikasi tatap
  muka (suara, gambar, gerak-gerik).
\item
  \textbf{Ramping (Lean):} Texting, email, online post (kata-kata
  ramping, miskin bobot nuansa).
\item
  \textbf{Potensi Media Sosial:}Memperkaya dan memenuhi kebutuhan relasi
  kapan saja, di mana saja.
\item
  Meningkatkan jumlah dan kualitas interaksi.
\item
  Relatif mudah dan tidak rumit.
\item
  \textbf{Risiko Media Sosial:}Miskin nuansa, dangkal, basa-basi.
\item
  Spontan dan singkat, tidak hangat (suam-suam kuku).
\item
  Efimeral (singkat), seringkali ilusi belaka.
\item
  \textbf{Kompetensi di Dunia Maya:Kehati-hatian:} Ucapan yang di dunia
  nyata biasa saja bisa terekam dan berbahaya di medsos.
\item
  \textbf{Respek:} Menghargai orang lain dan kebutuhan akan perhatian.
\item
  \textbf{Sopan:} Sadar banyak orang di sekitar.
\item
  \textbf{Keseimbangan:} Antara medsos dan tatap muka (termasuk
  interaktif seperti Zoom).
\item
  \textbf{Manfaatkan Keragaman Multimoda:} (gambar, video, dll).
\item
  \textbf{Tugas Pengembangan Diri:}Identifikasi kebutuhan yang dapat
  dipenuhi melalui komunikasi.
\item
  Pilih situasi, modelkan secara transaksional, terapkan konsep.
\item
  Evaluasi kompetensi (kompeten vs.~tidak kompeten) dan tentukan sasaran
  perbaikan.
\item
  Evaluasi level kompetensi medsos untuk membangun relasi dan cara
  meningkatkannya.
\item
  Latih repertoar (menyanyi, pidato, storytelling, puisi, dll.) secara
  konsisten.
\end{itemize}

\section{Kuis}\label{kuis}

(10 Soal Esai Singkat Untuk Di Submit ke
\url{https://forms.office.com/r/WCHBJJiwXc})

\begin{enumerate}
\def\labelenumi{\arabic{enumi}.}
\tightlist
\item
  Jelaskan empat aspek yang memicu dan dipicu perubahan dalam
  komunikasi. Berikan contoh singkat untuk setiap aspek.
\item
  Mengapa komunikasi disebut sebagai ``usaha dari komunikator untuk
  menghasilkan nilai yang menguntungkan bagi keempat pihak''? Sebutkan
  keempat pihak tersebut.
\item
  Apa yang dimaksud dengan ``repertoar'' dalam konteks komunikasi, dan
  mengapa penting untuk disiapkan?
\item
  Jelaskan perbedaan mendasar antara model komunikasi transaksional
  dengan model komunikasi satu arah (misalnya, A mengirim pesan ke B
  tanpa balasan).
\item
  Bedakan antara tujuan komunikasi yang bersifat ``personal'' dan
  ``impersonal'' berdasarkan fungsinya.
\item
  Sebutkan dan jelaskan secara singkat empat dari enam karakteristik
  kompetensi berkomunikasi yang disebutkan dalam materi.
\item
  Bagaimana media sosial dapat memfasilitasi dan memperkuat komunikasi
  interpersonal, terutama dalam mengatasi ``kelemahan individu''?
\item
  Jelaskan analogi komunikasi langsung (offline) sebagai ``tenis'' dan
  komunikasi media sosial (online) sebagai ``catur''. Apa implikasi dari
  perbedaan ini?
\item
  Meskipun memiliki banyak potensi, media sosial juga memiliki risiko
  dalam komunikasi interpersonal. Sebutkan tiga risiko utama tersebut.
\item
  Sebutkan tiga sikap atau tindakan yang perlu diperhatikan untuk
  menjaga kompetensi komunikasi di dunia maya.
\end{enumerate}

Jawaban \url{https://forms.office.com/r/WCHBJJiwXc} \emph{waktu max 5
hari}

\section{Pertanyaan Esai (Pekerjaan Rumah Tidak Perlu di
Submit)}\label{pertanyaan-esai-pekerjaan-rumah-tidak-perlu-di-submit}

\begin{enumerate}
\def\labelenumi{\arabic{enumi}.}
\tightlist
\item
  Analisis kasus tetangga yang mengalami kecelakaan dan membutuhkan
  pertolongan mobil. Bagaimana Anda akan merancang ``repertoar''
  komunikasi Anda, dengan mempertimbangkan tujuan komunikasi
  (pengertian, persetujuan, kemufakatan) dan empat pihak yang terlibat?
\item
  Diskusikan bagaimana konsep ``personal versus impersonal'' dalam
  tujuan komunikasi dapat memengaruhi pilihan bahasa dan strategi
  penyampaian pesan dalam konteks wawancara kerja vs.~upaya membangun
  relasi romantis.
\item
  Jelaskan secara mendalam bagaimana ``kompleksitas kognitif''
  (perceiving, thinking, knowing, remembering, judging, problem solving)
  berperan penting dalam meningkatkan kompetensi komunikasi
  interpersonal seseorang. Berikan contoh bagaimana masing-masing elemen
  berkontribusi.
\item
  Bandingkan potensi dan risiko media sosial dalam konteks membangun dan
  memelihara relasi. Menurut Anda, bagaimana cara mengoptimalkan potensi
  dan meminimalkan risiko agar media sosial menjadi alat yang efektif
  untuk komunikasi interpersonal yang berkualitas?
\item
  Pilih satu situasi di mana Anda pernah berkomunikasi secara tidak
  kompeten. Analisis situasi tersebut menggunakan konsep-konsep yang
  dipelajari (misalnya, kurang repertoar, tidak adaptif, kurang empati,
  dll.) dan rumuskan sasaran-sasaran spesifik untuk perbaikan diri di
  masa depan.
\end{enumerate}

\section{Glosarium Istilah Kunci}\label{glosarium-istilah-kunci}

\begin{itemize}
\tightlist
\item
  \textbf{Komunikasi Antar Pribadi (Interpersonal Communication):}
  Proses pertukaran pesan, makna, dan pemahaman antara dua orang atau
  lebih, yang bersifat tatap muka atau melalui media, dengan tujuan
  tertentu.
\item
  \textbf{Situasi:} Peristiwa eksternal atau kondisi lingkungan yang
  terjadi di luar individu dan memengaruhi komunikasi.
\item
  \textbf{Kondisi:} Keadaan internal atau psikologis individu yang
  dipengaruhi oleh situasi.
\item
  \textbf{Prospek:} Prediksi atau harapan tentang apa yang akan terjadi
  di masa depan, baik positif maupun negatif, yang memicu aksi
  komunikasi.
\item
  \textbf{Repertoar:} Persediaan bahan-bahan (ucapan, lakon, strategi)
  yang siap digunakan dalam berbagai situasi komunikasi.
\item
  \textbf{Komunikator:} Pihak yang mengirimkan pesan.
\item
  \textbf{Penerima:} Pihak yang menerima dan menafsirkan pesan.
\item
  \textbf{Komunikasi Transaksional:} Model komunikasi di mana pengirim
  dan penerima secara simultan mengirim dan menerima pesan, menciptakan
  interaksi yang dinamis dan adaptif.
\item
  \textbf{Tujuan Personal:} Tujuan komunikasi yang berfokus pada
  pembangunan dan pemeliharaan relasi dekat, seringkali dengan
  penggunaan bahasa unik.
\item
  \textbf{Tujuan Impersonal:} Tujuan komunikasi yang berfokus pada
  pencapaian tujuan praktis atau fungsional, menggunakan bahasa yang
  lebih umum dan formal.
\item
  \textbf{Koleksi Repertoar:} Kemampuan untuk memiliki beragam pesan dan
  keterampilan yang siap digunakan dalam berbagai konteks komunikasi.
\item
  \textbf{Adaptabilitas:} Kemampuan komunikator untuk menyesuaikan diri
  dan merespons reaksi atau umpan balik dari audiens.
\item
  \textbf{Empati:} Kemampuan untuk memahami dan merasakan apa yang
  dirasakan orang lain, melihat dari sudut pandang mereka.
\item
  \textbf{Kompleksitas Kognitif:} Kecerdasan berpikir yang melibatkan
  berbagai proses mental seperti mempersepsi, berpikir, mengetahui,
  mengingat, menilai, dan memecahkan masalah.
\item
  \textbf{Perceiving (Berpersepsi):} Proses menginterpretasi informasi
  sensorik dari lingkungan.
\item
  \textbf{Thinking (Berpikir):} Proses mengolah dan menganalisis
  informasi yang ada.
\item
  \textbf{Knowing (Mengetahui):} Pemahaman atau kesadaran akan suatu
  fakta atau informasi.
\item
  \textbf{Remembering (Mengingat):} Proses memanggil kembali informasi
  yang tersimpan dalam memori.
\item
  \textbf{Judging (Menilai/Membuat Keputusan):} Proses membentuk opini
  atau membuat keputusan berdasarkan informasi yang tersedia.
\item
  \textbf{Problem Solving (Memecahkan Masalah):} Proses mengidentifikasi
  masalah dan mencari solusi yang efektif.
\item
  \textbf{Refleksi Diri (Self-assessment):} Proses evaluasi diri untuk
  mengetahui tingkat kompetensi komunikasi dan area yang perlu
  ditingkatkan.
\item
  \textbf{Social Power (Kekuatan Sosial):} Kemampuan individu untuk
  memanfaatkan dukungan dan kontribusi dari jaringan sosial atau
  komunitas untuk memperkuat komunikasi.
\item
  \textbf{Kaya Nuansa:} Komunikasi yang mengandung banyak isyarat
  nonverbal (nada suara, ekspresi, gerak-gerik) yang memperkaya makna
  pesan.
\item
  \textbf{Ramping (Lean):} Komunikasi yang miskin nuansa, seperti teks
  atau email, di mana sebagian besar makna harus disimpulkan dari
  kata-kata saja.
\item
  \textbf{Efimeral:} Sesuatu yang berumur pendek atau cepat berlalu,
  seperti beberapa bentuk komunikasi di media sosial.
\item
  \textbf{Multimoda:} Penggunaan berbagai saluran atau format komunikasi
  (teks, gambar, video, suara) untuk menyampaikan pesan.
\end{itemize}

\bookmarksetup{startatroot}

\chapter{Kuliah 2: Memahami Komunikasi Diri dan Pementasan
Diri}\label{kuliah-2-memahami-komunikasi-diri-dan-pementasan-diri}

\href{https://youtube.com/playlist?list=PL_m-BplfO92GU2Qz6-_CtBAJEQ1Cv3wKr&si=sgbfefy8EFwEb2A8}{Video
Clips}

\href{https://forms.office.com/r/JA0MdRsjmg}{Quiz}

\section{\texorpdfstring{\textbf{Materi}}{Materi}}\label{materi}

Berikut adalah materi perkuliahan untuk minggu kedua, berfokus pada
\textbf{Konsep Komunikasi Diri (Self-Communication Concept)}:

Minggu 2: Konsep Komunikasi Diri (Self-Communication Concept)

\section{1. Pendahuluan Komunikasi Diri (Intrapersonal
Communication)}\label{pendahuluan-komunikasi-diri-intrapersonal-communication}

\begin{itemize}
\tightlist
\item
  \textbf{Definisi}: Komunikasi intrapribadi adalah proses komunikasi
  yang terjadi di dalam diri individu.
\item
  Merupakan \textbf{dasar} dari bagaimana kita memahami dunia dan
  berinteraksi dengan orang lain.
\item
  Dapat dianalogikan sebagai \textbf{percakan internal} atau seorang
  aktor yang berbicara kepada sutradara di balik layar saat latihan atau
  di belakang panggung.
\item
  Melibatkan berbagai \textbf{``roh''} atau \textbf{``spirit''} yang
  bercakap-cakap di dalam diri kita.

  \begin{itemize}
  \tightlist
  \item
    \textbf{Rene Descartes} merumuskan adanya sosok yang berpikir di
    dalam diri, yang ia sebut \textbf{ego}.
  \item
    \textbf{Sigmund Freud} mengembangkan konsep ini menjadi tiga jenis
    roh yang menggunakan \emph{username} yang sama: \textbf{ego} (diri
    yang menyadari, berlogika), \textbf{id} (mewakili emosi, keinginan
    anak kecil yang tidak sabaran), dan \textbf{superego} (mewakili
    hikmat orang dewasa, nasihat yang diterima).
  \item
    Film ``Inside Out'' menggambarkan \textbf{lima spirit emosional
    dasar} (Joy, Sadness, Anger, Fear, dan Disgust) yang bertugas
    menyelamatkan kehidupan individu dan sudah ada sejak lahir.
    Spirit-spirit ini memiliki personalisasi dan saling terjalin
    (\emph{perikoresis}) membentuk kepribadian.
  \item
    Berbagai roh ini, meskipun berbeda, bersatu dalam pengabdian untuk
    individu; jika jalinan ini pincang, perilaku negatif dapat muncul.
  \end{itemize}
\end{itemize}

\section{2. Konsep Diri (Self-Concept)}\label{konsep-diri-self-concept}

\begin{itemize}
\tightlist
\item
  \textbf{Definisi}: Keseluruhan persepsi kita tentang siapa diri kita.
\item
  \textbf{Pembentukan Konsep Diri}:

  \begin{itemize}
  \tightlist
  \item
    Berkembang dan dipengaruhi oleh orang lain melalui proses
    \textbf{``reflected appraisal''} (penilaian yang direfleksikan), di
    mana kita melihat diri kita berdasarkan bagaimana kita yakin orang
    lain memandang kita.
  \item
    Ini termasuk \textbf{roh primer} (pendapat orang-orang di sekitar
    kita seperti orang tua dan teman dekat) dan \textbf{roh sekunder}
    (evaluasi dari orang-orang berpengaruh seperti guru atau tokoh
    agama).
  \item
    Juga melalui \textbf{roh tersier} (perbandingan dengan orang lain)
    dan \textbf{roh kuarterner} (penghargaan atau rujukan seperti
    piala/piagam yang menandakan kesuksesan).
  \item
    Pada akhirnya, kita sendiri yang \textbf{paling bertanggung jawab}
    dalam membentuk konsep diri kita, meskipun pengaruh komunikasi
    eksternal pasti besar. Konsep diri akan semakin kuat apabila orang
    lain mengkonfirmasi hal tersebut.
  \item
    Konsep diri juga dibentuk oleh \textbf{budaya dan \emph{co-culture}}
    yang meracik roh di dalam diri kita.
  \item
    \textbf{Lima pertanyaan besar reflektif}, termasuk ``siapa saya'',
    adalah penting untuk mencapai hidup otentik.
  \end{itemize}
\item
  \textbf{Karakteristik Konsep Diri}:

  \begin{itemize}
  \tightlist
  \item
    Bersifat \textbf{dinamis dan dapat berubah}.
  \item
    Bersifat \textbf{subjektif}, sehingga bisa berlebihan atau kurang
    tepat. Ini dapat disebabkan oleh informasi yang kadaluarsa, umpan
    balik yang terdistorsi, mitos kesempurnaan, dan ekspektasi
    masyarakat.
  \item
    Konsep diri yang sehat seharusnya \textbf{fleksibel} dan tidak kaku.
    Namun, roh cenderung menolak perubahan, menyebabkan
    \textbf{konservatisme kognitif} di mana kita mencari informasi yang
    mendukung konsep diri kita.
  \end{itemize}
\item
  \textbf{Empat Jenis Roh dalam Konsep Diri} (dari terdalam hingga
  terluar):

  \begin{itemize}
  \tightlist
  \item
    \textbf{Primer}: Roh kepribadian bawaan kita (Joy, Sadness, dll.).
  \item
    \textbf{Sekunder}: Karakter hasil bentukan budaya atau
    \emph{co-culture}, yang dikenal juga sebagai superego.
  \item
    \textbf{Tersier}: Peran dalam pekerjaan atau profesi kita (contoh:
    dokter, polisi, dosen, pemain bola).
  \item
    \textbf{Kuarterner}: Peran atau karakter yang kita mainkan dalam
    permainan, drama, atau sandiwara (contoh: Ben Kingsley sebagai
    Mahatma Gandhi).
  \item
    Makin ke kanan (dari primer ke kuarterner), makin besar
    \emph{effort} yang diperlukan untuk menjadi orang lain; makin ke
    kiri, makin apa adanya. Idealnya, keempat roh ini dikembangkan
    secara seimbang, tetapi \textbf{prioritas utama adalah menjadi diri
    primer yang otentik}.
  \end{itemize}
\end{itemize}

\section{3. Harga Diri (Self-Esteem)}\label{harga-diri-self-esteem}

\begin{itemize}
\tightlist
\item
  \textbf{Definisi}: Evaluasi subjektif seseorang tentang nilai dan
  harga dirinya sebagai pribadi.
\item
  \textbf{Manfaat dan Kerugian Harga Diri Tinggi atau Rendah}:

  \begin{itemize}
  \tightlist
  \item
    \textbf{Siklus Positif}: Harga diri tinggi membuat seseorang percaya
    diri, bertindak, berusaha keras, dan akhirnya berhasil, yang
    kemudian semakin menaikkan harga diri.
  \item
    \textbf{Siklus Negatif}: Harga diri rendah membuat seseorang merasa
    tidak mampu, gampang menyerah, menyimpulkan bahwa ia memang tidak
    mampu, lalu harga diri semakin merosot.
  \end{itemize}
\item
  \textbf{Peran Komunikasi}: Komunikasi berperan besar dalam pembentukan
  harga diri seseorang.
\end{itemize}

\section{4. Presentasi Diri (Image Management /
Self-Presentation)}\label{presentasi-diri-image-management-self-presentation}

\begin{itemize}
\tightlist
\item
  \textbf{Definisi}: Bagaimana individu secara strategis
  mengkomunikasikan diri untuk mempengaruhi persepsi orang lain, dikenal
  sebagai \textbf{manajemen kesan}. Ini adalah upaya untuk ``mementaskan
  diri'' pada lingkaran publik.
\item
  \textbf{Tujuan Pementasan Diri}: Ingin menyampaikan sisi positif dan
  mungkin menyembunyikan sisi negatif dari diri kita.
\item
  \textbf{Tiga Versi Diri}: Versi yang sebenarnya, versi yang menurut
  kita kita itu siapa, dan versi yang ingin kita proyeksikan kepada
  orang lain.
\item
  \textbf{Empat Kemungkinan Jenis Komunikasi dalam Pementasan Diri}
  (berdasarkan persepsi diri dan proyeksi ke publik):

  \begin{itemize}
  \tightlist
  \item
    \textbf{Kuadran 1}: Kita tahu keistimewaan kita dan
    memproyeksikannya ke publik (contoh: mempromosikan kelebihan saat
    wawancara kerja).
  \item
    \textbf{Kuadran 2}: Melakukan \textbf{\emph{self-deprecation}}
    (merendah diri), yaitu tahu kelebihan tapi tidak mengakuinya di
    depan orang, seringkali agar bisa \emph{fit in} ke dalam suatu
    komunitas.
  \item
    \textbf{Kuadran 3}: Mengkomunikasikan kekurangan dengan jujur,
    sering untuk membangun \textbf{intimasi} atau mencari pertolongan
    (contoh: ke dokter).
  \item
    \textbf{Kuadran 4}: Melakukan \textbf{pencitraan}, yaitu menampilkan
    sesuatu yang istimewa padahal sebenarnya tidak benar. Tujuannya bisa
    positif (misalnya agar orang yang kita kasihi tidak khawatir),
    netral (misalnya dalam kampanye pemilihan umum), atau negatif
    (penipuan).
  \end{itemize}
\item
  \textbf{Pengelolaan Citra}:

  \begin{itemize}
  \tightlist
  \item
    Bersifat \textbf{kolaboratif dan adaptif}, bergantung pada reaksi
    orang lain (contoh: menyesuaikan tampilan sesuai lingkungan, seperti
    saat melamar kerja vs.~bergabung dengan komunitas \emph{bikers}).
  \item
    Tiga aspek yang perlu disiapkan saat mementaskan diri: \textbf{sopan
    santun} (kata-kata, gerak-gerik, bahasa tubuh),
    \textbf{pakaian/ornamen} yang mendukung, dan
    \textbf{setting/lingkungan} (background, benda dekoratif).
  \item
    Pencitraan juga bisa \textbf{jujur}, di mana kita menampilkan apa
    adanya, termasuk sesuatu yang bersifat pribadi yang bisa membuat
    kita jengah, seperti menutupi tubuh dengan pakaian.
  \end{itemize}
\item
  \textbf{Konsep ``Wajah'' (Face)} yang dapat ditunjukkan kepada orang
  lain:

  \begin{itemize}
  \tightlist
  \item
    \textbf{\emph{Fellowship face}}: Keinginan untuk disukai dan
    diterima.
  \item
    \textbf{\emph{Autonomy face}}: Keinginan untuk tidak diganggu.
  \item
    \textbf{\emph{Competence face}}: Keinginan untuk dihormati karena
    kecerdasan dan kemampuan.
  \end{itemize}
\item
  \textbf{Jendela Johari} sebagai model untuk membangun kedekatan:

  \begin{itemize}
  \tightlist
  \item
    \textbf{Area 1 (Open)}: Diri yang kita tahu dan orang lain tahu
    (daerah ``sama-sama tahu''). Kedekatan terjadi jika area ini semakin
    luas.
  \item
    \textbf{Area 2 (Blind)}: Diri kita yang orang lain tahu tapi kita
    sendiri tidak tahu.
  \item
    \textbf{Area 3 (Hidden)}: Rahasia kita yang kita tahu tapi orang
    lain tidak tahu.
  \item
    \textbf{Area 4 (Unknown)}: Fakta tentang diri kita yang kedua pihak
    sama-sama tidak tahu.
  \end{itemize}
\item
  \textbf{Manfaat Penyingkapan Diri (Self-Disclosure)}:

  \begin{itemize}
  \tightlist
  \item
    \textbf{Katarsis}: Merasa lega setelah mengungkapkan sesuatu yang
    dirahasiakan.
  \item
    \textbf{Swaklarifikasi}: Menjernihkan keyakinan, opini, pikiran,
    sikap, dan perasaan sendiri.
  \item
    \textbf{Swavalidasi}: Mendapatkan persetujuan dari pendengar tentang
    sosok kita.
  \item
    \textbf{Imbal Balas (Reciprocity)}: Berharap orang lain juga
    mengungkapkan rahasia mereka kepada kita.
  \item
    \textbf{Pembentukan Kesan}: Menarik simpati dari orang lain.
  \item
    \textbf{Menjaga dan Meningkatkan Relasi}: Memperluas Area 1 jendela
    Johari.
  \item
    \textbf{Kewajiban Moral}: Agar pihak berkepentingan tahu dan tidak
    merasa tertipu (contoh: memberitahukan penyakit tertentu).
  \end{itemize}
\item
  \textbf{Risiko Penyingkapan Diri}:

  \begin{itemize}
  \tightlist
  \item
    \textbf{Penolakan}.
  \item
    Menimbulkan \textbf{kesan negatif}.
  \item
    Menurunkan \textbf{tingkat kepuasan relasi}.
  \item
    \textbf{Kehilangan pengaruh}.
  \item
    \textbf{Kehilangan kendali}.
  \item
    \textbf{Menyakiti orang lain}.
  \end{itemize}
\item
  \textbf{Pertimbangan Sebelum Penyingkapan Diri}:

  \begin{itemize}
  \tightlist
  \item
    Apakah orang lain itu penting bagi kita.
  \item
    Apakah risiko penyingkapan layak diambil dan sebanding dengan
    manfaatnya.
  \item
    Apakah penyingkapan diri pantas dilakukan pada orang yang tepat.
  \item
    Apakah penyingkapan diri ini akan berbalas dengan penyingkapan diri
    orang lain.
  \item
    Apakah alasannya konstruktif (membangun).
  \end{itemize}
\item
  \textbf{Deception (Berbohong)}:

  \begin{itemize}
  \tightlist
  \item
    Tidak mengungkapkan yang sebenarnya.
  \item
    \textbf{Kejujuran adalah hal yang seringkali tepat meskipun
    menyakitkan}.
  \item
    Pertimbangan sebelum berbohong: kelayakan manfaat dari berbohong,
    apakah pihak yang dibohongi diuntungkan, apakah tidak ada jalan lain
    selain berbohong, dan bagaimana respons jika kebenaran terungkap.
  \item
    Pada akhirnya, kejujuran tentang konsep diri akan memerdekakan dan
    membawa sukacita sejati dalam jangka panjang. \textbf{Kompetensi
    untuk menjadi jujur} dalam segala situasi jauh lebih penting untuk
    dipelajari.
  \end{itemize}
\end{itemize}

\section{5. Peran Prediksi yang Memenuhi Diri Sendiri (Self-Fulfilling
Prophecy)}\label{peran-prediksi-yang-memenuhi-diri-sendiri-self-fulfilling-prophecy}

\begin{itemize}
\tightlist
\item
  \textbf{Definisi}: Harapan seseorang tentang suatu peristiwa, dan
  perilaku selanjutnya berdasarkan harapan tersebut, membuat hasil lebih
  mungkin terjadi.
\item
  \textbf{Mekanisme}: Kepercayaan pada suatu ramalan atau ekspektasi
  akan munculnya suatu peristiwa membuat kita bertindak berdasarkan
  ekspektasi tersebut, sehingga peluang peristiwa itu terjadi malah
  membesar.
\item
  Ekspektasi ini dapat dibentuk oleh diri sendiri atau oleh orang lain
  (asalkan dikomunikasikan kepada orang tersebut).
\item
  \textbf{Visualisasi Positif} adalah bentuk konstruktif dari prediksi
  yang memenuhi diri sendiri, yaitu membayangkan prospek positif yang
  menstimulasi diri untuk berhasil.
\end{itemize}

\section{6. Pentingnya ``Self-Talk'' dan Mengatasi ``Vultures''
(Gangguan
Psikologis)}\label{pentingnya-self-talk-dan-mengatasi-vultures-gangguan-psikologis}

\begin{itemize}
\tightlist
\item
  \textbf{\emph{Self-Talk}} (bicara dalam diri): Dapat bekerja untuk
  atau melawan seseorang.
\item
  \textbf{``\emph{Psychological Vultures}''}: Kritik diri negatif yang
  merusak harga diri.
\item
  Penting untuk \textbf{mengubah pikiran negatif menjadi pikiran yang
  memberdayakan} dan membangun kepercayaan diri.
\item
  Ini termasuk praktik \textbf{afirmasi} (kata-kata positif yang
  diucapkan di dalam hati) untuk mengatasi rasa takut.
\item
  Saran lain untuk mengatasi rasa takut berkomunikasi:

  \begin{itemize}
  \tightlist
  \item
    \textbf{Latihan/rehearsal} keterampilan percakapan secara berulang.
  \item
    \textbf{``Vaksinasi''}: Menghadapi situasi yang ditakuti dalam skala
    kecil secara bertahap untuk membangun keberanian.
  \item
    Memiliki \textbf{hasrat atau keinginan} untuk berkomunikasi, bukan
    karena harus tapi karena menyukainya.
  \item
    \textbf{Visualisasi Positif}: Membayangkan prospek positif yang
    terjadi.
  \item
    \textbf{Menerima rasa takut} itu ada, namun tetap bertindak.
  \end{itemize}
\end{itemize}

\section{7. Diskusi dan Aplikasi}\label{diskusi-dan-aplikasi}

\begin{itemize}
\tightlist
\item
  \textbf{Refleksi}: Mengajak mahasiswa untuk merefleksikan bagaimana
  konsep-konsep komunikasi diri (konsep diri, harga diri, manajemen
  kesan, prediksi yang memenuhi diri sendiri, dan \emph{self-talk})
  mempengaruhi komunikasi interpersonal mereka sehari-hari.
\item
  \textbf{Latihan Praktis}: ``Activity 2.7 Practicing Positive
  Visualization'' atau mengevaluasi ``\emph{vultures}'' mereka sendiri.
\item
  \textbf{Hidup Otentik}: Menekankan pentingnya hidup penuh integritas,
  yaitu hidup yang selaras atau koheren dengan identitas diri yang
  sebenarnya (\emph{true self}). Ini berarti apa yang kita ingini,
  yakini, pikirkan, ucapkan, miliki, dan buat itu sesuai dengan jati
  diri kita. Untuk itu, perlu mengembangkan empat lapis karakter yang
  selaras dengan nilai-nilai ideal (``love like God, think like a king,
  speak like a poet, serve like a slave'').
\item
  \textbf{Kejujuran}: Mengembangkan kompetensi untuk menjadi jujur dalam
  segala situasi, karena kejujuran akan memerdekakan dan membawa
  sukacita sejati dalam jangka panjang.
\end{itemize}

\section{\texorpdfstring{\textbf{Ringkasan}}{Ringkasan}}\label{ringkasan}

Panduan belajar ini dirancang untuk meninjau pemahaman Anda tentang
konsep-konsep kunci yang berkaitan dengan komunikasi diri, pementasan
diri, dan hidup otentik, seperti yang disajikan dalam materi sumber.

\section{I. Konsep Diri
(Self-Concept)}\label{i.-konsep-diri-self-concept}

\subsection{A. Definisi dan
Pentingnya}\label{a.-definisi-dan-pentingnya}

\begin{itemize}
\tightlist
\item
  \textbf{Definisi:} Konsep diri adalah persepsi keseluruhan kita
  tentang siapa diri kita. Ini adalah jawaban atas pertanyaan ``Siapa
  saya?''
\item
  \textbf{Pentingnya:} Hidup otentik, tanpa penyesalan, bergantung pada
  pemahaman dan penghidupan konsep diri sejati kita. Kegagalan untuk
  memahami ini dapat menyebabkan penyesalan mendalam di akhir hidup.
\item
  \textbf{Sifat Konsep Diri:} Subjektif, fleksibel (idealnya), dan
  terkadang resisten terhadap perubahan karena ``konservatisme
  kognitif'' atau pengaruh ``roh.''
\end{itemize}

\subsection{B. Pembentukan Konsep
Diri}\label{b.-pembentukan-konsep-diri}

\begin{itemize}
\tightlist
\item
  \textbf{Pengaruh Eksternal:Orang Sekitar (Primer):} Pendapat orang
  tua, teman dekat, dan mereka yang hidup bersama kita.
\item
  \textbf{Evaluasi Orang Berpengaruh (Sekunder):} Guru, tokoh agama,
  pengajar.
\item
  \textbf{Perbandingan Sosial (Tersier):} Membandingkan diri dengan
  orang lain dalam hal prestasi atau kinerja.
\item
  \textbf{Pengakuan Formal (Kuarter):} Piala, piagam, rujukan, atau
  bentuk pengakuan lainnya.
\item
  \textbf{Peran Komunikasi:} Komunikasi memainkan peran besar dalam
  membentuk konsep diri, seperti ``roh'' yang masuk ke dalam jiwa kita
  melalui interaksi.
\item
  \textbf{Tanggung Jawab Individu:} Meskipun ada banyak pengaruh
  eksternal, kita pada akhirnya yang paling bertanggung jawab dalam
  membentuk konsep diri kita sendiri.
\end{itemize}

\subsection{C. Roh Karakter}\label{c.-roh-karakter}

\begin{itemize}
\tightlist
\item
  \textbf{Definisi Roh Karakter:} Sebuah klasifikasi dari ``diri'' yang
  mengendalikan penggunaan ``username'' atau nama kita.
\end{itemize}

\begin{enumerate}
\def\labelenumi{\arabic{enumi}.}
\tightlist
\item
  \textbf{Empat Jenis Roh:Roh Primer (Kepribadian Bawaan):} Diri kita
  yang asli, kepribadian bawaan (misalnya, Joy, Sadness, Anger, Fear,
  Disgust dari film Inside Out). Ini adalah inti dari diri kita.
\item
  \textbf{Roh Sekunder (Karakter Budaya):} Karakter yang dibentuk oleh
  budaya atau \emph{co-culture} (superego Freud). Ini adalah bagaimana
  kita diukir oleh lingkungan sosial kita.
\item
  \textbf{Roh Tersier (Peran Profesi):} Peran yang kita mainkan dalam
  pekerjaan atau profesi (dokter, polisi, dosen, aktor).
\item
  \textbf{Roh Kuarter (Peran Drama/Permainan):} Peran atau karakter yang
  kita mainkan dalam drama, sandiwara, atau pementasan tertentu
  (misalnya, Ben Kingsley sebagai Mahatma Gandhi).
\end{enumerate}

\begin{itemize}
\tightlist
\item
  \textbf{Hubungan:} Semakin ke kanan (dari primer ke kuarter), semakin
  besar upaya yang dibutuhkan dan semakin kita menjadi ``orang lain.''
  Semakin ke kiri, semakin kita menjadi diri kita yang asli. Idealnya,
  keempat roh ini dikembangkan secara seimbang, tetapi roh primer adalah
  prioritas utama untuk kehidupan otentik.
\end{itemize}

\section{II. Pementasan Diri (Self-Presentation / Image
Management)}\label{ii.-pementasan-diri-self-presentation-image-management}

\subsection{A. Definisi dan Tujuan}\label{a.-definisi-dan-tujuan}

\begin{itemize}
\tightlist
\item
  \textbf{Definisi:} Upaya strategis untuk mengkomunikasikan siapa diri
  kita kepada publik, menampilkan sisi positif dan menyembunyikan sisi
  negatif.
\item
  \textbf{Tujuan:} Mempengaruhi persepsi orang lain, membangun
  kepercayaan, menjaga dan meningkatkan relasi, menarik simpati,
  menghindari kekhawatiran orang lain, atau bahkan untuk tujuan netral
  (pengadilan, pemilu) hingga negatif (penipuan).
\end{itemize}

\subsection{B. Tiga Versi Diri}\label{b.-tiga-versi-diri}

\begin{enumerate}
\def\labelenumi{\arabic{enumi}.}
\tightlist
\item
  \textbf{Versi Sebenarnya:} Diri kita yang sesungguhnya (seringkali
  tidak sepenuhnya kita tahu).
\item
  \textbf{Versi Menurut Kita:} Bagaimana kita melihat diri kita.
\item
  \textbf{Versi yang Diproyeksikan:} Bagaimana kita ingin orang lain
  melihat kita.
\end{enumerate}

\begin{itemize}
\tightlist
\item
  Fokus pementasan diri umumnya pada versi kedua dan ketiga.
\end{itemize}

\subsection{C. Diagram Kartesius Pementasan Diri (Perspektif
Diri)}\label{c.-diagram-kartesius-pementasan-diri-perspektif-diri}

\begin{itemize}
\tightlist
\item
  \textbf{Sumbu X (Pesan ke Publik):} Positif atau Negatif (Projecting
  Self)
\item
  \textbf{Sumbu Y (Persepsi Diri):} Positif atau Negatif (Perception of
  Self)
\end{itemize}

\begin{enumerate}
\def\labelenumi{\arabic{enumi}.}
\tightlist
\item
  \textbf{Kuadran 1 (Self-Enhancement):} Tahu kelebihan dan
  memproyeksikannya (misalnya, saat wawancara kerja).
\item
  \textbf{Kuadran 2 (Self-Deprecation):} Tahu kelebihan tapi merendah
  (misalnya, agar bisa diterima komunitas).
\item
  \textbf{Kuadran 3 (Jujur Mengkomunikasikan Kekurangan):} Tahu
  kelemahan dan mengkomunikasikannya (misalnya, untuk intimasi, ke
  dokter).
\item
  \textbf{Kuadran 4 (Pencitraan / Deception):} Menampilkan sesuatu yang
  istimewa padahal tahu itu tidak benar (misalnya, agar orang lain tidak
  khawatir, kampanye politik, penipuan).
\end{enumerate}

\subsection{D. Diagram Kartesius Pementasan Diri (Perspektif Orang
Lain)}\label{d.-diagram-kartesius-pementasan-diri-perspektif-orang-lain}

\begin{itemize}
\tightlist
\item
  \textbf{Sumbu X (Upaya Memproyeksikan Diri):} Sama dengan sebelumnya.
\item
  \textbf{Sumbu Y (Hasil di Mata Orang Lain):} Bagaimana orang lain
  melihat kita.
\end{itemize}

\begin{enumerate}
\def\labelenumi{\arabic{enumi}.}
\tightlist
\item
  \textbf{Kuadran 1 (Berhasil):} Komunikasi berhasil, kebaikan kita
  diketahui dan dilihat orang lain positif.
\item
  \textbf{Kuadran 2 (Istimewa):} Mengkomunikasikan kekurangan, tapi
  ditangkap sebagai kelebihan.
\item
  \textbf{Kuadran 3 (Ekstrem/Gagal):} Mengkomunikasikan kekurangan, dan
  orang lain melihatnya memang buruk (misalnya, penjahat).
\item
  \textbf{Kuadran 4 (Gagal / Pendusta):} Mencoba menyampaikan kesan
  positif, tapi ditangkap negatif (dianggap pendusta).
\end{enumerate}

\subsection{E. Aspek yang Perlu Disiapkan dalam Pementasan
Diri}\label{e.-aspek-yang-perlu-disiapkan-dalam-pementasan-diri}

\begin{enumerate}
\def\labelenumi{\arabic{enumi}.}
\tightlist
\item
  \textbf{Sopan Santun:} Kata-kata, gerak-gerik, bahasa tubuh.
\item
  \textbf{Pakaian/Ornamen:} Mendukung citra yang ingin ditampilkan.
\item
  \textbf{Settingan:} Lingkungan, \emph{background}, benda dekoratif.
\end{enumerate}

\subsection{F. Integritas dan Deception dalam Pementasan
Diri}\label{f.-integritas-dan-deception-dalam-pementasan-diri}

\begin{itemize}
\tightlist
\item
  Pementasan diri bersifat kolaboratif dan adaptif.
\item
  Pencitraan tidak selalu berarti berbohong; ada banyak tampilan jujur
  yang berbeda untuk situasi yang berbeda.
\item
  \emph{Deception} (penipuan) di dunia maya sangat umum. Penting untuk
  membedakan antara akting (seperti Ben Kingsley sebagai Gandhi) dan
  kebohongan yang disengaja.
\item
  Penting untuk memilih ``wajah'' yang tepat untuk situasi yang tepat
  tanpa merasa malu atau jengah.
\end{itemize}

\section{III. Penyingkapan Diri
(Self-Disclosure)}\label{iii.-penyingkapan-diri-self-disclosure}

\subsection{A. Definisi dan Aspek}\label{a.-definisi-dan-aspek}

\begin{itemize}
\tightlist
\item
  \textbf{Definisi:} Pengkomunikasian informasi tentang diri kita dengan
  sengaja kepada orang lain.
\end{itemize}

\begin{enumerate}
\def\labelenumi{\arabic{enumi}.}
\tightlist
\item
  \textbf{Empat Aspek:Kejujuran:} Jujur atau tidak.
\item
  \textbf{Kedalaman:} Seberapa dalam atau hanya permukaan.
\item
  \textbf{Kebaruan:} Informasi baru atau sudah umum diketahui.
\item
  \textbf{Membawa Kedekatan:} Apakah penyingkapan ini meningkatkan
  kedekatan.
\end{enumerate}

\subsection{B. Dimensi Keluasan dan Kedalaman
Informasi}\label{b.-dimensi-keluasan-dan-kedalaman-informasi}

\begin{itemize}
\tightlist
\item
  \textbf{Keluasan:} Berbagai topik yang bisa dibicarakan (perasaan
  hubungan, latar belakang keluarga, masalah akademik, opini, tampilan
  fisik, ambisi, karir).
\item
  \textbf{Kedalaman:} Seberapa rinci dan pribadi informasi tersebut.
\end{itemize}

\begin{enumerate}
\def\labelenumi{\arabic{enumi}.}
\tightlist
\item
  \textbf{Perbendaharaan Informasi (Lapis):Klise (Paling Kulit):} Ucapan
  umum, tidak bermakna mendalam.
\item
  \textbf{Fakta:} Informasi objektif.
\item
  \textbf{Opini:} Pendapat pribadi.
\item
  \textbf{Perasaan (Paling Dalam):} Emosi dan pengalaman subjektif.
\end{enumerate}

\subsection{C. Jendela Johari}\label{c.-jendela-johari}

\begin{itemize}
\tightlist
\item
  \textbf{Konsep:} Model untuk memahami dan meningkatkan kesadaran diri
  dan hubungan interpersonal.
\end{itemize}

\begin{enumerate}
\def\labelenumi{\arabic{enumi}.}
\tightlist
\item
  \textbf{Empat Area:Open Area:} Kita tahu, orang lain tahu (bertujuan
  untuk diperluas untuk kedekatan).
\item
  \textbf{Blind Area:} Orang lain tahu, kita tidak tahu.
\item
  \textbf{Hidden Area:} Kita tahu, orang lain tidak tahu (rahasia kita).
\item
  \textbf{Unknown Area:} Kedua pihak sama-sama tidak tahu.
\end{enumerate}

\begin{itemize}
\tightlist
\item
  \textbf{Tujuan:} Memperluas area \emph{Open} untuk membangun
  kedekatan.
\end{itemize}

\subsection{D. Manfaat dan Keuntungan Penyingkapan
Diri}\label{d.-manfaat-dan-keuntungan-penyingkapan-diri}

\begin{enumerate}
\def\labelenumi{\arabic{enumi}.}
\tightlist
\item
  \textbf{Katarsis:} Perasaan lega setelah mengungkapkan rahasia.
\item
  \textbf{Swaklarifikasi:} Menjernihkan pikiran, opini, sikap, dan
  perasaan kita sendiri.
\item
  \textbf{Swavalidasi:} Mendapatkan persetujuan dari pendengar tentang
  pikiran atau diri kita.
\item
  \textbf{Imbal Balas:} Berharap orang lain juga mengungkapkan rahasia
  mereka.
\item
  \textbf{Pembentukan Kesan:} Menarik simpati.
\item
  \textbf{Menjaga dan Meningkatkan Relasi:} Memperluas area \emph{Open}.
\item
  \textbf{Kewajiban Moral:} Menyampaikan informasi penting agar pihak
  berkepentingan tahu (misalnya, penyakit).
\end{enumerate}

\subsection{E. Risiko dan Kerugian Penyingkapan
Diri}\label{e.-risiko-dan-kerugian-penyingkapan-diri}

\begin{enumerate}
\def\labelenumi{\arabic{enumi}.}
\tightlist
\item
  \textbf{Penolakan:} Orang lain menolak atau menjauh.
\item
  \textbf{Kesan Negatif:} Menimbulkan pandangan buruk.
\item
  \textbf{Menurunkan Tingkat Kepuasan Relasi:} Konflik atau
  ketidaknyamanan dalam hubungan.
\item
  \textbf{Kehilangan Pengaruh:} Posisi atau kekuatan kita berkurang.
\item
  \textbf{Kehilangan Kendali:} Informasi yang diungkapkan digunakan atau
  disebarkan di luar keinginan kita.
\item
  \textbf{Menyakiti Orang Lain:} Ungkapan yang jujur dapat melukai
  perasaan.
\end{enumerate}

\subsection{F. Pertimbangan Sebelum Penyingkapan
Diri}\label{f.-pertimbangan-sebelum-penyingkapan-diri}

\begin{itemize}
\tightlist
\item
  \textbf{Pentingnya Orang Lain:} Apakah orang tersebut penting?
\item
  \textbf{Kelayakan Risiko:} Apakah manfaat sebanding dengan risikonya?
\item
  \textbf{Pantas:} Apakah sesuai dilakukan pada orang yang tepat?
\item
  \textbf{Timbal Balik:} Apakah akan ada balasan penyingkapan dari orang
  lain?
\item
  \textbf{Konstruktif:} Apakah alasannya membangun?
\end{itemize}

\subsection{G. Deception (Kebohongan)}\label{g.-deception-kebohongan}

\begin{itemize}
\tightlist
\item
  \textbf{Definisi:} Tidak mengungkapkan yang sebenarnya, berbohong.
\item
  \textbf{Alasan Kebohongan:} Bisa karena terpaksa dalam situasi sulit,
  atau untuk tujuan positif (misalnya, melindungi perasaan orang).
\item
  \textbf{Pentingnya Kejujuran:} Seringkali merupakan tindakan yang
  benar, meskipun menyakitkan.
\item
  \textbf{Pertimbangan Sebelum Berbohong:}Apakah manfaat kebohongan itu
  layak?
\item
  Apakah pihak yang dibohongi diuntungkan?
\item
  Apakah tidak ada jalan lain?
\item
  Bagaimana respons orang jika tahu kebenarannya?
\item
  \textbf{Kejujuran Jangka Panjang:} Akan memerdekakan dan membawa
  sukacita sejati.
\item
  \textbf{Mengapa Mempelajari Kebohongan:} Untuk memahami perilaku orang
  lain dan menghadapi orang jahat.
\end{itemize}

\section{IV. Hidup Otentik}\label{iv.-hidup-otentik}

\subsection{A. Definisi Integritas dan
Otentisitas}\label{a.-definisi-integritas-dan-otentisitas}

\begin{itemize}
\tightlist
\item
  \textbf{Hidup Penuh Integritas:} Hidup yang selaras (koheren) dengan
  identitas diri kita yang sebenarnya atau \emph{true self}.
\item
  \textbf{Hidup Otentik:} Apa yang kita inginkan, yakini, pikirkan,
  ucapkan, miliki (termasuk relasi), dan buat sesuai dengan jati diri
  kita.
\end{itemize}

\subsection{B. Empat Lapis Karakter untuk Hidup Otentik
(Panutan)}\label{b.-empat-lapis-karakter-untuk-hidup-otentik-panutan}

\begin{itemize}
\tightlist
\item
  \textbf{Primer: Love like God (Pribadi Penuh Kasih Tanpa
  Batas)}Mengasihi dan menyayangi seperti Tuhan.
\item
  Panutan: Tuhan Yang Maha Kuasa.
\item
  \textbf{Sekunder: Think like a King (Berpikir Seperti
  Raja/Ratu)}Bertanggung jawab, cakap, tegas, berani, amanah, adil untuk
  kepentingan banyak orang.
\item
  Panutan: Raja/Ratu (misalnya, Alexander The Great).
\item
  \textbf{Tersier: Speak like a Poet (Berbicara Seperti
  Pujangga)}Menguasai kata-kata yang menghidupkan, mencerdaskan, dan
  mengekspresikan keindahan.
\item
  Panutan: Pujangga.
\item
  \textbf{Kuarter: Serve like a Slave (Melayani Seperti Hamba)}Melayani
  dengan sepenuh hati, perhatian, keahlian, totalitas, dan pengorbanan
  tanpa perhitungan.
\item
  Panutan: Hamba yang melayani tuannya.
\end{itemize}

\subsection{C. Pembelajaran Karakter}\label{c.-pembelajaran-karakter}

\begin{itemize}
\tightlist
\item
  Dapat dipelajari melalui literatur dan studi.
\item
  Pembelajaran terbaik seringkali melalui penderitaan dan kekurangan
  diri.
\item
  Contoh Amanda Gorman: Mengubah kesulitan berbicara menjadi kekuatan
  melalui membaca dan menulis, serta melatih diri secara gigih.
\end{itemize}

\section{V. Komunikasi Intrapribadi (Intrapersonal
Communication)}\label{v.-komunikasi-intrapribadi-intrapersonal-communication}

\begin{itemize}
\tightlist
\item
  \textbf{Definisi:} Proses komunikasi yang terjadi di dalam diri
  individu.
\item
  \textbf{Dasar:} Fondasi bagaimana kita memahami dunia dan berinteraksi
  dengan orang lain.
\item
  \textbf{Self-Talk:} Pembicaraan dalam diri yang bisa bersifat positif
  atau negatif.
\item
  \textbf{Psychological Vultures:} Kritik diri negatif yang merusak
  harga diri. Penting untuk mengubahnya menjadi pikiran yang
  memberdayakan.
\item
  \textbf{Self-Fulfilling Prophecy:} Harapan atau keyakinan tentang
  suatu peristiwa dapat memengaruhi perilaku kita, sehingga membuat
  peristiwa itu lebih mungkin terjadi. Visualisasi positif adalah bentuk
  konstruktif dari ini.
\end{itemize}

\section{Kuis: Komunikasi Diri dan Pementasan
Diri}\label{kuis-komunikasi-diri-dan-pementasan-diri}

Jawablah pertanyaan-pertanyaan berikut dalam 2-3 kalimat.

\begin{enumerate}
\def\labelenumi{\arabic{enumi}.}
\tightlist
\item
  Jelaskan mengapa pemahaman tentang ``konsep diri'' dianggap krusial
  untuk menjalani hidup yang otentik, menurut materi sumber.
\item
  Sebutkan dan jelaskan secara singkat dua dari empat aspek yang
  membedakan jenis penyingkapan diri.
\item
  Dalam konteks Jendela Johari, apa yang dimaksud dengan ``Open Area''
  dan mengapa perlu diperluas untuk membangun kedekatan?
\item
  Identifikasi dua manfaat utama dari penyingkapan diri dan berikan
  contoh singkat untuk salah satunya.
\item
  Apa perbedaan antara ``Self-Enhancement'' dan ``Pencitraan'' dalam
  diagram kartesius pementasan diri, dari perspektif diri?
\item
  Menurut materi, sebutkan dua dari tiga aspek yang perlu disiapkan
  individu saat mementaskan diri di hadapan publik atau melalui media
  sosial.
\item
  Jelaskan konsep ``self-fulfilling prophecy'' dalam konteks pembentukan
  konsep diri.
\item
  Bagaimana konsep ``roh primer'' berbeda dengan ``roh kuarter'' dalam
  klasifikasi roh karakter?
\item
  Mengapa kejujuran dianggap penting dalam jangka panjang, meskipun
  kadang menyakitkan atau sulit?
\item
  Sebutkan salah satu dari empat panutan karakter yang disarankan untuk
  hidup otentik dan jelaskan maknanya secara singkat.
\end{enumerate}

\section{Pertanyaan Esai (Tidak Ada Jawaban
Disertakan)}\label{pertanyaan-esai-tidak-ada-jawaban-disertakan}

\begin{enumerate}
\def\labelenumi{\arabic{enumi}.}
\tightlist
\item
  Analisis hubungan antara ``self-fulfilling prophecy'' dan pembentukan
  konsep diri. Bagaimana siklus positif dan negatif dari
  \emph{self-esteem} dapat diperkuat atau dirusak oleh fenomena ini, dan
  apa peran komunikasi dalam proses tersebut?
\item
  Bandingkan dan kontraskan manfaat serta risiko dari penyingkapan diri.
  Dalam situasi apa penyingkapan diri dianggap layak diambil risikonya,
  dan bagaimana individu dapat menimbang keputusan untuk melakukan
  \emph{self-disclosure}?
\item
  Jelaskan konsep ``roh karakter'' (primer, sekunder, tersier, kuarter)
  dan bagaimana masing-masing lapis berkontribusi terhadap identitas
  diri seseorang. Mengapa materi sumber menekankan pentingnya
  mengembangkan roh primer secara otentik di atas yang lain?
\item
  Diskusikan implikasi etis dari ``pencitraan'' atau \emph{deception}
  dalam pementasan diri. Kapan \emph{deception} dapat dibenarkan (jika
  ada), dan bagaimana membedakannya dari ``akting'' yang dianggap netral
  atau positif?
\item
  Materi sumber menyajikan empat panutan karakter untuk hidup otentik:
  ``Love like God,'' ``Think like a King,'' ``Speak like a Poet,'' dan
  ``Serve like a Slave.'' Pilih dua dari panutan ini dan jelaskan secara
  mendalam bagaimana menginternalisasi dan menghidupinya dapat membantu
  seseorang mencapai integritas dan otentisitas dalam kehidupan
  sehari-hari.
\end{enumerate}

\section{Glosarium Istilah Kunci}\label{glosarium-istilah-kunci-1}

\begin{itemize}
\tightlist
\item
  \textbf{Konsep Diri (Self-Concept):} Keseluruhan persepsi, keyakinan,
  dan penilaian individu tentang siapa dirinya.
\item
  \textbf{Harga Diri (Self-Esteem):} Evaluasi subjektif individu tentang
  nilai dan harga dirinya sebagai pribadi.
\item
  \textbf{Siklus Positif/Negatif \emph{Self-Esteem}:} Pola perilaku dan
  pemikiran yang menaikkan (\emph{positif}) atau menurunkan
  (\emph{negatif}) harga diri seseorang.
\item
  \textbf{Reflected Appraisal:} Proses melihat diri kita berdasarkan
  bagaimana kita yakin orang lain memandang kita.
\item
  \textbf{Komunikasi Intrapribadi (Intrapersonal Communication):} Proses
  komunikasi yang terjadi di dalam diri individu, seperti berpikir,
  merenung, dan \emph{self-talk}.
\item
  \textbf{Roh Karakter:} Sebuah konsep untuk mengklasifikasikan berbagai
  ``diri'' atau persona yang kita miliki, mulai dari yang bawaan hingga
  yang diperankan.
\item
  \textbf{Roh Primer:} Kepribadian bawaan atau inti diri (misalnya,
  emosi dasar seperti Joy, Sadness).
\item
  \textbf{Roh Sekunder:} Karakter yang dibentuk oleh budaya atau
  \emph{co-culture} (misalnya, superego Freud).
\item
  \textbf{Roh Tersier:} Peran yang dimainkan dalam pekerjaan atau
  profesi.
\item
  \textbf{Roh Kuarter:} Peran atau karakter yang dimainkan dalam drama,
  sandiwara, atau pementasan tertentu.
\item
  \textbf{Pementasan Diri (Self-Presentation / Image Management):} Upaya
  strategis untuk mengkomunikasikan diri kepada orang lain guna
  memengaruhi persepsi mereka.
\item
  \textbf{Self-Enhancement:} Mengkomunikasikan kelebihan diri yang
  memang kita miliki.
\item
  \textbf{Self-Deprecation:} Merendah diri meskipun memiliki kelebihan.
\item
  \textbf{Pencitraan:} Menampilkan kesan positif atau istimewa yang
  mungkin tidak sepenuhnya benar, untuk tujuan tertentu.
\item
  \textbf{Deception:} Tindakan berbohong atau tidak mengungkapkan
  kebenaran.
\item
  \textbf{Penyingkapan Diri (Self-Disclosure):} Pengkomunikasian
  informasi tentang diri sendiri secara sengaja kepada orang lain.
\item
  \textbf{Jendela Johari:} Model untuk memahami dan meningkatkan
  kesadaran diri dan hubungan interpersonal, membagi diri menjadi empat
  area (Open, Blind, Hidden, Unknown).
\item
  \textbf{Katarsis:} Perasaan lega yang dialami setelah mengungkapkan
  emosi atau rahasia yang terpendam.
\item
  \textbf{Swaklarifikasi (Self-Clarification):} Proses menjernihkan
  pikiran, opini, atau perasaan kita sendiri melalui penyingkapan diri.
\item
  \textbf{Swavalidasi (Self-Validation):} Mendapatkan persetujuan atau
  konfirmasi dari orang lain tentang pikiran atau identitas diri kita.
\item
  \textbf{Konservatisme Kognitif (Cognitive Conservatism):}
  Kecenderungan untuk mencari informasi yang mendukung konsep diri yang
  sudah ada dan menolak informasi yang bertentangan.
\item
  \textbf{Self-Fulfilling Prophecy:} Fenomena di mana harapan seseorang
  tentang suatu peristiwa, dan perilaku selanjutnya berdasarkan harapan
  tersebut, membuat hasil lebih mungkin terjadi.
\item
  \textbf{Self-Talk:} Dialog internal atau pembicaraan yang kita lakukan
  dengan diri sendiri.
\item
  \textbf{Psychological Vultures:} Kritik diri negatif atau pikiran
  merusak yang mengikis harga diri.
\item
  \textbf{Hidup Otentik:} Kehidupan yang selaras dan koheren dengan jati
  diri sejati seseorang, di mana tindakan, pikiran, dan keyakinan
  konsisten dengan \emph{true self}.
\item
  \textbf{Integritas:} Keadaan menjadi utuh dan tidak terbagi, konsisten
  antara apa yang diyakini, dikatakan, dan dilakukan.
\item
  \textbf{Co-culture:} Kelompok dalam suatu budaya yang berbagi
  keyakinan, nilai, atau perilaku tertentu (misalnya, berdasarkan usia,
  ras, agama, profesi, atau aktivitas).
\item
  \textbf{Perikoresis:} Konsep relasi yang bersatu namun masing-masing
  memiliki kepribadian sendiri, seperti tarian melingkar dari roh-roh
  yang berbeda namun terjalin.
\end{itemize}

\bookmarksetup{startatroot}

\chapter{Kuliah 3 Memahami Persepsi dan
Realitas}\label{kuliah-3-memahami-persepsi-dan-realitas}

Video:
\href{https://youtube.com/playlist?list=PL_m-BplfO92HziIHJuNhhlF-2hOaFumHf&si=n0Zku45PZsSaYGYU}{link}

Kuiz: \href{https://forms.office.com/r/HPKrDLU2DN}{Submision}

Panduan ini dirancang untuk membantu Anda meninjau pemahaman Anda
tentang konsep persepsi, pembentukannya, pengaruhnya, dan hubungannya
dengan realitas, serta berbagai pandangan dunia yang dibahas.

\section{Bagian 1: Konsep Dasar
Persepsi}\label{bagian-1-konsep-dasar-persepsi}

\subsection{A. Definisi dan Fungsi
Persepsi}\label{a.-definisi-dan-fungsi-persepsi}

\begin{itemize}
\tightlist
\item
  \textbf{Persepsi:} Proses aktif menata, mengorganisir,
  mengidentifikasi, dan menafsirkan informasi indrawi dari lingkungan
  untuk memahami apa yang terjadi di sekitar kita.
\item
  \textbf{Persepsi Interpersonal:} Penerapan proses persepsi pada orang
  dan hubungan.
\item
  \textbf{Fungsi Persepsi:Menyelamatkan:} Memungkinkan kita
  mengidentifikasi bahaya dan peluang (misalnya, mangsa atau kompetitor
  bagi singa).
\item
  \textbf{Membangun Ekspektasi:} Membantu kita memprediksi apa yang akan
  terjadi dan mempersiapkan diri.
\item
  \textbf{Filter:} Menyaring rangsangan yang kita perhatikan dan
  memisahkan yang masuk akal dari yang tidak.
\item
  \textbf{Membentuk Konsep Diri:} Memengaruhi keyakinan kita tentang
  realitas, nilai, hal yang dihindari, dan emosi.
\end{itemize}

\subsection{B. Tahapan Pembentukan
Persepsi}\label{b.-tahapan-pembentukan-persepsi}

Persepsi umumnya melibatkan tiga atau empat tahapan dasar:

\begin{itemize}
\tightlist
\item
  \textbf{Seleksi (Selection):}Indra terstimulasi.
\item
  Cenderung memperhatikan rangsangan yang \textbf{intens, berulang, atau
  menunjukkan kontras/perubahan}.
\item
  \textbf{Organisasi (Organization):}Mengatur informasi yang dipilih
  dengan cara yang bermakna.
\item
  Menggunakan \textbf{skema persepsi} (kerangka kognitif) untuk
  mengkategorikan informasi:
\item
  \textbf{Konstruk fisik:} penampilan (usia, etnis, pakaian).
\item
  \textbf{Konstruk peran:} fungsi sosial/profesional (guru, mahasiswa).
\item
  \textbf{Konstruk interaksi:} perilaku/cara komunikasi (ramah,
  menuntut).
\item
  \textbf{Konstruk psikologis:} disposisi/suasana hati (bahagia, percaya
  diri).
\item
  \textbf{Pungtuasi:} Menentukan sebab dan akibat dalam interaksi,
  memengaruhi cara melihat perselisihan.
\item
  \textbf{Interpretasi (Interpretation):}Memberikan makna pada
  rangsangan yang telah diorganisir.
\item
  Dipengaruhi oleh \textbf{ekspektasi, pengalaman pribadi, kepribadian,
  dan asumsi tentang perilaku manusia}.
\item
  \textbf{Negosiasi (Negotiation):}Proses di mana komunikator saling
  memengaruhi persepsi satu sama lain.
\item
  Melalui pertukaran cerita atau ``narasi'' tentang dunia pribadi.
\end{itemize}

\section{Bagian 2: Faktor-faktor yang Memengaruhi
Persepsi}\label{bagian-2-faktor-faktor-yang-memengaruhi-persepsi}

\subsection{A. Penyebab Perbedaan
Persepsi}\label{a.-penyebab-perbedaan-persepsi}

\begin{itemize}
\tightlist
\item
  \textbf{Fisik:} Kondisi indra, usia, kesehatan, kelelahan, siklus
  biologis, rasa lapar, tantangan neurobehavioral.
\item
  \textbf{Psikologis:} Suasana hati, konsep diri, bias emosional,
  kecemasan.
\item
  \textbf{Sosial:} Peran gender, peran pekerjaan, peran relasional.
\item
  \textbf{Budaya dan Kokultur:} Nilai, norma, bahasa, sensitivitas
  budaya, cara menyampaikan umpan balik negatif (langsung vs.~tidak
  langsung). Contoh: perbedaan interpretasi ``I hear what you say'' atau
  ``quite good'' antara British dan Dutch.
\item
  \textbf{Akses Informasi:} Perbedaan informasi yang diterima dapat
  membentuk pandangan yang berbeda (misalnya, dalam politik).
\end{itemize}

\subsection{B. Kekuatan Fundamental dalam Persepsi
Interpersonal}\label{b.-kekuatan-fundamental-dalam-persepsi-interpersonal}

\begin{itemize}
\tightlist
\item
  \textbf{Stereotipe:} Generalisasi tentang kelompok saat memandang
  seseorang.
\item
  \textbf{Efek Primacy:} Kesan pertama sangat penting dan memengaruhi
  interaksi masa depan.
\item
  \textbf{Efek Recency:} Kesan paling baru lebih kuat daripada kesan
  sebelumnya.
\item
  \textbf{Perceptual Set:} Kecenderungan untuk hanya mempersepsikan apa
  yang diinginkan atau diharapkan.
\item
  \textbf{Bias Positif dan Negatif:} Kecenderungan memandang
  orang/peristiwa lebih positif atau negatif.
\item
  \textbf{Kesalahan Atribusi Fundamental:} Mengaitkan perilaku orang
  lain dengan penyebab internal (kepribadian) daripada eksternal
  (situasi).
\end{itemize}

\section{Bagian 3: Hubungan Persepsi dengan Realitas dan
Komunikasi}\label{bagian-3-hubungan-persepsi-dengan-realitas-dan-komunikasi}

\subsection{A. Persepsi dan Realitas}\label{a.-persepsi-dan-realitas}

\begin{itemize}
\tightlist
\item
  \textbf{Subjektivitas Realitas:} Realitas disimpulkan secara subjektif
  karena setiap orang memiliki ``roh karakter'' dan ``worldview'' yang
  berbeda.
\item
  \textbf{Kompleksitas Realitas:} Dunia sangat kompleks, dan indra
  menangkap banyak impuls, sehingga persepsi perlu menyaring dan menata
  informasi secara cepat.
\end{itemize}

\begin{enumerate}
\def\labelenumi{\arabic{enumi}.}
\tightlist
\item
  \textbf{Dua Tingkat Realitas:Tingkat Pertama:} Kualitas fisik yang
  dapat diamati (fakta).
\item
  \textbf{Tingkat Kedua:} Pemberian makna pada realitas tingkat pertama
  (opini/interpretasi).
\end{enumerate}

\begin{itemize}
\tightlist
\item
  \textbf{Konstruksi Realitas:} Realitas tidak ``ada di luar sana''
  dengan sendirinya, melainkan kita ciptakan bersama orang lain melalui
  komunikasi.
\item
  \textbf{Persepsi Memperkuat Konsep Diri:} Persepsi cenderung
  memperkuat konsep diri dari roh karakter kita, menciptakan siklus
  (karakter menentukan persepsi, persepsi memperkuat karakter).
\end{itemize}

\subsection{B. Persepsi dan
Komunikasi}\label{b.-persepsi-dan-komunikasi}

\begin{itemize}
\tightlist
\item
  \textbf{Persepsi sebagai Filter Komunikasi:} Pesan yang masuk
  ``disaring'' melalui persepsi, pengalaman, bias, dan keyakinan kita
  sendiri.
\item
  \textbf{Potensi Konflik:} Perbedaan persepsi, terutama jika subjektif
  dan tidak disadari, dapat menyebabkan konflik, frustrasi, dan gangguan
  dalam komunikasi dan relasi.
\item
  \textbf{Pentingnya Negosiasi Makna:} Diperlukan negosiasi makna
  bersama agar komunikasi yang memuaskan dapat terjadi, terutama ketika
  realitas tingkat kedua dianggap sebagai realitas tingkat pertama.
\item
  \textbf{Persepsi Ideal:} Berbasis fakta, cergas/gesit, memberdayakan,
  membangun relasi, dan mencerahkan.
\end{itemize}

\section{Bagian 4: Meningkatkan Kemampuan
Persepsi}\label{bagian-4-meningkatkan-kemampuan-persepsi}

\subsection{A. Strategi Peningkatan}\label{a.-strategi-peningkatan}

\begin{itemize}
\tightlist
\item
  \textbf{Mawas Diri terhadap Persepsi:} Memahami pengaruh atribut
  pribadi, budaya, dan keadaan fisiologis pada persepsi. Menyadari bias
  umum.
\item
  \textbf{Memeriksa Persepsi (Perception Checking):} Metode tiga bagian
  untuk memverifikasi interpretasi:
\end{itemize}

\begin{enumerate}
\def\labelenumi{\arabic{enumi}.}
\tightlist
\item
  Mengakui perilaku yang disaksikan.
\item
  Menawarkan dua kemungkinan interpretasi.
\item
  Meminta klarifikasi dari orang tersebut.
\end{enumerate}

\begin{itemize}
\tightlist
\item
  \textbf{Membangun Empati:} Kemampuan untuk mengalami dunia dari
  perspektif orang lain, melibatkan pengambilan perspektif, dimensi
  emosional, dan kepedulian tulus.
\item
  \textbf{Memisahkan Interpretasi dari Fakta:} Membedakan antara klaim
  faktual yang dapat diverifikasi dan opini pribadi.
\item
  \textbf{Menghasilkan Persepsi Alternatif:} Mencari cara lain untuk
  memahami situasi.
\item
  \textbf{Melatih Komunikasi Intrapersonal (Self-talk):} Memproses pesan
  intrapersonal secara cermat, menggunakan self-talk positif dan
  afirmasi.
\item
  \textbf{Memanfaatkan Self-fulfilling Prophecy secara Positif:}
  Mengubah ekspektasi menjadi lebih baik.
\item
  \textbf{Mengembangkan Kompetensi Komunikasi Antarbudaya:} Motivasi,
  sikap, toleransi terhadap ambiguitas, berpikiran terbuka, pengetahuan,
  keterampilan.
\end{itemize}

\subsection{B. Kompetensi yang Akan
Dikembangkan}\label{b.-kompetensi-yang-akan-dikembangkan}

\begin{itemize}
\tightlist
\item
  Mengenali persepsi.
\item
  Menyamakan persepsi (melalui humor, storytelling, sudut pandang orang
  lain).
\item
  Beradaptasi dengan persepsi yang berbeda.
\item
  Mengembangkan bahasa baru atau platform untuk komunikasi
  interpersonal.
\end{itemize}

\section{Bagian 5: Worldviews dan Realitas
Lainnya}\label{bagian-5-worldviews-dan-realitas-lainnya}

\subsection{A. Worldviews by Plato (Manusia
Gua)}\label{a.-worldviews-by-plato-manusia-gua}

\begin{itemize}
\tightlist
\item
  Realitas manusia gua bawah tanah dibangun oleh cahaya api unggun dan
  bayangan di dinding.
\item
  Plato mendesak keluar gua menuju cahaya matahari untuk melihat
  realitas alam yang lebih indah.
\item
  Pentingnya keseimbangan antara dimensi internal (spiritual) dan
  eksternal (visual).
\end{itemize}

\subsection{B. Lima Persepsi Realitas (Armein Z. R.
Langi)}\label{b.-lima-persepsi-realitas-armein-z.-r.-langi}

\begin{itemize}
\tightlist
\item
  \textbf{Realitas Visual: Naturalisme (KNOWLEDGE):}Realitas dibangun di
  pikiran menggunakan bahasa visual dalam lingkungan spasial (ruang).
  Ruang adalah medan cahaya, matahari adalah sumber utama.
\item
  Mengamati hal-hal secara visual sebagai alam, berkembang menurut hukum
  termodinamika (Kekekalan Energi, Entropi, Gaya Fundamental).
\item
  \textbf{Realitas Biologis: Spiritisme (ACTION):}Realitas dibangun di
  pikiran menggunakan bahasa konseptual atau abstrak tentang kekuatan,
  energi, dan makhluk berkehendak (roh).
\item
  Entitas dipengaruhi oleh gaya alam dan kehendak roh. Menambah dimensi
  waktu.
\item
  Kita adalah roh yang hidup dalam entitas alami, memiliki kehendak dan
  keinginan.
\item
  Hukum Evolusi (Survival of the fittest, Pool of genes, Mutasi).
\item
  \textbf{Realitas Bahasa: Simbolisme (SELF):}Realitas dunia baru
  dibangun oleh simbol dan bahasa. Manusia adalah \textbf{Homologos}
  (manusia kata-kata).
\item
  Eksistensi ditentukan oleh NAMA. Simbol dan Bahasa mampu
  merekonstruksi realitas.
\item
  Realitas internal subjek dikodekan ke dalam simbol bahasa dan
  dikomunikasikan.
\item
  Dunia dipahami secara simbolik di atas platform bahasa. Bidang ilmu
  pengetahuan adalah bahasa untuk memahami dunia pada skala yang sesuai.
\item
  Hukum Identitas Personal, Nilai (Personalization \& Identity: NAMES,
  Stories, History, Social/Judicial/Financial Laws).
\item
  \textbf{Realitas Kesadaran: Panteisme (BIG ORGANISM, THE
  COMMUNITY):}Realitas adalah kesadaran. Melalui cermin dan refleksi,
  kita mengamati keberadaan pengamat (kesadaran/Cahaya Pemahaman).
\item
  Kesadaran adalah roh di dalam kita, berfungsi sebagai ID, EGO, dan
  SUPEREGO. Melalui persepsi mereka, worldview dibangun.
\item
  Tuhan adalah penjumlahan kesadaran semua entitas. Realitas adalah roh
  di mana-mana.
\item
  Komunitas Bang: Keanggotaan dalam Organisme Besar (Teknologi Digital,
  Media Sosial, IoT memperluas realitas simbol dan bahasa menjadi
  realitas global).
\item
  \textbf{Realitas Allah: Teisme \& Panenteisme (VALUES):}Realitas
  adalah hubungan antara dunia spiritual dan dunia teater, di mana
  pengamat mengarahkan karakter teater dan sadar akan kesadaran
  karakter.
\item
  Dunia spiritual adalah yang nyata, dunia teater adalah arena bermain.
  Sutradara spiritual adalah Tuhan.
\item
  Kita sadar tentang Kesadaran Lain tentang kita (Tuhan).
\item
  Gods Bang: Lebih dari 5000 Dewa.
\item
  Value Bang: Hukum Sekuritisasi Nilai (Uang Digital, Token, Kontrak,
  Blockchain).
\end{itemize}

\section{Kuis: Persepsi dan Realitas}\label{kuis-persepsi-dan-realitas}

\textbf{Petunjuk:} Jawab setiap pertanyaan dalam 2-3 kalimat
\href{https://forms.office.com/r/HPKrDLU2DN}{submit}

\begin{enumerate}
\def\labelenumi{\arabic{enumi}.}
\tightlist
\item
  Jelaskan mengapa persepsi dianggap menyelamatkan dan penting untuk
  keselamatan kita.
\item
  Sebutkan tiga tahapan utama dalam proses pembentukan persepsi dan
  jelaskan secara singkat masing-masing.
\item
  Berikan contoh bagaimana perbedaan budaya atau kokultur dapat
  menyebabkan perbedaan persepsi dalam komunikasi verbal.
\item
  Apa yang dimaksud dengan ``Kesalahan Atribusi Fundamental'' dalam
  konteks persepsi interpersonal?
\item
  Bagaimana hubungan antara ``konsep diri'' dan ``persepsi'' menurut
  materi sumber?
\item
  Jelaskan konsep ``pungtuasi'' dan bagaimana ia memengaruhi cara kita
  melihat suatu perselisihan.
\item
  Apa yang dimaksud dengan ``persepsi ideal'' dan sebutkan dua
  karakternya.
\item
  Bagaimana teknologi digital seperti media sosial dikaitkan dengan
  perluasan realitas dalam pandangan ``Panteisme''?
\item
  Menurut pandangan ``Simbolisme'', mengapa manusia disebut
  ``Homologos''?
\item
  Sebutkan tiga cara praktis untuk meningkatkan kemampuan persepsi kita.
\end{enumerate}

\section{Pertanyaan Esai (Tanpa
Jawaban)}\label{pertanyaan-esai-tanpa-jawaban}

\begin{enumerate}
\def\labelenumi{\arabic{enumi}.}
\tightlist
\item
  Analisis bagaimana persepsi, menurut materi, bukan hanya sekadar alat
  untuk memahami realitas tetapi juga secara aktif membentuk realitas
  tersebut. Diskusikan implikasinya terhadap komunikasi interpersonal.
\item
  Bandingkan dan kontraskan bagaimana faktor fisiologis, psikologis,
  sosial, dan budaya memengaruhi pembentukan persepsi seseorang. Berikan
  contoh spesifik dari materi untuk mendukung argumen Anda.
\item
  Jelaskan proses empat tahapan pembentukan persepsi (seleksi,
  organisasi, interpretasi, dan negosiasi) secara rinci. Mengapa tahapan
  negosiasi, meskipun kadang tidak selalu disebut, sangat penting dalam
  konteks persepsi interpersonal?
\item
  Pilih dua dari lima pandangan realitas (Naturalisme, Spiritisme,
  Simbolisme, Panteisme, Teisme/Panenteisme) yang dijelaskan dalam
  materi. Bandingkan bagaimana masing-masing pandangan tersebut
  mengkonstruksi realitas dan apa yang menjadi fokus utamanya (misalnya,
  pengetahuan, tindakan, diri, organisme besar, nilai).
\item
  Diskusikan pentingnya ``membiasakan diri terhadap persepsi'' dan
  ``membangun empati'' sebagai strategi untuk meningkatkan kemampuan
  persepsi. Bagaimana kedua strategi ini saling melengkapi dalam
  meminimalkan konflik dan meningkatkan komunikasi yang efektif?
\end{enumerate}

\section{Glosarium Istilah Kunci}\label{glosarium-istilah-kunci-2}

\begin{itemize}
\tightlist
\item
  \textbf{Persepsi:} Proses aktif menata, mengorganisir,
  mengidentifikasi, dan menafsirkan informasi indrawi untuk memahami
  lingkungan.
\item
  \textbf{Persepsi Interpersonal:} Proses persepsi yang diterapkan pada
  orang dan hubungan.
\item
  \textbf{Konsep Diri (Self-concept):} Pemahaman individu tentang
  dirinya sendiri; seperangkat persepsi yang relatif stabil yang
  membentuk tindakan dan pilihan.
\item
  \textbf{Seleksi (Selection):} Tahap awal persepsi di mana indra
  terstimulasi dan kita memilih rangsangan yang intens, berulang, atau
  kontras.
\item
  \textbf{Organisasi (Organization):} Tahap di mana informasi yang
  dipilih diatur secara bermakna, sering menggunakan skema persepsi.
\item
  \textbf{Interpretasi (Interpretation):} Tahap di mana makna diberikan
  pada rangsangan yang diorganisir, dipengaruhi oleh ekspektasi,
  pengalaman, dan kepribadian.
\item
  \textbf{Negosiasi (Negotiation):} Proses di mana komunikator saling
  memengaruhi persepsi satu sama lain melalui pertukaran cerita atau
  narasi.
\item
  \textbf{Skema Persepsi:} Kerangka kognitif yang digunakan untuk
  mengkategorikan informasi selama tahap organisasi persepsi (misalnya,
  konstruk fisik, peran, interaksi, psikologis).
\item
  \textbf{Pungtuasi:} Konsep dalam organisasi persepsi yang mengacu pada
  penentuan sebab dan akibat dalam serangkaian interaksi.
\item
  \textbf{Stereotipe:} Generalisasi tentang suatu kelompok yang
  diterapkan pada individu saat membentuk persepsi.
\item
  \textbf{Efek Primacy:} Pengaruh kuat dari kesan pertama dalam
  membentuk persepsi dan interaksi di masa depan.
\item
  \textbf{Efek Recency:} Pengaruh kuat dari kesan paling baru dalam
  membentuk persepsi.
\item
  \textbf{Perceptual Set:} Kecenderungan untuk hanya mempersepsikan apa
  yang kita inginkan atau harapkan.
\item
  \textbf{Bias Positif dan Negatif:} Kecenderungan untuk memandang orang
  atau peristiwa secara lebih positif atau negatif.
\item
  \textbf{Kesalahan Atribusi Fundamental:} Kecenderungan untuk
  mengaitkan perilaku orang lain dengan penyebab internal daripada
  eksternal.
\item
  \textbf{Mawas Diri terhadap Persepsi:} Memahami bagaimana atribut
  pribadi, budaya, dan keadaan fisiologis memengaruhi persepsi kita.
\item
  \textbf{Memeriksa Persepsi (Perception Checking):} Metode tiga bagian
  untuk memverifikasi keakuratan interpretasi perilaku orang lain.
\item
  \textbf{Empati:} Kemampuan untuk mengalami dunia dari perspektif orang
  lain, termasuk mengambil perspektif dan merasakan emosi mereka.
\item
  \textbf{Homologos:} Istilah yang menggambarkan manusia sebagai
  ``manusia kata-kata'' atau ``manusia bahasa'' dalam pandangan
  Simbolisme.
\item
  \textbf{Realitas Tingkat Pertama:} Kualitas fisik yang dapat diamati
  dari suatu hal atau situasi (fakta objektif).
\item
  \textbf{Realitas Tingkat Kedua:} Makna atau interpretasi yang kita
  berikan pada realitas tingkat pertama (subjektif).
\item
  \textbf{Naturalisme (Persepsi Realitas):} Pandangan dunia yang
  mengkonstruksi realitas melalui bahasa visual, berfokus pada
  pengamatan alam dan hukum termodinamika.
\item
  \textbf{Spiritisme (Persepsi Realitas):} Pandangan dunia yang
  mengkonstruksi realitas melalui bahasa konseptual tentang kekuatan,
  energi, dan roh, berfokus pada kehendak dan hukum evolusi.
\item
  \textbf{Simbolisme (Persepsi Realitas):} Pandangan dunia yang
  mengkonstruksi realitas melalui simbol dan bahasa, menekankan peran
  nama, cerita, dan identitas.
\item
  \textbf{Panteisme (Persepsi Realitas):} Pandangan dunia yang
  mengkonstruksi realitas sebagai kesadaran, di mana Tuhan adalah
  penjumlahan kesadaran semua entitas dan realitas adalah roh di
  mana-mana.
\item
  \textbf{Teisme \& Panenteisme (Persepsi Realitas):} Pandangan dunia
  yang mengkonstruksi realitas sebagai hubungan antara dunia spiritual
  (nyata) dan dunia teater (arena bermain) yang diarahkan oleh Tuhan.
\end{itemize}

\bookmarksetup{startatroot}

\chapter{Kuliah 4: Kekuatan dan Penggunaan Bahasa dalam
Komunikasi}\label{kuliah-4-kekuatan-dan-penggunaan-bahasa-dalam-komunikasi}

Video Clip:
\url{https://youtube.com/playlist?list=PL_m-BplfO92F71yM3fOiHWCAROtz5P472&si=VTVuxqQgzYDJwq4m}

Submit Kuis: \url{https://forms.office.com/r/50GSJJfDPW}

\section{Sesi 1: Pengantar: Kekuatan, Sifat, dan Penerapan Bahasa dalam
Komunikasi
Interpersonal}\label{sesi-1-pengantar-kekuatan-sifat-dan-penerapan-bahasa-dalam-komunikasi-interpersonal}

\subsection{Ringkasan Eksekutif}\label{ringkasan-eksekutif}

Dokumen ini menyajikan sintesis komprehensif mengenai peran fundamental
bahasa dalam komunikasi interpersonal, berdasarkan analisis mendalam
dari berbagai sumber. Poin-poin utama menyoroti bahwa bahasa adalah
sistem simbol yang kuat dan terstruktur, yang tidak hanya berfungsi
sebagai alat transmisi informasi, tetapi juga secara aktif membentuk
persepsi, memengaruhi perilaku, dan mendefinisikan hubungan. Kekuatan
bahasa termanifestasi dalam berbagai bentuk, mulai dari identitas yang
melekat pada sebuah nama, kekuatan persuasif melalui etos, patos, dan
logos, hingga kemampuannya untuk mengekspresikan kasih sayang dan
memberikan kenyamanan.

Efektivitas komunikasi sangat bergantung pada penciptaan iklim yang
positif, yang dicapai dengan menggunakan pesan-pesan yang mengkonfirmasi
eksistensi dan perasaan orang lain, serta mengambil kepemilikan atas
emosi pribadi melalui ``pernyataan-saya'' (I-statements). Sebaliknya,
respons diskonfirmasi dan defensif dapat merusak interaksi. Analisis
juga menggarisbawahi pentingnya membedakan antara klaim faktual yang
dapat diverifikasi dan opini yang bersifat subjektif. Lebih jauh,
dokumen ini mengkaji penggunaan dan penyalahgunaan bahasa melalui humor,
eufemisme, slang, dan bentuk-bentuk yang merusak seperti pencemaran nama
baik dan ujaran kebencian. Terakhir, disajikan panduan praktis untuk
menavigasi tantangan unik dalam komunikasi di era digital, menekankan
perlunya kesabaran, kehati-hatian, dan kesadaran akan keterbatasan
medium.

\subsection{I. Kekuatan Fundamental
Kata-Kata}\label{i.-kekuatan-fundamental-kata-kata}

Kata-kata memiliki kekuatan yang signifikan untuk mengidentifikasi
individu, membujuk, membangun keintiman, dan memberikan kenyamanan.
Pemahaman terhadap kekuatan ini merupakan inti dari komunikasi
interpersonal yang efektif.

\subsubsection{A. Nama Sebagai Identitas dan
Simbol}\label{a.-nama-sebagai-identitas-dan-simbol}

Nama adalah perangkat linguistik kuat yang berfungsi sebagai kata
pertama yang mengidentifikasi dan melambangkan diri seseorang. Nama
sering kali menjadi informasi awal yang dipelajari tentang orang lain
dan dapat membawa berbagai informasi demografis.

\begin{itemize}
\item
  \textbf{Informasi Demografis:} Nama dapat memberikan asumsi---meskipun
  tidak selalu akurat---tentang jenis kelamin, etnis, atau bahkan era
  kelahiran seseorang. Popularitas nama cenderung berfluktuasi seiring
  waktu, sehingga nama tertentu dapat diasosiasikan dengan generasi
  tertentu.
\item
  \textbf{Stereotip dan Diskriminasi:} Asumsi berdasarkan nama terkadang
  bisa berbahaya dan mengarah pada diskriminasi. Fenomena penggantian
  nama pernah terjadi di Indonesia sebagai respons terhadap diskriminasi
  terhadap nama-nama tertentu.
\item
  \textbf{Simbol Cerita:} Setiap nama dianggap mewakili sebuah cerita
  kehidupan, menjadi bagian tak terpisahkan dari narasi pribadi
  seseorang.
\end{itemize}

\subsubsection{B. Kekuatan Persuasi: Etos, Patos, dan
Logos}\label{b.-kekuatan-persuasi-etos-patos-dan-logos}

Persuasi adalah proses menggunakan kata-kata untuk menggerakkan orang
lain agar memiliki pemikiran atau melakukan tindakan tertentu. Filsuf
Yunani Aristoteles mengidentifikasi tiga pilar persuasi yang masih
relevan hingga kini.

\begin{enumerate}
\def\labelenumi{\arabic{enumi}.}
\item
  \textbf{Etos (Kredibilitas):} Ini berkaitan dengan bagaimana audiens
  memandang kompetensi dan kepercayaan terhadap pembicara. Kredibilitas
  yang tinggi membuat pesan lebih persuasif. Faktor-faktor yang
  memengaruhi etos meliputi:

  \begin{itemize}
  \item
    \textbf{Pilihan Bahasa:} Penggunaan bahasa yang mencerminkan
    pengetahuan tinggi dapat meningkatkan kredibilitas.
  \item
    \textbf{Faktor Perusak Kredibilitas:}

    \begin{itemize}
    \item
      \textbf{Klise:} Ungkapan yang sudah usang dan terlalu sering
      digunakan (misalnya, ``berpikir \emph{out of the box}'').
    \item
      \textbf{Dialek:} Dialek yang tidak cocok atau berkonotasi negatif
      dapat memengaruhi persepsi audiens.
    \item
      \textbf{Bahasa Ragu-ragu (Equivocation):} Pernyataan yang sengaja
      dibuat kabur dan tidak jelas.
    \item
      \textbf{\emph{Weasel Words}} \textbf{(Kata-kata Musang):}
      Kata-kata terselubung yang menyiratkan sesuatu yang tidak
      sepenuhnya benar untuk memengaruhi pendengar (misalnya, ``empat
      dari lima dokter gigi merekomendasikan,'' tanpa menjelaskan
      konteks survei yang sebenarnya).
    \item
      \textbf{Pernyataan Absolut:} Klaim yang mencakup ``semua'' tanpa
      pengecualian (misalnya, ``semua pemerintah bobrok''), yang merusak
      integritas karena kurangnya spesifisitas.
    \end{itemize}
  \end{itemize}
\item
  \textbf{Patos (Emosi):} Dianggap sebagai alat persuasi yang paling
  kuat, patos bekerja dengan memicu respons emosional audiens terhadap
  suatu topik, membuat mereka lebih cenderung menerima klaim yang
  diajukan.
\item
  \textbf{Logos (Logika):} Ini adalah daya tarik terhadap akal budi dan
  logika. Penggunaan fakta, data, dan statistik untuk meyakinkan audiens
  berdasarkan bukti rasional.
\end{enumerate}

\subsubsection{C. Ekspresi Kasih Sayang dan
Keintiman}\label{c.-ekspresi-kasih-sayang-dan-keintiman}

Kata-kata memiliki kekuatan untuk mengekspresikan kasih sayang, yang
merupakan komponen vital dalam menjaga hubungan interpersonal.
Komunikasi afektif tidak hanya baik untuk kesehatan hubungan tetapi juga
untuk kesehatan individu. Sebuah studi menunjukkan bahwa pasangan yang
sering mengkomunikasikan kasih sayang dalam dua tahun pertama pernikahan
cenderung memiliki hubungan yang lebih langgeng 13 tahun kemudian.

\subsubsection{D. Memberikan Kenyamanan dan
Dukungan}\label{d.-memberikan-kenyamanan-dan-dukungan}

Bahasa dapat berfungsi sebagai alat untuk menghibur orang lain yang
sedang mengalami kesulitan atau diri sendiri.

\begin{itemize}
\item
  \textbf{Menghibur Orang Lain:} Ungkapan sederhana seperti, ``Saya
  turut prihatin kamu mengalami ini,'' dapat menunjukkan empati,
  mengakui perasaan orang tersebut, dan menawarkan dukungan tulus.
\item
  \textbf{Menghibur Diri Sendiri:} Mengungkapkan kesusahan hati kepada
  orang lain, menuliskannya dalam buku harian, atau berbagi melalui blog
  terbukti dapat memberikan kelegaan, menurunkan tingkat stres, dan
  meningkatkan perasaan positif terhadap diri sendiri.
\end{itemize}

\subsection{II. Sifat dan Struktur
Bahasa}\label{ii.-sifat-dan-struktur-bahasa}

Bahasa adalah sistem komunikasi unik yang dimiliki manusia, dicirikan
oleh penggunaan simbol-simbol terstruktur (kata-kata) yang diatur oleh
seperangkat aturan kompleks dan memiliki makna berlapis.

\subsubsection{A. Bahasa Sebagai Sistem Simbol yang
Arbitrer}\label{a.-bahasa-sebagai-sistem-simbol-yang-arbitrer}

Bahasa pada dasarnya bersifat simbolik. Kata-kata mewakili ide atau
objek, tetapi kata itu sendiri bukanlah objek yang diwakilinya. Hubungan
antara kata dan maknanya bersifat \textbf{arbitrer}, artinya makna
tersebut ada karena kesepakatan bersama para penggunanya. Karena sifat
arbitrer ini, makna kata dapat berubah seiring waktu sesuai dengan
kesepakatan sosial yang baru.

\subsubsection{B. Aturan-Aturan yang Mengatur
Bahasa}\label{b.-aturan-aturan-yang-mengatur-bahasa}

Agar dapat dipahami, bahasa diatur oleh empat jenis aturan yang
digunakan secara intuitif oleh penutur yang mahir.

\begin{itemize}
\item
  \textbf{Aturan Fonologis:} Mengatur bagaimana bunyi diucapkan untuk
  membentuk kata (\emph{pronunciation}).
\item
  \textbf{Aturan Sintaksis:} Mengatur cara kata-kata digabungkan untuk
  membentuk kalimat yang benar secara gramatikal.
\item
  \textbf{Aturan Semantik:} Berkaitan dengan makna dari setiap kata.
\item
  \textbf{Aturan Pragmatis:} Mengatur interpretasi makna berdasarkan
  konteks sosial dan budaya saat kata itu digunakan.
\end{itemize}

\subsubsection{C. Lapisan Makna: Denotatif dan
Konotatif}\label{c.-lapisan-makna-denotatif-dan-konotatif}

Kata-kata sering kali memiliki lebih dari satu lapisan makna, yang dapat
dibedakan menjadi dua jenis utama.

\begin{longtable}[]{@{}
  >{\raggedright\arraybackslash}p{(\linewidth - 4\tabcolsep) * \real{0.3333}}
  >{\raggedright\arraybackslash}p{(\linewidth - 4\tabcolsep) * \real{0.3333}}
  >{\raggedright\arraybackslash}p{(\linewidth - 4\tabcolsep) * \real{0.3333}}@{}}
\toprule\noalign{}
\endhead
\bottomrule\noalign{}
\endlastfoot
Jenis Makna & Deskripsi & Contoh: Kata ``Rumah'' \\
\textbf{Denotatif} & Makna literal, objektif, atau definisi kamus dari
sebuah kata. & Sebuah bangunan fisik tempat tinggal. \\
\textbf{Konotatif} & Makna tersirat, subjektif, dan emosional yang
melekat pada sebuah kata, yang bisa berbeda bagi setiap orang. & Tempat
berkumpul keluarga, rasa aman, kehangatan, atau tempat pulang setelah
beraktivitas. \\
\end{longtable}

\begin{itemize}
\item
  \textbf{Segitiga Semantik:} Model ini menggambarkan hubungan antara
  tiga elemen: \textbf{simbol} (kata itu sendiri), \textbf{makna
  denotatif} (objek atau referen yang ditunjuk), dan \textbf{makna
  konotatif} (pikiran atau referensi subjektif yang muncul di benak
  pendengar).
\item
  \textbf{\emph{Loaded Language}} \textbf{(Bahasa Berbobot):} Ini adalah
  kata-kata yang makna denotatifnya mungkin netral, tetapi memiliki
  konotasi emosional yang sangat kuat (positif atau negatif). Contohnya
  termasuk ``kanker,'' ``merdeka,'' ``keluarga,'' dan ``damai.''
  Kata-kata ini sangat efektif dalam persuasi karena kemampuannya
  membangkitkan emosi yang kuat.
\end{itemize}

\subsubsection{D. Variasi Bahasa: Kejelasan dan
Abstraksi}\label{d.-variasi-bahasa-kejelasan-dan-abstraksi}

\begin{itemize}
\item
  \textbf{Kejelasan:} Karena banyak kata memiliki lebih dari satu arti,
  bahasa dapat menjadi ambigu. Kejelasan sering kali bergantung pada
  konteks kalimat atau situasi di mana kata tersebut digunakan.
\item
  \textbf{Abstraksi:} Bahasa bervariasi dari konkret ke abstrak.

  \begin{itemize}
  \item
    \textbf{Konkret:} Merujuk pada sesuatu yang dapat diidentifikasi
    secara spesifik dan dapat dirasakan oleh indra.
  \item
    \textbf{Abstrak:} Merujuk pada konsep, ide, atau kategori yang lebih
    luas yang tidak dapat ditunjuk secara fisik.
  \end{itemize}
\end{itemize}

\subsubsection{E. Hubungan Bahasa dan Budaya: Hipotesis
Sapir-Whorf}\label{e.-hubungan-bahasa-dan-budaya-hipotesis-sapir-whorf}

Bahasa dan budaya saling terkait erat. Budaya (misalnya, kolektivis
vs.~individualistis) memengaruhi pilihan kata dan gaya komunikasi.
Hipotesis Sapir-Whorf mengemukakan bahwa bahasa membentuk cara kita
memandang dunia melalui dua prinsip:

\begin{enumerate}
\def\labelenumi{\arabic{enumi}.}
\item
  \textbf{Determinisme Linguistik:} Struktur bahasa \emph{menentukan}
  cara kita berpikir. Jika sebuah bahasa tidak memiliki kata untuk suatu
  konsep, maka penuturnya tidak dapat memikirkan konsep tersebut.
\item
  \textbf{Relativitas Linguistik:} Orang yang menggunakan bahasa berbeda
  akan melihat dan memahami dunia dengan cara yang berbeda pula.
\end{enumerate}

\subsection{III. Membangun Iklim Komunikasi yang
Positif}\label{iii.-membangun-iklim-komunikasi-yang-positif}

Menciptakan suasana komunikasi yang kondusif dan suportif adalah kunci
untuk interaksi yang sukses. Ini melibatkan penggunaan pesan yang
membangun, menghindari defensif, dan memahami perbedaan antara fakta dan
opini.

\subsubsection{A. Pesan Mengkonfirmasi
vs.~Diskonfirmasi}\label{a.-pesan-mengkonfirmasi-vs.-diskonfirmasi}

Cara kita merespons orang lain dapat membangun atau merusak iklim
komunikasi.

\begin{longtable}[]{@{}
  >{\raggedright\arraybackslash}p{(\linewidth - 4\tabcolsep) * \real{0.3333}}
  >{\raggedright\arraybackslash}p{(\linewidth - 4\tabcolsep) * \real{0.3333}}
  >{\raggedright\arraybackslash}p{(\linewidth - 4\tabcolsep) * \real{0.3333}}@{}}
\toprule\noalign{}
\endhead
\bottomrule\noalign{}
\endlastfoot
Tipe Pesan & Deskripsi & Contoh Respons \\
\textbf{Mengkonfirmasi} & Mengakui keberadaan, perasaan, dan pikiran
orang lain, bahkan jika kita tidak setuju. Ini menunjukkan bahwa kita
mendengar dan menghargai sudut pandang mereka. & ``Saya bisa memahami
mengapa kamu merasa seperti itu.'' \\
\textbf{Diskonfirmasi} & Mengabaikan atau menolak nilai orang lain. Ini
membuat mereka merasa tidak dihargai atau tidak ada. & Diam, mengalihkan
topik, respons tidak relevan, generalisasi (``Kamu selalu\ldots{}''),
atau jawaban impersonal (``Yah, hidup memang berat.''). \\
\end{longtable}

\subsubsection{B. Strategi Menghindari Respons
Defensif}\label{b.-strategi-menghindari-respons-defensif}

Untuk berkomunikasi secara efektif tanpa membuat lawan bicara membela
diri, beberapa strategi dapat diterapkan:

\begin{itemize}
\item
  \textbf{Deskriptif, Bukan Evaluatif:} Jelaskan apa yang terjadi, bukan
  menghakimi (``Tugas ini perlu lebih detail'' vs.~``Kerjaanmu jelek'').
\item
  \textbf{Berorientasi pada Masalah, Bukan Kontrol:} Ajak orang lain
  berdiskusi untuk mencari solusi bersama, bukan memaksakan kehendak.
\item
  \textbf{Tunjukkan Empati dan Kepedulian:} Pastikan orang lain tahu
  bahwa kita peduli dengan perasaan mereka.
\item
  \textbf{Hindari Superioritas:} Bersikap terbuka terhadap ide orang
  lain dan jangan bersikap seolah-olah serba tahu.
\item
  \textbf{Memberikan Umpan Balik Secara Konstruktif:} Jika umpan balik
  diperlukan, mulailah dengan hal positif sebelum menyampaikan area yang
  perlu perbaikan.
\end{itemize}

\subsubsection{\texorpdfstring{C. Kepemilikan Perasaan: Pernyataan
``Saya'' vs.~``Kamu'' (\emph{I-Statements}
vs.~\emph{You-Statements})}{C. Kepemilikan Perasaan: Pernyataan ``Saya'' vs.~``Kamu'' (I-Statements vs.~You-Statements)}}\label{c.-kepemilikan-perasaan-pernyataan-saya-vs.-kamu-i-statements-vs.-you-statements}

Komunikator yang baik mengambil kepemilikan atas pikiran dan perasaan
mereka sendiri.

\begin{itemize}
\item
  \textbf{\emph{You-Statement}} \textbf{(Menyalahkan):} ``Kamu membuat
  saya marah karena tidak mengunci gerbang.'' Pernyataan ini menempatkan
  tanggung jawab emosi pada orang lain dan cenderung memicu defensif.
\item
  \textbf{\emph{I-Statement}} \textbf{(Mengambil Kepemilikan):} ``Saya
  merasa kesal saat menemukan gerbang tidak terkunci karena saya
  khawatir tentang keamanan.'' Pernyataan ini mengakui perasaan pribadi
  tanpa menyalahkan, membuka jalan untuk solusi konstruktif.
\end{itemize}

\subsubsection{D. Membedakan Fakta dan
Opini}\label{d.-membedakan-fakta-dan-opini}

Banyak konflik muncul karena kegagalan membedakan antara klaim faktual
dan opini.

\begin{itemize}
\item
  \textbf{Klaim Faktual:} Pernyataan yang dapat diverifikasi
  kebenarannya (benar atau salah) dengan bukti.
\item
  \textbf{Opini:} Pernyataan yang mengungkapkan penilaian atau keyakinan
  pribadi. Respons yang tepat untuk opini adalah ``setuju'' atau ``tidak
  setuju,'' bukan ``benar'' atau ``salah.'' Menyatakan opini orang lain
  ``salah'' tidak produktif dan hanya memicu perdebatan sia-sia.
\end{itemize}

\subsection{IV. Penggunaan dan Penyalahgunaan
Bahasa}\label{iv.-penggunaan-dan-penyalahgunaan-bahasa}

Bahasa dapat digunakan untuk tujuan positif seperti membangun keakraban
melalui humor, atau disalahgunakan untuk menipu, mengucilkan, dan
menyakiti orang lain.

\subsubsection{A. Humor: Pembangun Hubungan dan Potensi
Kerusakan}\label{a.-humor-pembangun-hubungan-dan-potensi-kerusakan}

Humor muncul dari pelanggaran terhadap ekspektasi.

\begin{itemize}
\item
  \textbf{Penggunaan Positif:} Meredakan ketegangan, membuat orang
  merasa nyaman, dan memperkuat keintiman saat tertawa bersama.
\item
  \textbf{Penyalahgunaan:} Lelucon yang merendahkan, kejam, melecehkan
  fisik, atau menyinggung kelompok sosial tertentu dapat menyebabkan
  kerusakan emosional yang serius.
\end{itemize}

\subsubsection{B. Eufemisme: Penghalusan Bahasa dan
Desensitisasi}\label{b.-eufemisme-penghalusan-bahasa-dan-desensitisasi}

Eufemisme adalah ekspresi halus yang digunakan untuk menggantikan
istilah yang dianggap kasar atau tidak menyenangkan.

\begin{itemize}
\item
  \textbf{Penggunaan Positif:} Membantu membahas topik yang sensitif
  atau tidak nyaman secara lebih sopan (misalnya, ``merampingkan'' untuk
  PHK).
\item
  \textbf{Penyalahgunaan:} Dapat menyebabkan desensitisasi terhadap
  hal-hal yang buruk atau kejam, seperti penggunaan istilah
  ``\emph{collateral damage}'' untuk korban sipil dalam perang.
\end{itemize}

\subsubsection{C. Slang dan Jargon: Inklusi dan
Eksklusi}\label{c.-slang-dan-jargon-inklusi-dan-eksklusi}

\begin{itemize}
\item
  \textbf{Slang:} Bahasa informal yang digunakan oleh kelompok sosial
  atau co-culture tertentu (berdasarkan usia, daerah, dll.).
\item
  \textbf{Jargon:} Kosakata teknis atau khusus yang digunakan dalam
  suatu profesi atau bidang tertentu.
\item
  \textbf{Dampak Ganda:} Di satu sisi, slang dan jargon dapat memperkuat
  identitas dan rasa kebersamaan di dalam kelompok. Di sisi lain,
  keduanya dapat mengucilkan atau membuat orang di luar kelompok merasa
  tersisih.
\end{itemize}

\subsubsection{D. Penyalahgunaan Destruktif: Pencemaran Nama Baik, Kata
Vulgar, dan Ujaran
Kebencian}\label{d.-penyalahgunaan-destruktif-pencemaran-nama-baik-kata-vulgar-dan-ujaran-kebencian}

Ini adalah contoh ekstrem dari penyalahgunaan kekuatan bahasa.

\begin{itemize}
\item
  \textbf{Pencemaran Nama Baik (Fitnah):} Pernyataan yang sengaja
  disebarkan untuk merusak reputasi seseorang. Ini bisa berupa
  \textbf{Slander} (lisan) atau \textbf{Libel} (tertulis). Pernyataan
  tersebut bisa jadi tidak benar, atau bahkan informasi benar yang
  bersifat pribadi dan disebarkan tanpa kepentingan publik untuk tujuan
  merendahkan.
\item
  \textbf{Kata-kata Vulgar (Profanitas):} Bahasa kasar yang sering
  digunakan untuk mengungkapkan keterkejutan, kemarahan, atau untuk
  menghina orang lain. Meskipun dalam konteks pertemanan yang sangat
  dekat dapat ditoleransi, secara umum penggunaannya harus dihindari.
\item
  \textbf{Ujaran Kebencian (\emph{Hate Speech}):} Bentuk bahasa yang
  secara spesifik ditujukan untuk merendahkan, mengintimidasi, atau
  mengobarkan kekerasan terhadap individu atau kelompok berdasarkan
  karakteristik tertentu seperti etnis, agama, jenis kelamin, atau
  orientasi seksual.
\end{itemize}

\subsection{V. Panduan Komunikasi di Era
Digital}\label{v.-panduan-komunikasi-di-era-digital}

Komunikasi melalui platform digital memiliki karakteristik unik yang
menuntut kesadaran dan kehati-hatian ekstra.

\begin{enumerate}
\def\labelenumi{\arabic{enumi}.}
\item
  \textbf{Jangan Mengharapkan Respons Instan:} Bersabarlah dan jangan
  mudah tersinggung jika pesan tidak segera dibalas. Kita tidak pernah
  tahu situasi yang sedang dihadapi oleh penerima pesan.
\item
  \textbf{Hindari Platform Digital untuk Isu Sensitif:} Jangan
  menggunakan email atau pesan teks untuk menyampaikan keputusan penting
  yang berdampak besar pada hubungan personal, seperti pemutusan
  hubungan kerja atau perceraian. Hal-hal seperti ini lebih baik
  dibicarakan secara langsung.
\item
  \textbf{Minta Izin Sebelum Berbagi:} Selalu minta izin sebelum
  membagikan foto atau informasi pribadi orang lain.
\item
  \textbf{Waspada terhadap Koreksi Otomatis (\emph{Autocorrect}):}
  Selalu periksa kembali tulisan sebelum menekan tombol kirim untuk
  menghindari kesalahpahaman fatal akibat koreksi otomatis.
\item
  \textbf{Tafsirkan Pesan dengan Hati-hati:} Ingatlah bahwa komunikasi
  digital bersifat ``ramping'' dan sering kali kehilangan nuansa
  nonverbal. Sebelum bereaksi negatif, pertimbangkan kemungkinan
  interpretasi lain dari pesan yang diterima dan jangan ragu untuk
  meminta klarifikasi.
\end{enumerate}

\section{SESI 2: Kekuatan Tersembunyi di Balik Kata-Kata: Panduan untuk
Memahami dan
Menggunakannya}\label{sesi-2-kekuatan-tersembunyi-di-balik-kata-kata-panduan-untuk-memahami-dan-menggunakannya}

\subsection{\texorpdfstring{1. \textbf{Pendahuluan: Lebih dari Sekadar
Ucapan}}{1. Pendahuluan: Lebih dari Sekadar Ucapan}}\label{pendahuluan-lebih-dari-sekadar-ucapan}

Dari semua makhluk hidup yang kita kenal, hanya manusia yang memiliki
kemampuan canggih untuk belajar dan menggunakan simbol-simbol
terstruktur yang kita sebut `bahasa'. Kemampuan ini memungkinkan kita
untuk mewakili gagasan, perasaan, dan pikiran kita kepada orang lain.
Namun, kata-kata bukan hanya sekadar alat untuk menyampaikan informasi.
Kata-kata adalah perangkat yang sangat kuat, mampu membentuk pikiran,
membangkitkan perasaan, dan mendorong tindakan.

Dokumen ini akan mengajak Anda menjelajahi berbagai dimensi kekuatan
kata-kata. Kita akan melihat bagaimana bahasa dapat digunakan untuk
membujuk, menunjukkan kasih sayang, memberikan penghiburan, dan
bagaimana kita dapat memegang kekuatan ini dengan lebih bijaksana dan
bertanggung jawab.

\begin{center}\rule{0.5\linewidth}{0.5pt}\end{center}

\subsection{2. Kekuatan Membujuk: Menggerakkan Hati dan
Pikiran}\label{kekuatan-membujuk-menggerakkan-hati-dan-pikiran}

Setiap hari, kita menghabiskan banyak waktu untuk membujuk (mempersuasi)
orang lain atau sebaliknya, dibujuk oleh mereka. \textbf{Persuasi}
adalah proses menggerakkan orang lain untuk memiliki pemikiran tertentu
atau melakukan tindakan dengan cara tertentu. Ribuan tahun yang lalu,
filsuf Yunani Aristoteles mengidentifikasi tiga pilar fundamental yang
membuat sebuah pesan menjadi persuasif: Ethos, Pathos, dan Logos. Hingga
hari ini, ketiganya masih menjadi inti dari komunikasi yang efektif.

\subsubsection{2.1. Ethos: Kekuatan
Kredibilitas}\label{ethos-kekuatan-kredibilitas}

Ethos adalah daya tarik yang berpusat pada \textbf{kredibilitas dan
karakter} pembicara. Sederhananya, mengapa kita lebih percaya pada
nasihat dokter tentang kesehatan daripada nasihat orang asing di jalan?
Jawabannya adalah Ethos. Audiens akan lebih mudah dibujuk jika mereka
memandang pembicara sebagai seseorang yang kompeten, dapat dipercaya,
dan memiliki integritas. Ketika audiens menghormati Anda, mereka akan
lebih terbuka terhadap pesan yang Anda sampaikan.

Namun, sama seperti membangun kredibilitas, pilihan kata yang salah
justru dapat merusaknya. Beberapa jebakan bahasa yang umum dapat membuat
seorang pembicara terdengar kurang kredibel, di antaranya:

\begin{itemize}
\item
  \textbf{Klise (Clichés):} Ini adalah frasa yang terlalu sering
  digunakan hingga kehilangan maknanya dan terdengar basi. Mengatakan,
  ``Mari kita berpikir \emph{out of the box},'' bisa membuat seorang
  pembicara terdengar tidak orisinal.
\item
  \textbf{Dialek (Dialects):} Meskipun tidak adil, terkadang dialek
  kedaerahan atau sosial tertentu dapat memengaruhi persepsi audiens
  terhadap kredibilitas seorang pembicara.
\item
  \textbf{Bahasa yang Ragu-ragu:} Menggunakan kata-kata yang sengaja
  dibuat kabur atau tidak jelas dapat membuat audiens merasa pembicara
  tidak dapat dipercaya atau tidak berkomitmen pada ucapannya.
\end{itemize}

\subsubsection{2.2. Pathos: Kekuatan Emosi}\label{pathos-kekuatan-emosi}

Pathos adalah daya tarik yang bertujuan untuk \textbf{membangkitkan
emosi} pendengar. Dari ketiga pilar, Pathos seringkali dianggap sebagai
alat persuasi yang paling kuat karena keputusan manusia seringkali
didorong oleh perasaan, bukan murni logika. Iklan yang membuat kita
terharu, pidato yang membangkitkan semangat kebangsaan, atau cerita yang
membuat kita merasa bahagia adalah contoh penggunaan Pathos untuk memicu
respons emosional dan membuat pesan lebih mudah diterima.

\subsubsection{2.3. Logos: Kekuatan Logika}\label{logos-kekuatan-logika}

Logos adalah daya tarik yang menggunakan \textbf{akal sehat, fakta, dan
statistik} untuk membangun argumen yang meyakinkan. Ketika sebuah pesan
didukung oleh data, audiens akan melihatnya sebagai sesuatu yang
rasional. Namun, pilar ini juga bisa dimanipulasi. Perhatikan contoh
klasik berikut:

\emph{``Empat dari lima dokter gigi merekomendasikan odol ini.''}

Kalimat ini terdengar logis (Logos), tetapi bisa jadi menyesatkan.
Istilah seperti ini disebut \textbf{``kata-kata musang''} (\emph{weasel
words})---frasa yang menyiratkan sesuatu tanpa menyatakannya secara
gamblang. Mungkin saja perusahaan hanya bertanya kepada lima dokter gigi
yang sudah mereka kenal. Meskipun secara teknis tidak berbohong,
penggunaan logika yang licik seperti ini dapat merusak kredibilitas
(Ethos) pembicara dalam jangka panjang jika audiens merasa tertipu.

\subsubsection{2.4. Ringkasan Pilar
Persuasi}\label{ringkasan-pilar-persuasi}

Tabel berikut merangkum ketiga pilar persuasi untuk memudahkan
pemahaman.

\begin{longtable}[]{@{}
  >{\raggedright\arraybackslash}p{(\linewidth - 4\tabcolsep) * \real{0.3333}}
  >{\raggedright\arraybackslash}p{(\linewidth - 4\tabcolsep) * \real{0.3333}}
  >{\raggedright\arraybackslash}p{(\linewidth - 4\tabcolsep) * \real{0.3333}}@{}}
\toprule\noalign{}
\endhead
\bottomrule\noalign{}
\endlastfoot
Pilar Persuasi & Fokus Utama & Contoh Sederhana dalam Kalimat \\
\textbf{Ethos} & Kredibilitas \& Kepercayaan & ``Sebagai seorang ahli
gizi, saya menyarankan\ldots{}'' \\
\textbf{Pathos} & Emosi \& Perasaan & ``Bayangkan kebahagiaan anak-anak
saat menerima hadiah ini.'' \\
\textbf{Logos} & Logika \& Fakta & ``Berdasarkan data survei, 80\%
pengguna merasa puas.'' \\
\end{longtable}

Selain untuk meyakinkan orang lain, kata-kata juga memiliki kekuatan
luar biasa untuk mempererat ikatan antarmanusia.

\begin{center}\rule{0.5\linewidth}{0.5pt}\end{center}

\subsection{3. Kata-kata yang Membangun Jembatan
Emosional}\label{kata-kata-yang-membangun-jembatan-emosional}

Kata-kata adalah fondasi utama untuk membangun, merawat, dan memperbaiki
hubungan. Melalui bahasa, kita dapat berbagi dunia batin kita dengan
orang lain, menciptakan koneksi yang mendalam dan bermakna.

\subsubsection{3.1. Menyatakan Kasih Sayang dan
Keintiman}\label{menyatakan-kasih-sayang-dan-keintiman}

Mengungkapkan perasaan kasih sayang secara verbal adalah bagian vital
dalam menjaga sebuah hubungan. Ini bukan hanya sekadar perasaan yang
menyenangkan, tetapi juga memiliki dampak nyata. Sebuah studi menemukan
bahwa pasangan yang sering mengkomunikasikan kasih sayang di awal
pernikahan mereka cenderung memiliki hubungan yang jauh lebih langgeng
bertahun-tahun kemudian. Lebih dari itu, mengekspresikan dan menerima
afeksi juga terbukti baik untuk kesehatan fisik dan mental kita.

\subsubsection{3.2. Memberi Penghiburan di Masa
Sulit}\label{memberi-penghiburan-di-masa-sulit}

Di saat seseorang mengalami kesulitan atau kehilangan, kata-kata yang
tepat dapat berfungsi sebagai obat yang menyembuhkan. Terkadang, kita
bingung harus berkata apa, namun kalimat sederhana yang menunjukkan
empati bisa sangat berarti. Mengatakan:

\emph{``Saya ikut sedih kamu mengalami ini.''}

sudah cukup untuk menunjukkan bahwa kita mengakui perasaan mereka dan
menawarkan dukungan. Selain menghibur orang lain, kita juga bisa
menggunakan kata-kata untuk menghibur diri sendiri. Menuliskan
kegundahan dalam buku harian atau blog terbukti dapat menurunkan tingkat
stres dan membantu kita merasa lebih baik.

\subsubsection{3.3. Menciptakan Iklim Komunikasi yang
Positif}\label{menciptakan-iklim-komunikasi-yang-positif}

Cara kita berkomunikasi menentukan ``suasana'' atau iklim dalam sebuah
interaksi. Untuk menciptakan iklim yang positif dan suportif, ada dua
strategi kunci yang bisa kita terapkan:

\begin{itemize}
\item
  \textbf{Gunakan Pesan yang Mengonfirmasi (Confirming Messages)} Pesan
  ini mengakui keberadaan, pikiran, dan perasaan orang lain, bahkan jika
  kita tidak setuju dengan mereka. Ini menunjukkan bahwa kita mendengar
  dan menghargai mereka. Sebaliknya, \textbf{pesan diskonfirmasi}
  membuat orang lain merasa tidak dianggap. Beberapa bentuknya antara
  lain:

  \begin{itemize}
  \item
    \textbf{Respons Kedap:} Diam atau sama sekali tidak memberikan
    respons.
  \item
    \textbf{Respons yang Tidak Relevan:} Mengalihkan pembicaraan ke
    topik yang tidak berhubungan.
  \item
    \textbf{Generalisasi:} Menanggapi keluhan spesifik dengan pernyataan
    umum yang menyapu rata, seperti, ``Kamu selalu mikir tentang kamu
    saja.''
  \item
    \textbf{Respons yang Tidak Personal:} Menjawab curahan hati personal
    dengan kalimat klise yang umum, seperti, ``Ya memang hidup ini penuh
    penderitaan.''
  \end{itemize}
\item
  \textbf{Gunakan ``I-Statement'' (Pernyataan Saya)} Strategi ini adalah
  tentang mengambil kepemilikan atas perasaan kita sendiri, alih-alih
  menyalahkan orang lain. Perhatikan perbedaan antara dua pendekatan
  berikut saat menghadapi teman kos yang lupa mengunci gerbang:
\item
  ``I-Statement'' jauh lebih efektif karena memungkinkan kita untuk
  menyatakan perasaan dan mengatasi masalah tanpa membuat orang lain
  merasa diserang dan bersikap defensif.
\end{itemize}

Namun, kekuatan yang sama yang dapat membangun jembatan juga dapat
digunakan untuk menciptakan jarak dan bahkan merusak.

\begin{center}\rule{0.5\linewidth}{0.5pt}\end{center}

\subsection{4. Pedang Bermata Dua: Penggunaan dan Penyalahgunaan
Bahasa}\label{pedang-bermata-dua-penggunaan-dan-penyalahgunaan-bahasa}

Banyak bentuk bahasa memiliki dua sisi: satu yang dapat menghubungkan,
dan satu lagi yang dapat menyakiti. Memahami kedua sisi ini penting agar
kita bisa menggunakan bahasa secara sadar.

\begin{longtable}[]{@{}
  >{\raggedright\arraybackslash}p{(\linewidth - 4\tabcolsep) * \real{0.3333}}
  >{\raggedright\arraybackslash}p{(\linewidth - 4\tabcolsep) * \real{0.3333}}
  >{\raggedright\arraybackslash}p{(\linewidth - 4\tabcolsep) * \real{0.3333}}@{}}
\toprule\noalign{}
\endhead
\bottomrule\noalign{}
\endlastfoot
Bentuk Bahasa & Penggunaan Positif (Membangun) & Penyalahgunaan Negatif
(Merusak) \\
\textbf{Humor} & Memperkuat keintiman dan membuat suasana rileks saat
tertawa bersama. & Lelucon yang merendahkan, melecehkan, atau
mengorbankan orang lain untuk ditertawakan. \\
\textbf{Eufemisme (Penghalusan)} & Membantu membicarakan topik sensitif
dengan lebih halus (misal: ``merampingkan'' untuk PHK). & Membuat kita
tidak peka terhadap hal-hal buruk (misal: ``kerusakan kolateral'' untuk
korban sipil). \\
\textbf{Bahasa Gaul/Jargon} & Menciptakan identitas dan keakraban dalam
satu kelompok (rekan kerja, teman sebaya). & Mengucilkan atau membuat
orang di luar kelompok merasa tersisih dan tidak paham. \\
\end{longtable}

\subsubsection{4.1. Bahasa yang Secara Terang-terangan
Merusak}\label{bahasa-yang-secara-terang-terangan-merusak}

Selain penyalahgunaan di atas, ada bentuk-bentuk bahasa yang tujuannya
memang secara inheren negatif dan merusak. Dua di antaranya yang paling
berbahaya adalah:

\begin{enumerate}
\def\labelenumi{\arabic{enumi}.}
\item
  \textbf{Pencemaran Nama Baik (Fitnah)} Ini adalah pernyataan yang
  sengaja disebarkan dengan tujuan merusak reputasi seseorang. Ada dua
  bentuk utama: \emph{Slander} (fitnah lisan) dan \emph{Libel} (fitnah
  tertulis). Fitnah tertulis sering dianggap lebih serius karena
  sifatnya yang permanen dan dapat disebarkan lebih luas.
\item
  \textbf{Ujaran Kebencian (Hate Speech)} Ini adalah bentuk ucapan yang
  secara spesifik dimaksudkan untuk merendahkan atau mengintimidasi
  sekelompok orang berdasarkan identitas mereka (seperti suku, agama,
  jenis kelamin, dll.). Ujaran kebencian sangat berbahaya karena dapat
  memicu dan mengobarkan semangat untuk melakukan kekerasan.
\end{enumerate}

Lalu, apa yang membuat kata-kata memiliki bobot dan kekuatan sebesar
ini? Jawabannya terletak pada sifat dasar bahasa itu sendiri.

\begin{center}\rule{0.5\linewidth}{0.5pt}\end{center}

\subsection{5. Di Balik Layar: Mengapa Kata-kata Begitu
Berpengaruh?}\label{di-balik-layar-mengapa-kata-kata-begitu-berpengaruh}

Kekuatan sebuah kata tidak terletak pada susunan huruf atau bunyinya,
melainkan pada makna yang kita---sebagai masyarakat---sepakati bersama
untuk melekat padanya. Makna ini memiliki beberapa lapisan yang
kompleks.

\subsubsection{5.1. Makna Denotatif
vs.~Konotatif}\label{makna-denotatif-vs.-konotatif}

Setiap kata memiliki setidaknya dua lapisan makna:

\begin{itemize}
\item
  \textbf{Makna Denotatif:} Ini adalah makna harfiah, objektif, atau
  ``makna kamus'' dari sebuah kata.
\item
  \textbf{Makna Konotatif:} Ini adalah makna tersirat, subjektif, atau
  emosional yang kita lekatkan pada sebuah kata berdasarkan pengalaman
  dan asosiasi pribadi atau budaya.
\end{itemize}

Mari kita ambil contoh kata \textbf{``kucing''}:

\begin{itemize}
\item
  \textbf{Denotasi:} Seekor hewan mamalia domestik dari famili Felidae.
  (Definisi netral dan literal).
\item
  \textbf{Konotasi:} Bagi seorang pecinta kucing, konotasinya mungkin
  ``lucu, berbulu halus, dan menggemaskan.'' Namun, bagi orang yang
  tidak suka atau alergi, konotasinya bisa jadi ``menjijikkan, bau, dan
  menyebalkan.''
\end{itemize}

\subsubsection{5.2. Bahasa Bermuatan (Loaded
Language)}\label{bahasa-bermuatan-loaded-language}

Ini adalah kata-kata yang makna denotatifnya mungkin netral, tetapi
memiliki muatan emosional (konotatif) yang sangat kuat. Konsep ini
dikenal sebagai \emph{Loaded Language}, atau seperti yang digambarkan
dalam sumbernya, \textbf{``bahasa \emph{loodit}''}---kata-kata yang
memiliki \textbf{``bobot''} emosional yang signifikan. Kata-kata ini
adalah alat yang sangat efektif dalam persuasi (terutama Pathos) karena
kemampuannya membangkitkan emosi yang kuat secara instan.

\begin{itemize}
\item
  \textbf{Contoh Bermuatan Negatif:} \textbf{``kanker''}. Menggambarkan
  sesuatu sebagai ``kanker masyarakat'' langsung menciptakan konotasi
  yang sangat negatif.
\item
  \textbf{Contoh Bermuatan Positif:} \textbf{``keluarga,'' ``damai,''}
  dan \textbf{``merdeka''}. Kata-kata ini cenderung memicu perasaan dan
  asosiasi yang sangat positif di benak kebanyakan orang.
\end{itemize}

\begin{center}\rule{0.5\linewidth}{0.5pt}\end{center}

\subsection{6. Kesimpulan: Gunakan Kekuatanmu dengan
Bijak}\label{kesimpulan-gunakan-kekuatanmu-dengan-bijak}

Kata-kata adalah salah satu alat paling kuat yang kita miliki. Seperti
yang telah kita lihat, bahasa dapat digunakan untuk meyakinkan dan
menggerakkan, untuk membangun hubungan yang dalam dan menyembuhkan luka
emosional. Namun, di tangan yang salah atau dengan niat yang buruk,
kekuatan yang sama bisa digunakan untuk merendahkan, memecah belah, dan
menyakiti.

Setiap kali Anda berbicara atau menulis, sadarilah bahwa Anda sedang
memegang kekuatan tersebut. Pilihlah kata-kata Anda dengan sadar dan
bijaksana. Gunakan kekuatan Anda tidak hanya untuk menyampaikan
informasi, tetapi untuk membangun, menghubungkan, dan menciptakan dampak
positif bagi orang-orang di sekitar Anda.

\section{Sesi 3: MEMORANDUM STRATEGIS: Kerangka Komunikasi Persuasif
Berbasis Prinsip
Aristoteles}\label{sesi-3-memorandum-strategis-kerangka-komunikasi-persuasif-berbasis-prinsip-aristoteles}

\textbf{UNTUK:} Para Profesional

\textbf{DARI:} Pakar Strategi Komunikasi

\textbf{TANGGAL:} 23 September 2025

\textbf{SUBJEK:} Peningkatan Efektivitas Komunikasi Persuasif Melalui
Kerangka Ethos, Pathos, dan Logos

\begin{center}\rule{0.5\linewidth}{0.5pt}\end{center}

\subsection{\texorpdfstring{\textbf{1. Pendahuluan: Memanfaatkan
Kekuatan Persuasi dalam Konteks
Profesional}}{1. Pendahuluan: Memanfaatkan Kekuatan Persuasi dalam Konteks Profesional}}\label{pendahuluan-memanfaatkan-kekuatan-persuasi-dalam-konteks-profesional}

Dalam arena profesional, setiap interaksi adalah sebuah negosiasi
pengaruh. Kemampuan untuk menggerakkan pemikiran dan tindakan secara
etis---bukan melalui paksaan, tetapi melalui persuasi strategis---adalah
kompetensi yang memisahkan manajer dari pemimpin dan ide biasa dari
inovasi yang disruptif. Baik dalam memimpin tim, menegosiasikan
kesepakatan, maupun mempresentasikan ide, kekuatan kata-kata yang kita
pilih akan menentukan keberhasilan kita dalam membangun pengaruh dan
mencapai tujuan.

Untuk menguasai seni ini, kita dapat merujuk pada kerangka retorika yang
dirumuskan oleh filsuf Yunani, Aristoteles, lebih dari 2.000 tahun yang
lalu. Model ini---yang terdiri dari Ethos, Pathos, dan Logos---terbukti
tetap relevan dan sangat efektif untuk membangun pengaruh secara etis
dan berkelanjutan di era modern. Memo strategis ini akan menguraikan
ketiga pilar fundamental tersebut dan menyajikan panduan praktis untuk
mengimplementasikannya dalam komunikasi profesional Anda sehari-hari.

\subsection{\texorpdfstring{\textbf{2. Fondasi Persuasi: Tiga Pilar
Retorika
Aristoteles}}{2. Fondasi Persuasi: Tiga Pilar Retorika Aristoteles}}\label{fondasi-persuasi-tiga-pilar-retorika-aristoteles}

Pemahaman mendalam tentang Ethos, Pathos, dan Logos adalah langkah
fundamental dalam merancang komunikasi yang berdampak. Aristoteles
mengidentifikasi ketiga elemen ini sebagai komponen utama yang
meyakinkan audiens. Keberhasilan persuasi yang sesungguhnya tidak
terletak pada penggunaan salah satu pilar secara dominan, melainkan pada
kemampuan untuk menyeimbangkan ketiganya secara harmonis sesuai dengan
konteks dan audiens yang dihadapi.

\textbf{2.1. Ethos: Membangun Fondasi Kredibilitas dan Kepercayaan}
Ethos merujuk pada persepsi audiens terhadap kredibilitas, integritas,
dan kompetensi pembicara atau lembaga yang diwakilinya. Sederhananya,
audiens akan lebih mudah diyakinkan oleh individu yang mereka hormati
dan percayai. Kredibilitas ini terbentuk ketika audiens merasa bahwa
pembicara tidak memiliki kepentingan egoistis dan justru memikirkan
kepentingan terbaik bagi mereka. Ketika Ethos seorang pembicara tinggi,
audiens secara natural akan memberikan bobot lebih pada pesan yang
disampaikannya.

\textbf{2.2. Pathos: Menggugah Resonansi Emosional} Pathos adalah daya
tarik yang ditujukan pada emosi audiens. Secara strategis, Pathos adalah
alat persuasi yang paling kuat. Mekanismenya bekerja dengan memicu
respons emosional terkait suatu topik, yang pada gilirannya membuat
audiens lebih reseptif dan cenderung menerima klaim yang diajukan.
Komunikasi yang berhasil memanfaatkan Pathos mampu menciptakan hubungan
emosional yang melampaui sekadar logika, mendorong audiens untuk peduli
dan bertindak.

\textbf{2.3. Logos: Menyajikan Argumen yang Logis dan Masuk Akal} Logos
adalah pilar persuasi yang mengandalkan penalaran logis (\emph{akal
budi}), fakta, data, dan statistik untuk membangun argumen yang kokoh.
Pendekatan ini bertujuan untuk meyakinkan audiens melalui penalaran
rasional mereka. Dengan menyajikan bukti yang kuat dan argumen yang
terstruktur, Logos menunjukkan bahwa klaim yang diajukan tidak hanya
beralasan tetapi juga dapat dipertanggungjawabkan secara objektif.

Ketiga pilar ini bekerja secara sinergis untuk menciptakan pesan yang
utuh dan meyakinkan. Namun, fondasi dari semua persuasi yang efektif
adalah Ethos. Tanpa kredibilitas yang kokoh, Pathos akan terasa sebagai
manipulasi murahan, dan Logos akan dianggap sebagai data tanpa konteks
yang dapat diabaikan. Bagian selanjutnya akan fokus pada cara praktis
untuk membangun dan melindungi pilar fundamental ini.

\subsection{\texorpdfstring{\textbf{3. Panduan Aksi: Membangun dan
Menjaga Ethos
(Kredibilitas)}}{3. Panduan Aksi: Membangun dan Menjaga Ethos (Kredibilitas)}}\label{panduan-aksi-membangun-dan-menjaga-ethos-kredibilitas}

Kredibilitas profesional tidak diberikan, melainkan dibangun---dan
dipertahankan---melalui disiplin linguistik yang ketat. Setiap pilihan
kata dapat memperkuat atau mengikis Ethos Anda. Bagian ini berfungsi
sebagai panduan defensif untuk mengidentifikasi dan menetralkan ancaman
linguistik umum terhadap kredibilitas Anda.

\textbf{3.1. Perangkap Umum yang Merusak Kredibilitas}

\begin{itemize}
\item
  \textbf{Klise:} Penggunaan frasa basi seperti ``berpikir \emph{out of
  the box}'' dapat membuat seorang profesional terdengar tidak orisinal.
  \emph{(Dampak: Merusak persepsi \textbf{kompetensi} karena menandakan
  kurangnya pemikiran orisinal.)}
\item
  \textbf{Dialek yang Tidak Sesuai:} Meskipun tidak adil, audiens sering
  membuat penilaian berdasarkan dialek. Penggunaan dialek yang dianggap
  tidak cocok dengan konteks profesional dapat memicu penilaian negatif.
  \emph{(Dampak: Dapat secara tidak sadar menurunkan persepsi
  \textbf{kompetensi} di mata audiens tertentu.)}
\item
  \textbf{Bahasa Ragu-Ragu atau Kabur:} Ketidakjelasan dalam berbahasa
  sering kali diartikan sebagai kurangnya kompetensi, ketidakpercayaan
  diri, atau niat untuk menyembunyikan sesuatu. \emph{(Dampak: Mengikis
  persepsi \textbf{kompetensi} dan \textbf{integritas} secara
  bersamaan.)}
\item
  \textbf{``Kata-kata Musang'' (\emph{Weasel Words}):} Ini adalah
  kata-kata terselubung yang menyiratkan sesuatu yang tidak sepenuhnya
  benar untuk menyesatkan pendengar. Contoh klasiknya adalah ``empat
  dari lima dokter gigi merekomendasikan produk ini,'' yang bisa jadi
  hanya didasarkan pada survei terhadap lima orang dokter.
  \emph{(Dampak: Menghancurkan persepsi \textbf{integritas} dan
  kepercayaan karena menyiratkan niat untuk menyesatkan.)}
\item
  \textbf{Generalisasi Berlebihan:} Pernyataan absolut yang menggunakan
  kata ``semua'' atau ``selalu'' (misalnya, ``semua pemerintah bobrok'')
  secara instan mengurangi kredibilitas karena klaim tersebut hampir
  mustahil dibuktikan. \emph{(Dampak: Merusak persepsi
  \textbf{integritas} karena menunjukkan kurangnya analisis yang cermat
  dan kejujuran intelektual.)}
\end{itemize}

Menghindari jebakan-jebakan ini adalah langkah defensif yang krusial
untuk melindungi kredibilitas Anda. Selanjutnya, kita akan beralih ke
strategi proaktif untuk menciptakan lingkungan komunikasi yang mendukung
diterimanya pesan persuasif.

\subsection{\texorpdfstring{\textbf{4. Strategi Lanjutan: Menciptakan
Iklim Komunikasi yang
Reseptif}}{4. Strategi Lanjutan: Menciptakan Iklim Komunikasi yang Reseptif}}\label{strategi-lanjutan-menciptakan-iklim-komunikasi-yang-reseptif}

Sebelum argumen logis (Logos) dapat diterima, Anda harus terlebih dahulu
menciptakan landasan psikologis yang reseptif. Sebuah argumen yang
paling brilian sekalipun akan gagal jika disampaikan dalam iklim yang
defensif atau negatif. Strategi berikut adalah teknik untuk `priming'
lingkungan komunikasi, secara proaktif menurunkan resistensi audiens dan
membuat mereka lebih terbuka terhadap pengaruh.

\textbf{4.1. Prioritaskan Pesan yang Mengkonfirmasi} Pesan yang
mengkonfirmasi adalah tindakan mengakui eksistensi, pemikiran, dan
perasaan orang lain---bahkan jika kita tidak setuju dengan mereka. Ini
menunjukkan bahwa kita mendengar dan menghargai sudut pandang mereka.
Hal ini sangat kontras dengan ``pesan yang diskonfirmasi,'' seperti
mengabaikan lawan bicara, memberikan respons tidak relevan, atau
menggeneralisasi keluhan mereka (misalnya, ``ah kamu cuma mikirin diri
sendiri, selalu mikir tentang kamu saja begitu''). Dengan secara aktif
memberikan konfirmasi, kita mengurangi resistensi dan membuka pintu
untuk dialog yang produktif.

\textbf{4.2. Ambil Kepemilikan Pesan dengan ``I-Statement''} Prinsip
dasarnya adalah mengambil kepemilikan penuh (\emph{ownership}) atas
perasaan kita. Komunikator yang efektif memahami bahwa perasaan mereka
adalah milik mereka sendiri, bukan sesuatu yang `disebabkan' oleh orang
lain. Kegagalan untuk memahami ini mengarah pada penggunaan
``\emph{You-statement}'' (pernyataan kamu) yang bersifat menuduh dan
memicu sikap defensif (misalnya, ``Kamu membuat saya marah karena tidak
mengunci pintu''). Sebaliknya, ``\emph{I-statement}'' (pernyataan saya)
berfokus pada perasaan kita sendiri tanpa menyalahkan. Contoh yang lebih
konstruktif adalah, ``Saya merasa kesal karena menemukan pintu tidak
terkunci. Lingkungan ini tidak aman, dan saya khawatir kita bisa
kecurian.'' Pendekatan ini mengatasi masalah sambil menjaga martabat
lawan bicara dan mencegah eskalasi konflik.

\textbf{4.3. Bedakan antara Klaim Faktual dan Opini} Konflik yang tidak
perlu sering kali muncul karena kegagalan membedakan antara fakta dan
opini. Klaim faktual adalah pernyataan yang dapat diverifikasi sebagai
benar atau salah. Sementara itu, opini adalah penilaian pribadi yang
hanya bisa disetujui atau tidak disetujui. Memperlakukan opini
seolah-olah itu adalah fakta (misalnya, dengan mengatakan, ``Pendapat
kamu itu salah'') adalah tindakan yang tidak produktif dan memicu
pertengkaran yang sia-sia. Cara yang lebih efektif adalah mengakui hak
orang lain untuk memiliki pendapatnya, bahkan jika kita tidak setuju,
dan kemudian menjelaskan perspektif kita dengan cara yang terhormat.

Dengan menerapkan strategi-strategi ini, kita menciptakan landasan
psikologis yang subur di mana pesan persuasif kita lebih mungkin
didengar, dipertimbangkan, dan diterima.

\subsection{\texorpdfstring{\textbf{5. Kesimpulan dan
Rekomendasi}}{5. Kesimpulan dan Rekomendasi}}\label{kesimpulan-dan-rekomendasi}

Persuasi yang efektif dan berkelanjutan bukanlah hasil dari satu taktik
tunggal, melainkan perpaduan strategis antara kredibilitas personal
(Ethos), resonansi emosional (Pathos), dan argumen yang kuat (Logos).
Ketiga pilar ini, ketika disampaikan dalam iklim komunikasi yang
didasari oleh rasa saling menghargai, akan secara signifikan
meningkatkan kemampuan kita untuk mempengaruhi dan memimpin secara
positif, serta membuka kekuatan bahasa yang lebih luas untuk menghibur,
menginspirasi, dan membangun hubungan yang mendalam.

Untuk mengimplementasikan prinsip-prinsip ini secara langsung, berikut
adalah beberapa langkah tindakan yang direkomendasikan:

\begin{itemize}
\item
  \textbf{Lakukan Audit Kredibilitas Diri:} Secara sadar, analisis
  pilihan bahasa Anda dalam komunikasi sehari-hari. Identifikasi dan
  kurangi kebiasaan yang mungkin merusak Ethos Anda, seperti penggunaan
  klise, bahasa yang kabur, atau generalisasi yang berlebihan.
\item
  \textbf{Latih Penggunaan ``I-Statement'':} Mulailah secara aktif
  mengganti ``\emph{You-statement}'' yang bersifat menuduh dengan
  ``\emph{I-statement}'' dalam interaksi Anda. Praktik ini akan membantu
  mengurangi konflik, mencegah sikap defensif dari lawan bicara, dan
  membangun dialog yang lebih konstruktif.
\item
  \textbf{Fokus pada Keseimbangan Tiga Pilar:} Dalam setiap upaya
  persuasi---baik itu email, rapat, atau presentasi---rencanakan secara
  sadar bagaimana Anda akan membangun Ethos, membangkitkan Pathos yang
  relevan, dan mendukung argumen Anda dengan Logos yang solid.
  Keseimbangan ini adalah kunci untuk komunikasi yang benar-benar
  meyakinkan.
\end{itemize}

Menguasai kerangka ini bukan sekadar tentang memenangkan argumen,
melainkan tentang membangun kepemimpinan yang berpengaruh dan
berkelanjutan. Gunakan kekuatan ini dengan bijaksana.

\section{SESI 4: Panduan Praktik Terbaik: Membangun Iklim Komunikasi
yang Positif dan Suportif di Tempat
Kerja}\label{sesi-4-panduan-praktik-terbaik-membangun-iklim-komunikasi-yang-positif-dan-suportif-di-tempat-kerja}

\subsection{1. Pendahuluan: Mengapa Iklim Komunikasi Adalah Aset
Strategis
Manajer}\label{pendahuluan-mengapa-iklim-komunikasi-adalah-aset-strategis-manajer}

Sebagai seorang manajer, alat paling strategis yang Anda miliki bukanlah
perangkat lunak manajemen proyek atau analisis finansial, melainkan
pilihan kata-kata Anda. Cara Anda berkomunikasi setiap hari---dalam
rapat, email, atau percakapan informal---secara langsung membentuk
\emph{iklim komunikasi} di dalam tim. Penggunaan bahasa yang cermat
bukanlah sekadar ``nice-to-have'' atau soal kesopanan; ini adalah alat
fundamental untuk membangun kepercayaan, mengubah potensi konflik
menjadi peluang perbaikan, dan mendorong kinerja serta inovasi tim.

Tujuan dari panduan ini adalah untuk membekali Anda dengan teknik-teknik
konkret yang dapat segera diterapkan, berdasarkan prinsip-prinsip inti
komunikasi interpersonal. Kita akan membahas cara menggunakan pesan yang
membangun (mengkonfirmasi), menghindari pemicu reaksi defensif, dan
mengelola dialog yang sulit secara konstruktif. Dengan menguasai
keterampilan ini, Anda dapat secara proaktif menciptakan lingkungan di
mana setiap anggota tim merasa dihargai, aman untuk menyuarakan ide, dan
termotivasi untuk berkontribusi secara maksimal.

Langkah pertama dalam perjalanan ini adalah memahami perbedaan krusial
antara pesan yang membangun fondasi kepercayaan dan pesan yang tanpa
disadari justru meruntuhkannya. Mari kita mulai dengan membedah konsep
inti: pesan yang mengkonfirmasi versus pesan yang men-diskonfirmasi.

\subsection{2. Fondasi Komunikasi Suportif: Kekuatan Pesan yang
Mengkonfirmasi}\label{fondasi-komunikasi-suportif-kekuatan-pesan-yang-mengkonfirmasi}

Landasan dari setiap interaksi profesional yang sehat adalah penggunaan
``pesan yang mengkonfirmasi''. Mengkonfirmasi seseorang bukan berarti
Anda harus selalu setuju dengan mereka. Sebaliknya, ini adalah tindakan
mendasar untuk mengakui keberadaan, perasaan, dan pemikiran mereka
sebagai sesuatu yang valid dan layak didengar. Ketika anggota tim merasa
keberadaan dan perspektif mereka diakui, mereka akan merasa dihargai,
aman secara psikologis, dan lebih terbuka untuk berkolaborasi. Ini
adalah langkah pertama dan paling esensial dalam membangun iklim yang
suportif.

Secara definitif, kedua jenis pesan ini dapat dibedakan sebagai berikut:

\begin{itemize}
\item
  \textbf{Pesan yang Mengkonfirmasi:} Pesan yang secara eksplisit
  mengakui eksistensi, perasaan, dan pemikiran orang lain. Anda
  menunjukkan bahwa Anda mendengar dan memahami sudut pandang mereka,
  bahkan jika Anda tidak sepakat.
\item
  \textbf{Pesan yang Men-diskonfirmasi:} Pesan atau respons yang
  mengabaikan, meniadakan, atau meremehkan orang lain. Respons ini
  secara efektif membuat lawan bicara merasa ``tidak ada'' atau tidak
  penting.
\end{itemize}

Perbedaan dampaknya sangat signifikan. Tabel berikut mengilustrasikan
respons diskonfirmasi yang umum terjadi dan bagaimana Anda dapat
mengubahnya menjadi interaksi yang mengkonfirmasi.

\begin{longtable}[]{@{}
  >{\raggedright\arraybackslash}p{(\linewidth - 2\tabcolsep) * \real{0.5000}}
  >{\raggedright\arraybackslash}p{(\linewidth - 2\tabcolsep) * \real{0.5000}}@{}}
\toprule\noalign{}
\endhead
\bottomrule\noalign{}
\endlastfoot
Respons yang Men-diskonfirmasi (Hindari) & Dampak Negatif dan Alternatif
Konfirmasi (Praktikkan) \\
\textbf{Respons Kedap/Diam:} Sama sekali tidak memberikan respons saat
seseorang berbicara atau mengajukan pertanyaan. & \textbf{Dampak:} Ini
adalah bentuk pengabaian paling ekstrem, yang secara efektif
``meniadakan'' eksistensi orang
tersebut.\textless br\textgreater\textless br\textgreater{}\textbf{Alternatif:}
Berikan pengakuan singkat, bahkan jika Anda sibuk. \emph{``Terima kasih
sudah menyampaikan ini. Saya sedang fokus pada hal lain sekarang,
bisakah kita diskusikan ini dalam 15 menit?''} \\
\textbf{Mengabaikan:} Secara sengaja mengalihkan pembicaraan atau
mengabaikan kontribusi seseorang dalam diskusi. & \textbf{Dampak:}
Membuat individu merasa kontribusinya tidak berharga dan enggan untuk
berbicara di kemudian
hari.\textless br\textgreater\textless br\textgreater{}\textbf{Alternatif:}
Akui kontribusi mereka sebelum melanjutkan. \emph{``Itu poin yang
menarik, Budi. Mari kita catat dulu dan kembali ke topik utama. Saya
ingin mendengarnya lebih lanjut nanti.''} \\
\textbf{Generalisasi Keluhan:} Menggunakan kata-kata absolut seperti
``kamu \emph{selalu}\ldots{}'' atau ``kamu \emph{hanya} memikirkan
dirimu sendiri.'' & \textbf{Dampak:} Serangan ini meniadakan identitas
individu dan memicu defensif, bukan penyelesaian
masalah.\textless br\textgreater\textless br\textgreater{}\textbf{Alternatif:}
Fokus pada perilaku spesifik dan dampaknya. Ganti ``Kamu
selalu\ldots{}'' dengan ``Ketika X terjadi, dampaknya adalah Y. Mari
kita diskusikan bagaimana kita bisa mengatasinya.'' \\
\textbf{Respons Tidak Relevan:} Menanggapi keluhan atau ide dengan topik
yang sama sekali tidak berhubungan. & \textbf{Dampak:} Mengirimkan
sinyal bahwa Anda tidak mendengarkan atau tidak peduli dengan apa yang
mereka
katakan.\textless br\textgreater\textless br\textgreater{}\textbf{Alternatif:}
Tetap pada topik. Jika Anda harus mengubahnya, lakukan transisi yang
jelas. \emph{``Saya paham kekhawatiranmu tentang A. Sebelum kita bahas
solusinya, ada informasi penting tentang B yang perlu saya
sampaikan.''} \\
\textbf{Respons Impersonal:} Menjawab keluhan pribadi dengan klise atau
generalisasi, seperti \emph{``Ya memang hidup ini penuh penderitaan.''}
& \textbf{Dampak:} Menolak untuk mengakui pengalaman unik individu dan
membuat mereka merasa
sendirian.\textless br\textgreater\textless br\textgreater{}\textbf{Alternatif:}
Tunjukkan empati dan akui perasaan mereka. \textbf{Tip Manajer:} Hindari
klise. Jika Anda tidak tahu harus berkata apa, mengakui perasaan mereka
(``Itu terdengar sangat sulit'') jauh lebih kuat daripada menawarkan
solusi atau generalisasi yang dangkal. \\
\textbf{Serangan Verbal Langsung:} Menggunakan kata-kata yang secara
sengaja menyakiti, meremehkan, atau menghina. & \textbf{Dampak:}
Menghancurkan rasa aman psikologis secara total, membunuh kepercayaan,
dan menciptakan lingkungan kerja yang
toksik.\textless br\textgreater\textless br\textgreater{}\textbf{Alternatif:}
Pisahkan antara isu dan individu. Alih-alih menyerang orangnya, gunakan
prinsip ``Pernyataan Saya'' (dibahas di Bab 4) untuk mengatasi masalah
tanpa merendahkan martabat mereka. \\
\end{longtable}

Setelah Anda berhasil membangun fondasi dengan mengakui eksistensi dan
pemikiran seseorang, langkah selanjutnya adalah memastikan cara Anda
berkomunikasi tidak memicu perlawanan atau pembelaan diri, terutama saat
menyampaikan umpan balik atau membahas masalah.

\subsection{3. Mencegah Defensif: Teknik Komunikasi
Non-Evaluatif}\label{mencegah-defensif-teknik-komunikasi-non-evaluatif}

Reaksi defensif adalah respons alami manusia ketika merasa dihakimi,
diserang, atau dikendalikan. Di tempat kerja, sikap defensif dapat
mematikan dialog yang produktif, mengubah sesi umpan balik menjadi ajang
perdebatan, dan merusak hubungan kerja. Tujuan Anda bukanlah memenangkan
argumen, melainkan menjaga agar dialog tetap terbuka, kolaboratif, dan
fokus pada solusi. Berikut adalah strategi proaktif untuk berkomunikasi
dengan cara yang meminimalkan defensif.

\begin{itemize}
\item
  \textbf{Praktikkan Komunikasi Deskriptif, Hindari Evaluasi Personal}
  Daripada memberi label atau menghakimi seseorang (\emph{evaluatif}),
  fokuslah untuk mendeskripsikan perilaku atau situasi secara objektif
  (\emph{deskriptif}). Evaluasi bersifat personal dan sering kali memicu
  perlawanan.

  \begin{itemize}
  \item
    \emph{Evaluatif (Hindari):} ``Laporanmu berantakan.''
  \item
    \emph{Deskriptif (Gunakan):} ``Saya melihat ada beberapa bagian
    dalam laporan ini yang formatnya belum konsisten. Mari kita pastikan
    semuanya rapi sebelum dikirim ke klien.''
  \end{itemize}
\item
  \textbf{Fokus pada Solusi Kolaboratif, Bukan Pengendalian Sepihak}
  Ketika seorang manajer mencoba memaksakan kehendak atau solusi
  tunggal, anggota tim cenderung akan menolak. Pendekatan yang lebih
  efektif adalah mengundang kolaborasi dalam mencari solusi, yang
  meningkatkan kepemilikan dan komitmen.

  \begin{itemize}
  \item
    \emph{Pengendalian (Hindari):} ``Kamu harus menyelesaikan ini dengan
    cara X sekarang juga.''
  \item
    \emph{Orientasi Masalah (Gunakan):} ``Kita menghadapi tantangan X.
    Menurutmu, apa pendekatan terbaik untuk menyelesaikannya?''
  \end{itemize}
\item
  \textbf{Bangun Kepercayaan Melalui Kepedulian dan Kesetaraan} Hindari
  sikap superior atau serba tahu. Tunjukkan bahwa Anda menghargai
  perspektif mereka dan terbuka terhadap ide-ide baru. Mengkomunikasikan
  bahwa Anda peduli pada mereka sebagai individu, bukan hanya sebagai
  unit produktivitas, akan membangun kepercayaan yang mendalam.

  \begin{itemize}
  \item
    \emph{Superior (Hindari):} ``Saya sudah lebih lama di sini, jadi
    dengarkan saja cara saya.''
  \item
    \emph{Kesetaraan (Gunakan):} ``Saya punya beberapa ide berdasarkan
    pengalaman saya, tapi saya ingin mendengar perspektifmu juga.
    Mungkin ada hal yang saya lewatkan.''
  \end{itemize}
\item
  \textbf{Gunakan Pertanyaan untuk Memahami, Bukan untuk Menghakimi}
  Sebelum memberikan umpan balik atau penilaian, ajukan pertanyaan
  terlebih dahulu untuk memahami konteks dan perspektif lawan bicara
  Anda. Ini menunjukkan bahwa Anda berusaha memahami, bukan hanya
  menghakimi.

  \begin{itemize}
  \item
    \emph{Menilai (Hindari):} ``Kamu terlambat menyerahkan tugas ini.''
  \item
    \emph{Bertanya (Gunakan):} ``Saya perhatikan tugas ini melewati
    tenggat waktu. Apakah ada kendala yang kamu hadapi?''
  \end{itemize}
\end{itemize}

\subsubsection{\texorpdfstring{\textbf{Cara Memberikan Umpan Balik yang
Konstruktif}}{Cara Memberikan Umpan Balik yang Konstruktif}}\label{cara-memberikan-umpan-balik-yang-konstruktif}

Bahkan dengan teknik di atas, ada kalanya umpan balik eksplisit
diperlukan. Sebelum memberikannya, gunakan proses dua langkah strategis
berikut:

\begin{enumerate}
\def\labelenumi{\arabic{enumi}.}
\item
  \textbf{Lakukan Konfirmasi Terlebih Dahulu:} Tanyakan pada diri
  sendiri: \emph{Apakah orang ini benar-benar memerlukan umpan balik,
  atau dia hanya perlu didengarkan?} Menawarkan telinga adalah tindakan
  \textbf{konfirmasi} yang kuat. Terkadang, yang dibutuhkan seseorang
  hanyalah ruang untuk meluapkan isi hatinya tanpa nasihat, yang akan
  memperkuat hubungan Anda.
\item
  \textbf{Pilih Jenis Umpan Balik yang Tepat:} Jika umpan balik memang
  diperlukan, pilih strategi Anda:

  \begin{itemize}
  \item
    \textbf{Umpan Balik Nonevaluatif:} Jika tujuannya mendukung dan
    memastikan pemahaman, gunakan pendekatan ini. Ajukan pertanyaan
    klarifikasi dan parafrasakan apa yang mereka katakan (\emph{``Jadi,
    jika saya memahaminya dengan benar, Anda merasa\ldots{}''}) untuk
    menunjukkan bahwa Anda benar-benar mendengarkan.
  \item
    \textbf{Umpan Balik Evaluatif:} Jika perbaikan konkret diperlukan,
    gunakan struktur ini: \emph{selalu mulai dengan sesuatu yang baik
    dan tulus}, baru kemudian sampaikan area yang perlu diperbaiki. Ini
    membantu lawan bicara tetap terbuka untuk menerima masukan.
  \end{itemize}
\end{enumerate}

Dari pencegahan defensif secara umum, mari kita beralih ke alat spesifik
yang paling ampuh untuk mengelola konflik dan kepemilikan emosi:
\textbf{``Pernyataan Saya''}.

\subsection{4. Mengelola Konflik dengan `Pernyataan Saya'
(I-Statements)}\label{mengelola-konflik-dengan-pernyataan-saya-i-statements}

Seorang komunikator yang matang memahami pentingnya ``memiliki''
(\emph{ownership}) pikiran dan perasaan mereka sendiri. Kesalahan umum
dalam konflik adalah berpikir bahwa orang lain \emph{menyebabkan}
perasaan kita. \textbf{`Pernyataan Saya' (I-Statement)} adalah alat
linguistik yang memungkinkan kita mengklaim kepemilikan tersebut,
mengubah potensi konflik menjadi peluang untuk perbaikan proses dan
keselarasan tim. Ini adalah alat pamungkas untuk mempraktikkan
komunikasi \emph{deskriptif}, \emph{berorientasi masalah}, dan penuh
\emph{kepedulian} yang telah kita bahas.

Perbedaannya sangat mendasar:

\begin{itemize}
\item
  \textbf{`You-Statement'} (misalnya, ``Kamu membuat saya marah'')
  cenderung menyalahkan, memicu defensif, dan mengeskalasi konflik.
\item
  \textbf{`I-Statement'} berfokus pada perasaan pembicara dan dampaknya,
  memungkinkan dialog yang jujur tanpa menyerang orang lain.
\end{itemize}

Mari adaptasi sebuah studi kasus ke dalam konteks profesional. Bayangkan
seorang rekan kerja sering lupa mengunci pintu ruang server yang berisi
data sensitif.

\begin{itemize}
\item
  \textbf{Pendekatan yang Salah (You-Statement):}
\item
  Pendekatan ini menyerang karakter (``tidak bertanggung jawab''), yang
  hampir pasti akan memicu pertengkaran, bukan solusi.
\item
  \textbf{Pendekatan yang Benar (I-Statement):}
\end{itemize}

\subsubsection{\texorpdfstring{\textbf{Formula `I-Statement' yang
Efektif}}{Formula `I-Statement' yang Efektif}}\label{formula-i-statement-yang-efektif}

Gunakan formula empat langkah ini untuk menyusun pesan Anda secara
konstruktif:

\begin{enumerate}
\def\labelenumi{\arabic{enumi}.}
\item
  \textbf{Emosi Saya:} Mulailah dengan menyatakan perasaan Anda.

  \begin{itemize}
  \tightlist
  \item
    \emph{``\textbf{Saya merasa}\ldots{}
    {[}khawatir/frustrasi/bingung{]}''}
  \end{itemize}
\item
  \textbf{Perilaku Spesifik:} Deskripsikan situasi atau perilaku
  objektif yang memicu emosi tersebut. Ini kuncinya: fokus pada
  \textbf{situasi} (``pintu tidak terkunci'') bukan \textbf{tindakan
  orang lain} (``kamu tidak mengunci''), yang membuatnya terasa jauh
  lebih tidak menuduh.

  \begin{itemize}
  \tightlist
  \item
    \emph{``\textbf{ketika saya melihat/menemukan}\ldots{} {[}deskripsi
    objektif dari perilaku: laporan belum selesai, data tidak
    konsisten{]}''}
  \end{itemize}
\item
  \textbf{Dampak Konkret:} Jelaskan mengapa Anda merasakan hal itu
  dengan menguraikan dampak nyata dari perilaku tersebut.

  \begin{itemize}
  \tightlist
  \item
    \emph{``\textbf{karena ini berdampak pada}\ldots{} {[}keamanan
    data/timeline proyek/kualitas kerja{]}''}
  \end{itemize}
\item
  \textbf{Permintaan Solutif:} Nyatakan permintaan Anda dengan jelas dan
  positif, berfokus pada solusi di masa depan.

  \begin{itemize}
  \tightlist
  \item
    \emph{``\textbf{Saya akan sangat menghargai jika kita bisa}\ldots{}
    {[}permintaan yang jelas dan positif{]}''}
  \end{itemize}
\end{enumerate}

Dengan menguasai `I-Statement', Anda memindahkan fokus dari menyalahkan
ke penyelesaian masalah kolaboratif. Ini adalah jembatan dari kejelasan
emosional ke kejelasan faktual, yang membawa kita pada keterampilan
penting berikutnya.

\subsection{5. Membedakan Fakta dan Opini untuk Dialog yang
Konstruktif}\label{membedakan-fakta-dan-opini-untuk-dialog-yang-konstruktif}

Banyak konflik di tempat kerja berasal dari kegagalan membedakan antara
klaim faktual yang dapat diverifikasi dan penilaian pribadi (opini).
Menguasai perbedaan ini adalah keterampilan penting untuk menjaga
diskusi tetap rasional, berbasis data, dan terhindar dari perdebatan
yang tidak produktif.

\begin{itemize}
\item
  \textbf{Klaim Faktual:} Sebuah pernyataan yang dapat diverifikasi
  sebagai \emph{benar} atau \emph{salah} dengan bukti. Contoh: ``Laporan
  penjualan kuartal lalu menunjukkan penurunan 5\%.''
\item
  \textbf{Opini:} Sebuah penilaian atau keyakinan pribadi. Respons yang
  tepat adalah \emph{setuju} atau \emph{tidak setuju}. Contoh: ``Menurut
  saya, strategi pemasaran kita saat ini kurang efektif.''
\end{itemize}

Pegang teguh prinsip kunci berikut saat berhadapan dengan opini:

\emph{``Menyatakan bahwa opini seseorang `salah' sama sekali tidak
berguna dan hanya akan menimbulkan pertengkaran. Opini tidak bisa salah;
kita hanya bisa setuju atau tidak setuju dengannya.''}

Alih-alih mencoba ``memperbaiki'' opini orang lain, gunakan strategi
tiga langkah yang lebih profesional dan konstruktif saat Anda tidak
setuju:

\begin{enumerate}
\def\labelenumi{\arabic{enumi}.}
\item
  \textbf{Pahami:} Berusahalah untuk benar-benar memahami \emph{mengapa}
  orang tersebut memiliki pandangan tersebut.

  \begin{itemize}
  \tightlist
  \item
    \textbf{Gunakan Skrip ini:} \emph{``Bantu saya memahami lebih jauh,
    apa yang mendasari pandangan Anda?''} atau \emph{``Itu perspektif
    yang menarik. Bisakah Anda jelaskan lebih lanjut?''}
  \end{itemize}
\item
  \textbf{Akui:} Akui hak mereka untuk memiliki pendapat tersebut,
  bahkan jika Anda tidak setuju. Ini menunjukkan rasa hormat.

  \begin{itemize}
  \tightlist
  \item
    \textbf{Gunakan Skrip ini:} \emph{``Terima kasih sudah berbagi. Saya
    bisa melihat alur berpikir Anda untuk sampai ke sana.''}
  \end{itemize}
\item
  \textbf{Nyatakan:} Sampaikan dengan hormat bahwa Anda memiliki
  pandangan yang berbeda, tanpa menyiratkan bahwa pandangan mereka tidak
  valid.

  \begin{itemize}
  \tightlist
  \item
    \textbf{Gunakan Skrip ini:} \emph{``Saya menghargai pandangan itu.
    Dari sisi saya, saya melihatnya sedikit berbeda, di mana\ldots{}''}
  \end{itemize}
\end{enumerate}

Fokusnya adalah menciptakan dialog di mana berbagai perspektif dapat
didiskusikan secara terbuka untuk mencapai keputusan terbaik, bukan
memenangkan perdebatan.

\subsection{6. Kesimpulan: Komitmen Manajer terhadap Komunikasi
Sadar}\label{kesimpulan-komitmen-manajer-terhadap-komunikasi-sadar}

Membangun iklim komunikasi yang suportif bukanlah hasil dari satu kali
pelatihan, melainkan komitmen sadar seorang manajer untuk mempraktikkan
keunggulan komunikasi setiap hari. Ini bukan lagi \emph{soft skill},
melainkan kompetensi kepemimpinan inti yang mendorong keunggulan
kompetitif. Panduan ini telah menguraikan pilar-pilar utamanya:
\textbf{pesan yang mengkonfirmasi} untuk membuat setiap orang merasa
dihargai, \textbf{teknik non-defensif} untuk menjaga dialog tetap
terbuka, \textbf{`I-Statements'} untuk mengelola konflik secara dewasa,
dan kemampuan untuk \textbf{membedakan fakta dari opini} untuk menjaga
diskusi tetap produktif.

Sebagai seorang pemimpin, setiap kata yang Anda ucapkan memiliki
kekuatan untuk membangun atau merusak, untuk menginspirasi atau
mematahkan semangat. Praktikkan prinsip-prinsip ini dengan sengaja dalam
setiap interaksi. Proses ini adalah sebuah perjalanan berkelanjutan,
namun imbalannya sangat besar: sebuah tim yang lebih kohesif, inovatif,
dan tangguh. Pada akhirnya, warisan kepemimpinan Anda tidak diukur dari
proyek yang selesai, tetapi dari kualitas percakapan yang Anda pimpin di
sepanjang jalan.

\section{Pekerjaan Rumah: Kuis 4}\label{pekerjaan-rumah-kuis-4}

Panduan ini dirancang untuk meninjau dan memperdalam pemahaman mengenai
kekuatan, sifat, dan penggunaan bahasa dalam komunikasi interpersonal
berdasarkan konteks yang disediakan. Dokumen ini mencakup kuis singkat,
soal esai untuk refleksi lebih lanjut, dan glosarium istilah-istilah
kunci.

\subsection{Kuis Pemahaman}\label{kuis-pemahaman}

Jawablah setiap pertanyaan berikut dalam 2-3 kalimat, berdasarkan
informasi yang terdapat dalam materi sumber.

\begin{enumerate}
\def\labelenumi{\arabic{enumi}.}
\item
  Mengapa nama dianggap sebagai perangkat linguistik yang sangat kuat?
\item
  Jelaskan secara singkat tiga pilar persuasi menurut Aristoteles: etos,
  patos, dan logos.
\item
  Sebutkan dan jelaskan dua jenis bahasa yang dapat merusak kredibilitas
  seseorang saat berbicara.
\item
  Apa perbedaan utama antara pesan yang mengkonfirmasi (confirming
  messages) dan pesan yang diskonfirmasi (disconfirming messages)?
\item
  Jelaskan perbedaan mendasar antara \emph{I-statement} (pernyataan
  saya) dan \emph{you-statement} (pernyataan kamu) dalam mengambil
  kepemilikan atas perasaan.
\item
  Bagaimana cara membedakan antara klaim faktual dan opini, serta
  bagaimana seharusnya kita menanggapi masing-masing?
\item
  Apa itu efemisme dan berikan contoh penggunaannya? Apa potensi dampak
  negatif dari penggunaan efemisme?
\item
  Jelaskan perbedaan antara arti denotatif dan arti konotatif dari
  sebuah kata, gunakan contoh kata ``rumah''.
\item
  Apa yang dimaksud dengan \emph{loaded language} (bahasa yang
  berbobot/dimuati)?
\item
  Jelaskan secara singkat dua prinsip dalam hipotesis Sapir-Whorf.
\end{enumerate}

\begin{center}\rule{0.5\linewidth}{0.5pt}\end{center}

\subsection{Soal Esai}\label{soal-esai}

\begin{enumerate}
\def\labelenumi{\arabic{enumi}.}
\item
  Aristoteles mengidentifikasi etos, patos, dan logos sebagai pilar
  persuasi. Analisislah bagaimana seorang pembicara dapat menggunakan
  bahasa untuk membangun ketiga pilar ini secara efektif. Sebaliknya,
  jelaskan bagaimana pilihan kata yang salah (seperti ujaran kebencian,
  fitnah, atau klise) dapat menghancurkan ketiganya.
\item
  Jelaskan konsep ``pesan yang mengkonfirmasi'' dan ``pesan yang
  diskonfirmasi''. Berikan skenario percakapan imajiner di mana satu
  orang menggunakan pesan diskonfirmasi, dan tulis ulang skenario
  tersebut untuk menunjukkan bagaimana penggunaan pesan konfirmasi dapat
  menciptakan iklim komunikasi yang lebih positif dan kondusif.
\item
  Konsep \emph{I-statement} dan \emph{you-statement} menekankan
  pentingnya kepemilikan atas perasaan. Mengapa mengambil kepemilikan
  atas perasaan kita sendiri sangat penting dalam komunikasi
  interpersonal? Hubungkan ide ini dengan upaya untuk menghindari
  respons defensif dari lawan bicara.
\item
  Bahasa memiliki lapisan makna (denotatif dan konotatif) serta
  bervariasi dalam tingkat abstraksi. Diskusikan bagaimana
  kesalahpahaman dalam komunikasi dapat timbul dari perbedaan
  interpretasi makna konotatif dan penggunaan kata-kata yang terlalu
  abstrak. Berikan contoh spesifik untuk mendukung argumen Anda.
\item
  Dengan mempertimbangkan pedoman komunikasi digital yang diuraikan
  dalam materi, analisislah tantangan unik yang muncul saat mencoba
  menyampaikan kasih sayang, memberikan hiburan (\emph{comfort}), atau
  menyelesaikan konflik melalui platform seperti WhatsApp atau email.
  Bagaimana ``kerampingan'' komunikasi digital memengaruhi proses ini?
\end{enumerate}

\begin{center}\rule{0.5\linewidth}{0.5pt}\end{center}

\subsection{Glosarium Istilah Kunci}\label{glosarium-istilah-kunci-3}

\begin{longtable}[]{@{}
  >{\raggedright\arraybackslash}p{(\linewidth - 2\tabcolsep) * \real{0.5000}}
  >{\raggedright\arraybackslash}p{(\linewidth - 2\tabcolsep) * \real{0.5000}}@{}}
\toprule\noalign{}
\endhead
\bottomrule\noalign{}
\endlastfoot
Istilah & Definisi \\
\textbf{Arbitrer} & Sifat bahasa di mana makna sebuah kata hanya
didasarkan pada kesepakatan bersama para penggunanya, bukan karena
hubungan inheren antara kata dan objek yang diwakilinya. \\
\textbf{Aturan Fonologis} & Aturan yang berkaitan dengan bagaimana
kata-kata diucapkan (\emph{pronunciation}) dalam suatu bahasa agar dapat
dipahami secara seragam. \\
\textbf{Aturan Pragmatis} & Aturan yang mengatur bagaimana penggunaan
kata-kata bergantung pada konteks sosial dan budaya. Makna ditentukan
oleh situasi praktis saat bahasa digunakan. \\
\textbf{Aturan Semantik} & Aturan yang berkaitan dengan makna dari
setiap kata. Ini merujuk pada definisi kata-kata dalam sebuah
kalimat. \\
\textbf{Aturan Sintaksis} & Aturan yang mengatur cara menggabungkan
kata-kata untuk membentuk kalimat yang benar secara tata bahasa. \\
\textbf{Determinisme Linguistik} & Bagian dari hipotesis Sapir-Whorf
yang menyatakan bahwa struktur bahasa menentukan cara kita berpikir dan
memandang dunia. \\
\textbf{Dialek} & Variasi dari satu bahasa yang didasarkan pada
perbedaan regional atau sosial, yang dapat memengaruhi persepsi orang
terhadap pembicara. \\
\textbf{Efemisme} & Ekspresi yang tidak jelas atau halus yang digunakan
untuk melambangkan sesuatu yang umumnya dianggap buruk, kasar, atau
tidak nyaman dibicarakan secara terbuka. \\
\textbf{Etos} & Pilar persuasi menurut Aristoteles yang berkaitan dengan
kredibilitas, integritas, dan rasa hormat yang dimiliki audiens terhadap
pembicara. \\
\textbf{Fitnah (Pencemaran Nama Baik)} & Pernyataan yang dengan sengaja
merusak reputasi seseorang. Terdiri dari \emph{slender} (lisan) dan
\emph{libel} (tertulis). \\
\textbf{Humor} & Sesuatu yang dianggap lucu karena melanggar ekspektasi.
Dapat memperkuat keintiman, tetapi bisa disalahgunakan jika merendahkan
orang lain. \\
\textbf{I-Statement (Pernyataan Saya)} & Pernyataan yang menunjukkan
kepemilikan atas pikiran dan perasaan sendiri tanpa menyalahkan orang
lain. \\
\textbf{Jargon} & Bahasa gaul atau kosakata khusus yang digunakan dalam
kelompok profesional atau lingkungan kerja tertentu, yang mungkin tidak
dipahami oleh orang di luar kelompok tersebut. \\
\textbf{Klaim Faktual} & Pernyataan yang kebenarannya dapat diverifikasi
dengan bukti dan dapat dinilai sebagai ``benar'' atau ``salah''. \\
\textbf{Klise} & Kata-kata atau frasa yang terlalu sering digunakan
sehingga kehilangan maknanya dan dapat merusak kredibilitas
pembicara. \\
\textbf{Kredibilitas} & Sejauh mana orang lain menganggap seseorang
kompeten dan dapat dipercaya. \\
\textbf{Loaded Language (Bahasa Berbobot)} & Kata-kata yang memiliki
arti denotatif netral tetapi arti konotatif yang sangat kuat (positif
atau negatif), sehingga dapat membangkitkan emosi yang kuat. \\
\textbf{Logos} & Pilar persuasi menurut Aristoteles yang menggunakan
akal budi, logika, fakta, dan data statistik untuk meyakinkan. \\
\textbf{Makna Denotatif} & Arti harfiah atau definisi kamus dari sebuah
kata. \\
\textbf{Makna Konotatif} & Arti tersirat, subjektif, dan emosional dari
sebuah kata yang dapat bervariasi antar individu. \\
\textbf{Opini} & Ungkapan penilaian atau keyakinan pribadi yang tidak
dapat diverifikasi sebagai benar atau salah, melainkan ditanggapi dengan
``setuju'' atau ``tidak setuju''. \\
\textbf{Patos} & Pilar persuasi menurut Aristoteles yang berfokus pada
emosi, yaitu upaya untuk memicu perasaan audiens agar lebih menerima
suatu klaim. \\
\textbf{Persuasi} & Proses mencoba menggerakkan orang untuk memiliki
pemikiran tertentu atau melakukan tindakan tertentu. \\
\textbf{Pesan Diskonfirmasi} & Pesan yang mengabaikan, meniadakan, atau
tidak mengakui keberadaan, perasaan, atau pemikiran orang lain. \\
\textbf{Pesan Konfirmasi} & Pesan yang mengakui keberadaan, perasaan,
dan pemikiran orang lain, bahkan tanpa harus setuju sepenuhnya. \\
\textbf{Relativitas Linguistik} & Bagian dari hipotesis Sapir-Whorf yang
menyatakan bahwa orang-orang dengan bahasa berbeda melihat dunia dengan
cara yang berbeda. \\
\textbf{Segitiga Semantik} & Model yang menggambarkan hubungan antara
simbol (kata), arti konotatif (pikiran atau referensi), dan arti
denotatif (objek atau rujukan). \\
\textbf{Slang} & Bahasa informal yang digunakan oleh kelompok sosial
atau generasi tertentu untuk mengidentifikasi diri sebagai bagian dari
kelompok tersebut. \\
\textbf{Ujaran Kebencian (\emph{Hate Speech})} & Bentuk kata-kata kasar
yang dimaksudkan untuk merendahkan, mengintimidasi, atau mengobarkan
kekerasan terhadap seseorang atau kelompok berdasarkan karakteristik
tertentu. \\
\textbf{Weasel Words (Kata-kata Musang)} & Kata-kata terselubung atau
kabur yang menyiratkan sesuatu yang tidak sepenuhnya benar untuk
menyesatkan pendengar. \\
\textbf{You-Statement (Pernyataan Kamu)} & Pernyataan yang cenderung
menyalahkan orang lain sebagai penyebab perasaan atau masalah yang
dialami. \\
\end{longtable}

\bookmarksetup{startatroot}

\chapter{Kuliah 5 Memahami Bahasa Sunyi: Panduan Komunikasi Nonverbal
untuk
Pemula}\label{kuliah-5-memahami-bahasa-sunyi-panduan-komunikasi-nonverbal-untuk-pemula}

KUIZ: \url{https://forms.office.com/r/nTAZaeMXBF}

Video
Clip:\url{https://youtube.com/playlist?list=PL_m-BplfO92GYDdpY5ZKk7HVREye-Oma3&si=3OB2iFn6ehnczvoF}

Podcast : \url{https://youtu.be/YfNuqVdVz3I}

\section{Pendahuluan: Mengapa Gerak Tubuh Lebih Jujur dari
Kata-kata?}\label{pendahuluan-mengapa-gerak-tubuh-lebih-jujur-dari-kata-kata}

Jika kata-kata seseorang bertentangan dengan bahasa tubuhnya, mana yang
akan Anda lebih percaya? Pertanyaan ini menyingkap sebuah kebenaran
fundamental dalam interaksi manusia: kita berkomunikasi jauh lebih
banyak daripada sekadar apa yang kita ucapkan. Ada sebuah ``bahasa
sunyi'' (\emph{the silent language}) yang terus-menerus kita gunakan,
sering kali tanpa kita sadari. Inilah dunia komunikasi nonverbal.

Menurut penelitian, komunikasi nonverbal sering kali dianggap lebih
dapat dipercaya daripada komunikasi verbal, terutama ketika keduanya
saling bertentangan. Kemampuan untuk memahami dan menggunakan bahasa
sunyi ini adalah kunci untuk membangun kredibilitas (atau \emph{ethos}),
yang merupakan fondasi dari semua komunikasi yang efektif.

\begin{center}\rule{0.5\linewidth}{0.5pt}\end{center}

\section{1. Lima Karakteristik Utama Komunikasi
Nonverbal}\label{lima-karakteristik-utama-komunikasi-nonverbal}

Untuk memahami kekuatannya, kita perlu mengenali lima karakteristik
fundamental yang mendefinisikan komunikasi nonverbal.

\begin{enumerate}
\def\labelenumi{\arabic{enumi}.}
\item
  \textbf{Selalu Hadir di Mana Saja} Komunikasi nonverbal hadir dalam
  sebagian besar percakapan antarpribadi. Bahkan saat kita diam, postur
  tubuh atau ekspresi wajah kita tetap mengirimkan pesan. Artinya, Anda
  tidak bisa \emph{tidak} berkomunikasi secara nonverbal, dan menyadari
  hal ini adalah langkah pertama untuk menjadi komunikator yang lebih
  baik.
\item
  \textbf{Menyampaikan Lebih Banyak Informasi} Umumnya, kita
  menyampaikan lebih banyak informasi melalui saluran nonverbal
  dibandingkan dengan kata-kata. Sebuah tatapan, nada suara, atau
  gerakan tangan dapat membawa lapisan makna yang kompleks, artinya
  pesan yang sebenarnya sering kali tersembunyi di antara baris-baris
  kata, yaitu dalam isyarat nonverbal yang menyertainya.
\item
  \textbf{Lebih Bisa Dipercaya} Ketika pesan verbal dan nonverbal tidak
  sinkron, orang cenderung lebih memercayai isyarat nonverbal karena
  tubuh sulit untuk berbohong. Inilah mengapa Anda harus memercayai
  intuisi Anda ketika gerak-gerik seseorang terasa ``tidak pas'' dengan
  apa yang mereka katakan.
\item
  \textbf{Saluran Utama Ekspresi Emosi} Komunikasi nonverbal adalah
  sarana utama kita untuk menyampaikan emosi. Kebahagiaan, kesedihan,
  kemarahan, dan ketakutan lebih jelas terlihat di wajah dan terdengar
  dari nada suara kita daripada dari pilihan kata. Untuk benar-benar
  memahami perasaan seseorang, perhatikan bagaimana mereka
  mengatakannya, bukan hanya apa yang mereka katakan.
\item
  \textbf{Bentuk Meta-Komunikasi} Nonverbal adalah bentuk komunikasi
  tentang komunikasi (\emph{meta-communication}). Isyarat ini memberikan
  konteks pada kata-kata kita; misalnya, senyuman dapat menunjukkan
  bahwa sebuah komentar dimaksudkan sebagai lelucon, bukan hinaan.
  Dengan kata lain, isyarat nonverbal adalah instruksi bagi pendengar
  tentang cara menafsirkan pesan verbal Anda.
\end{enumerate}

Karakteristik ini menunjukkan betapa kuatnya bahasa sunyi. Selanjutnya,
mari kita lihat bagaimana kita menggunakannya dalam fungsi sehari-hari.

\begin{center}\rule{0.5\linewidth}{0.5pt}\end{center}

\section{2. Enam Fungsi Penting dalam Interaksi
Sehari-hari}\label{enam-fungsi-penting-dalam-interaksi-sehari-hari}

Di luar karakteristik dasarnya, komunikasi nonverbal melayani enam
fungsi utama yang membantu kita menavigasi dunia sosial.

\begin{itemize}
\item
  \textbf{Mengatur Percakapan} Kita menggunakan isyarat nonverbal untuk
  mengelola alur percakapan, yang bertindak layaknya rambu lalu lintas
  dalam dialog. Sebagai contoh, kita bisa menganggukkan kepala untuk
  mendorong seseorang agar terus berbicara, atau mengangkat tangan untuk
  memberi sinyal bahwa kita ingin giliran berbicara.
\item
  \textbf{Mengekspresikan Emosi} Fungsi ini adalah yang paling jelas;
  wajah, suara, dan postur kita adalah kanvas untuk melukiskan perasaan
  internal kita, seperti mata yang berbinar dan senyum lebar saat
  menerima kabar baik.
\item
  \textbf{Memelihara Hubungan} Sentuhan, kedekatan fisik, dan kontak
  mata adalah alat penting untuk membangun dan memelihara ikatan dengan
  orang lain. Sebuah tepukan di punggung, misalnya, dapat menunjukkan
  dukungan kepada seorang teman yang sedang sedih.
\item
  \textbf{Membentuk Kesan} Sejak detik pertama pertemuan, orang lain
  membentuk kesan tentang kita berdasarkan penampilan, postur, dan cara
  kita bergerak. Contohnya adalah berjabat tangan dengan erat dan
  menjaga kontak mata saat wawancara kerja untuk menunjukkan kepercayaan
  diri.
\item
  \textbf{Melakukan Persuasi} Gerak tubuh yang ekspresif, nada suara
  yang meyakinkan, dan postur yang terbuka dapat secara signifikan
  meningkatkan daya persuasi pesan kita, seperti seorang pembicara yang
  menggunakan gerakan tangan dinamis untuk menekankan poin-poin penting
  dalam presentasinya.
\item
  \textbf{Menyembunyikan Informasi} Terkadang, kita menggunakan
  komunikasi nonverbal untuk menyembunyikan perasaan atau informasi yang
  sebenarnya. Salah satu contohnya adalah menjaga ``wajah datar'' atau
  \emph{poker face} saat negosiasi untuk tidak mengungkapkan posisi kita
  yang sebenarnya.
\end{itemize}

Fungsi-fungsi ini dikirim melalui berbagai saluran yang berbeda.

\begin{center}\rule{0.5\linewidth}{0.5pt}\end{center}

\section{3. Saluran Komunikasi: Bagaimana Pesan Nonverbal
Dikirim?}\label{saluran-komunikasi-bagaimana-pesan-nonverbal-dikirim}

Pesan nonverbal dikirim dan diterima melalui berbagai ``saluran''.
Memahami saluran ini membantu kita menjadi lebih peka dalam membaca
isyarat orang lain.

\begin{longtable}[]{@{}
  >{\raggedright\arraybackslash}p{(\linewidth - 4\tabcolsep) * \real{0.3333}}
  >{\raggedright\arraybackslash}p{(\linewidth - 4\tabcolsep) * \real{0.3333}}
  >{\raggedright\arraybackslash}p{(\linewidth - 4\tabcolsep) * \real{0.3333}}@{}}
\toprule\noalign{}
\endhead
\bottomrule\noalign{}
\endlastfoot
Saluran (Channel) & Contoh Sederhana & Pesan yang Disampaikan \\
\textbf{Wajah \& Mata} & Tersenyum, kontak mata, cemberut, membelalakkan
mata. & Emosi (senang, marah), ketertarikan, perhatian, kejutan. \\
\textbf{Gerak Tubuh \& Postur} & Mengangguk, melambaikan tangan, postur
tegap, membungkuk. & Persetujuan, sapaan, kepercayaan diri, rasa tidak
nyaman. \\
\textbf{Suara (Vokalisasi)} & Nada suara tinggi, berbicara cepat,
berbisik, diam sejenak. & Kegembiraan, urgensi, kerahasiaan, penekanan,
ketegangan. \\
\textbf{Sentuhan} & Berjabat tangan, menepuk punggung, menggenggam
tangan. & Sapaan formal, dukungan, kasih sayang, kendali. \\
\textbf{Ruang \& Waktu} & Berdiri sangat dekat, datang terlambat ke
rapat. & Keintiman (ruang intim), status, rasa hormat (atau
kurangnya). \\
\end{longtable}

Meskipun ini adalah saluran utama, pesan nonverbal juga dapat dikirim
melalui indra penciuman (wewangian) atau benda-benda yang kita gunakan
(artifak), seperti pakaian atau gawai.

Setelah memahami cara pesan ini dikirim, kita dapat melangkah lebih
jauh: bagaimana isyarat kita sendiri dapat memengaruhi pikiran dan
perasaan kita?

\begin{center}\rule{0.5\linewidth}{0.5pt}\end{center}

\section{4. Mengubah Diri dari Luar ke Dalam: Kekuatan ``Power
Posing''}\label{mengubah-diri-dari-luar-ke-dalam-kekuatan-power-posing}

Penelitian dari psikolog sosial Amy Cuddy mengungkapkan sebuah ide yang
revolusioner: perilaku nonverbal kita tidak hanya memengaruhi orang
lain, tetapi juga memengaruhi pikiran dan perasaan kita sendiri. Dengan
kata lain, \textbf{tubuh kita dapat mengubah pikiran kita.}

Konsep utamanya adalah perbedaan antara ``pose berkuasa'' (\emph{power
poses}) dan ``pose tidak berdaya'' (\emph{powerless poses}).

\begin{itemize}
\item
  \textbf{Power Poses:} Ini adalah postur yang ekspansif dan terbuka.
  Bayangkan berdiri dengan tangan di pinggang (seperti Wonder Woman)
  atau merentangkan tangan ke atas membentuk huruf V setelah meraih
  kemenangan. Pose ini mengambil ruang.
\item
  \textbf{Powerless Poses:} Ini adalah postur yang tertutup dan membuat
  diri terlihat kecil. Contohnya termasuk menyilangkan lengan,
  membungkuk, atau merapatkan kaki dengan erat.
\end{itemize}

Wawasan fisiologis utamanya adalah bahwa berpose seperti ini dapat
mengubah hormon dalam tubuh kita. \emph{Power poses} dapat meningkatkan
kadar testosteron (hormon dominasi) dan menurunkan kadar kortisol
(hormon stres).

Saran yang paling penting dan dapat segera Anda praktikkan adalah:

Sebelum menghadapi situasi yang penuh tekanan (seperti wawancara,
presentasi, atau ujian), carilah ruang pribadi---seperti toilet atau
ruangan kosong---dan tahan \emph{power pose} selama \textbf{dua menit}.

Tujuannya bukan untuk menunjukkan dominasi kepada orang lain, tetapi
untuk mengubah kondisi internal Anda sendiri. Ini adalah aplikasi nyata
dari konsep \emph{``Fake it till you become it''} (Pura-pura sampai Anda
menjadi seperti itu). Dengan mengubah postur, Anda mempersiapkan pikiran
Anda untuk merasa lebih percaya diri dan tenang.

\begin{center}\rule{0.5\linewidth}{0.5pt}\end{center}

\section{5. Dua Langkah untuk Meningkatkan Keterampilan Nonverbal
Anda}\label{dua-langkah-untuk-meningkatkan-keterampilan-nonverbal-anda}

Menguasai bahasa sunyi adalah sebuah perjalanan, tetapi dapat dimulai
dengan dua langkah praktis.

\begin{enumerate}
\def\labelenumi{\arabic{enumi}.}
\item
  \textbf{Menjadi Lebih Peka} Langkah pertama adalah secara sadar
  memperhatikan isyarat nonverbal orang lain. Latihlah diri Anda untuk
  mengamati berbagai \textbf{saluran yang telah kita bahas di Bagian
  3}---mulai dari ekspresi wajah dan kontak mata hingga nada suara dan
  postur---dan belajarlah menafsirkan maknanya dalam konteks. Kepekaan
  ini akan mempertajam kemampuan Anda untuk ``mendengar'' apa yang tidak
  diucapkan.
\item
  \textbf{Berlatih Menjadi Lebih Ekspresif} Langkah kedua adalah
  meningkatkan kemampuan Anda untuk menggunakan isyarat nonverbal secara
  efektif. Amati orang-orang yang Anda anggap ekspresif dan belajarlah
  dari mereka. Berlatihlah menggunakan gerak tubuh dan variasi vokal
  untuk membuat komunikasi Anda lebih hidup dan kuat, sehingga Anda
  tidak ``berbicara seperti robot''. Keterampilan ini sangat penting,
  terutama untuk persuasi yang efektif.
\end{enumerate}

Menguasai komunikasi nonverbal tidak terjadi dalam semalam. Namun,
seperti yang ditunjukkan oleh konsep \emph{power posing}, perubahan
kecil dapat menghasilkan dampak yang besar. Ingatlah selalu kata-kata
Amy Cuddy: \emph{``Tiny tweaks can lead to big changes''}---perubahan
kecil dapat membawa perubahan besar. Dengan sedikit kesadaran dan
latihan, Anda dapat membuka kekuatan bahasa sunyi untuk menjadi
komunikator yang lebih kredibel, peka, dan berpengaruh.

\bookmarksetup{startatroot}

\chapter{10 Pertanyaan Kuiz Komunikasi
Nonverbal}\label{pertanyaan-kuiz-komunikasi-nonverbal}

\begin{enumerate}
\def\labelenumi{\arabic{enumi}.}
\tightlist
\item
  Menurut sumber, apa ciri utama dari komunikasi nonverbal yang
  membedakannya dari komunikasi verbal?
\item
  Sebutkan lima karakteristik utama dari komunikasi nonverbal yang
  membuatnya penting dalam interaksi antarpribadi.
\item
  Dalam konteks kredibilitas (Ethos), jika terjadi konflik antara pesan
  verbal yang meyakinkan dan isyarat nonverbal yang gugup (seperti
  postur kaku), pesan mana yang cenderung lebih dipercaya?
\item
  Selain mimik tampilan wajah dan perilaku mata, sebutkan minimal tiga
  dari sepuluh saluran komunikasi nonverbal lainnya yang dibahas dalam
  sumber.
\item
  Komunikasi nonverbal memiliki enam fungsi, salah satunya adalah
  \emph{consiling}. Jelaskan fungsi \emph{consiling} dan berikan dua
  fungsi nonverbal lainnya.
\item
  \emph{Chronemics} adalah salah satu saluran komunikasi nonverbal yang
  melibatkan penggunaan waktu. Jelaskan perbedaan konsep budaya
  \emph{monokronik} dan \emph{polikronik} terkait pengelolaan waktu.
\item
  Apa yang dimaksud dengan \emph{efek Halo} dalam kaitannya dengan
  tampilan fisik dan komunikasi nonverbal?
\item
  Dalam konteks ekspresi kekuasaan dan dominasi nonverbal, individu yang
  memiliki \emph{high power} (kekuasaan tinggi) cenderung memiliki
  tingkat dua hormon kunci apa?
\item
  Perilaku nonverbal yang fokus pada mempertahankan luas daerah atau
  domain disebut \emph{ruang}. Sebutkan empat tingkat domain ruang yang
  dijelaskan dalam sumber.
\item
  Untuk meningkatkan keterampilan komunikasi nonverbal, dua saran utama
  yang diberikan adalah menjadi \emph{peka} dan menjadi
  \emph{ekspresif}. Jelaskan apa yang dimaksud dengan kemampuan untuk
  menjadi ``peka'' terhadap pesan nonverbal.
\end{enumerate}

\bookmarksetup{startatroot}

\chapter{Kuliah 6 Ketrampilan Komunikasi
Pribadi}\label{kuliah-6-ketrampilan-komunikasi-pribadi}

KUiz 6: \url{https://forms.office.com/r/xJBpLWQCKU}

Materi:
\url{https://youtube.com/playlist?list=PL_m-BplfO92G6xpXg-NRsWUo2yfhqZwAH&si=r19H8cEcBxa5jxSU}

\href{../bahan/Akitivitas_6.xlsx}{Lembar Kerja}

\section{I. Agenda Kuliah (120 Menit)}\label{i.-agenda-kuliah-120-menit}

\textbf{Mata Kuliah:} II-4472 Komunikasi Interpersonal (Kuliah 13:
Keterampilan Komunikasi Interpersonal) \textbf{Tema Utama:}
Mengembangkan Keterampilan Komunikasi Interpersonal Tingkat Lanjut.

\begin{longtable}[]{@{}
  >{\raggedright\arraybackslash}p{(\linewidth - 8\tabcolsep) * \real{0.2000}}
  >{\raggedright\arraybackslash}p{(\linewidth - 8\tabcolsep) * \real{0.2000}}
  >{\raggedright\arraybackslash}p{(\linewidth - 8\tabcolsep) * \real{0.2000}}
  >{\raggedright\arraybackslash}p{(\linewidth - 8\tabcolsep) * \real{0.2000}}
  >{\raggedright\arraybackslash}p{(\linewidth - 8\tabcolsep) * \real{0.2000}}@{}}
\toprule\noalign{}
\begin{minipage}[b]{\linewidth}\raggedright
Waktu
\end{minipage} & \begin{minipage}[b]{\linewidth}\raggedright
Durasi
\end{minipage} & \begin{minipage}[b]{\linewidth}\raggedright
Topik/Aktivitas
\end{minipage} & \begin{minipage}[b]{\linewidth}\raggedright
Metode \& Keterlibatan (\emph{Engagement})
\end{minipage} & \begin{minipage}[b]{\linewidth}\raggedright
Sumber Acuan
\end{minipage} \\
\midrule\noalign{}
\endhead
\bottomrule\noalign{}
\endlastfoot
\textbf{09:00} & 10 menit & \textbf{A. Ekspresi Diri dan Penerimaan
Diri} & Ceramah singkat tentang afirmasi (termasuk praktik ``berbicara
pada diri sendiri di depan cermin'') dan pentingnya menyukai diri
sendiri. & \\
\textbf{09:10} & 15 menit & \textbf{B. Menghindari Perilaku Mencari
Persetujuan (\emph{Approval Seeking})} & Ceramah interaktif: Membahas
strategi mental (Percaya diri, membedakan pikiran/perasaan). Dilanjutkan
dengan \textbf{Aktivitas 13.1 \& 13.2} (Diskusi singkat hasil kuesioner,
jika telah diisi). & \\
\textbf{09:25} & 20 menit & \textbf{C. Mengelola Perbedaan Sudut
Pandang} & \textbf{1. Power \& Kepatuhan:} Menjelaskan definisi
\emph{power} dan strategi kepatuhan (Take \& Give, Otoritas, Utang
Budi). \textbf{2. Meminta Maaf:} Diskusi tentang cara meminta maaf yang
efektif: perbaiki kerusakan dan obati luka emosional. & \\
\textbf{09:45} & 10 menit & \textbf{QUIZ BREAK 1 (Soal 1-5)} & Peserta
mengerjakan kuis pendek untuk menguji pemahaman materi awal. & \\
\textbf{09:55} & 15 menit & \textbf{D. Meningkatkan Keterampilan
Percakapan} & \textbf{1. Small Talk \& Mendengarkan:} Strategi memulai
percakapan dan menjaga percakapan (bertanya, memparafrasekan,
menggunakan nama lawan bicara). \textbf{2. Percakapan Mendalam:}
Membahas pendekatan analitis vs.~holistik. & \\
\textbf{10:10} & 15 menit & \textbf{E. Strategi Komunikasi Berisiko
Tinggi} & \textbf{1. Memberi Arahan:} Pentingnya rincian spesifik dan
urutan kronologis/tata ruang. \textbf{2. Probing:} Fungsi
\emph{restatement}, definisi, dan klarifikasi untuk menghilangkan
\emph{noise}. \textbf{3. Kabar Buruk:} Strategi yang efektif
(mendengarkan empati, jangan gunakan klise). & \\
\textbf{10:25} & 15 menit & \textbf{F. Membangkitkan Kreativitas
Pribadi} & Membahas manfaat kreativitas (memecahkan masalah, keluar dari
rutinitas) dan hambatan (rasa takut gagal). Menjelaskan dua teknik:
pendekatan analitis (mengurai komponen dan mencari alternatif) dan
manipulasi detail. & \\
\textbf{10:40} & 10 menit & \textbf{QUIZ BREAK 2 (Soal 6-10)} & Peserta
mengerjakan kuis kedua. & \\
\textbf{10:50} & 10 menit & \textbf{Penutup dan Q\&A} & Review singkat
materi penting, penugasan proyek kelompok (Digital Contents), dan sesi
tanya jawab. & \\
\end{longtable}

\begin{center}\rule{0.5\linewidth}{0.5pt}\end{center}

\section{II. Lecture Notes (Materi
Kuliah)}\label{ii.-lecture-notes-materi-kuliah}

\subsection{A. Ekspresi Diri dan Pengungkapan Diri yang
Sesuai}\label{a.-ekspresi-diri-dan-pengungkapan-diri-yang-sesuai}

\begin{enumerate}
\def\labelenumi{\arabic{enumi}.}
\tightlist
\item
  \textbf{Ekspresi Diri adalah Pesan Komunikasi:} Semua komunikasi
  merupakan pesan ekspresi dari diri kita tentang perasaan, pikiran, dan
  opini kita.
\item
  \textbf{Pentingnya Menyukai Diri Sendiri (Afirmasi):} Untuk
  berekspresi dengan baik, penting untuk menyukai diri sendiri, yang
  berarti memutuskan untuk menerima diri sendiri sebagai orang yang
  layak. Afirmasi adalah pernyataan positif yang dimaksudkan untuk
  memandu pemikiran positif, misalnya: ``Saya adalah orang yang baik''.
  Ini bisa dilakukan melalui pembicaraan diri yang positif di depan
  cermin setiap pagi.
\item
  \textbf{Pengungkapan Diri (\emph{Self-Disclosure}):} Melibatkan
  berbagi informasi tentang diri sendiri, termasuk sejarah hidup, emosi,
  dan pikiran saat ini. Pengungkapan diri harus \emph{appropriate}
  (sesuai) dengan atmosfer dan jenis hubungan. Tujuannya adalah membantu
  orang lain mengenali siapa kita dan membantu kita memahami diri
  sendiri.
\item
  \textbf{Menjadi Komunikator Efektif:} Pesan kita harus merupakan
  \textbf{ekspresi otentik} dari diri kita, bukan pemikiran orang lain.
  Gunakan kata ganti pribadi \textbf{``Saya''} untuk mengekspresikan
  kepemilikan gagasan.
\end{enumerate}

\subsection{\texorpdfstring{B. Mengenali dan Menghentikan Perilaku
Mencari Persetujuan (\emph{Approval
Seeking})}{B. Mengenali dan Menghentikan Perilaku Mencari Persetujuan (Approval Seeking)}}\label{b.-mengenali-dan-menghentikan-perilaku-mencari-persetujuan-approval-seeking}

Perilaku mencari persetujuan (\emph{approval seeking}) dapat berakibat
buruk pada relasi. Cara menghilangkannya adalah dengan menghindari
saling mendominasi.

\begin{enumerate}
\def\labelenumi{\arabic{enumi}.}
\tightlist
\item
  \textbf{Persiapan Mental:}

  \begin{itemize}
  \tightlist
  \item
    Ikuti tujuan pribadi Anda dalam komunikasi.
  \item
    Belajar menerima diri sendiri, termasuk bahwa tidak semua orang akan
    memahami Anda, dan itu \textbf{tidak masalah}.
  \item
    Percaya dan Yakin pada diri sendiri (Yakin bahwa Anda memiliki
    kemampuan berdasarkan pengalaman dan nilai-nilai pribadi).
  \item
    Ketahui perbedaan antara \textbf{perasaan} (emosi, tidak perlu
    bukti) dan \textbf{pikiran} (logis, dapat diperdebatkan).
  \end{itemize}
\item
  \textbf{Ekspresi dan Perbuatan:}

  \begin{itemize}
  \tightlist
  \item
    \emph{Speak up} dan jangan langsung menerima jika teman tidak
    menyetujui.
  \item
    Berhenti mencoba memverifikasi ide-ide Anda dengan membuatnya
    terbukti bagi orang lain.
  \item
    Jangan terlalu mudah meminta maaf untuk pendapat sendiri.
  \end{itemize}
\end{enumerate}

\subsection{C. Mengelola Sudut Pandang yang
Berbeda}\label{c.-mengelola-sudut-pandang-yang-berbeda}

\begin{enumerate}
\def\labelenumi{\arabic{enumi}.}
\tightlist
\item
  \textbf{Strategi Perolehan Kepatuhan (\emph{Compliance Strategies}):}
  Strategi bersifat transaksional (\emph{take and give}).

  \begin{itemize}
  \tightlist
  \item
    \textbf{\emph{Pregiving}}: Memberi sesuatu terlebih dahulu, berharap
    bantuan di lain waktu.
  \item
    \textbf{\emph{Promising}}: Menjanjikan sesuatu sebagai imbalan.
  \item
    \textbf{\emph{Otoritas Influencing}}: Menggunakan kuasa, permintaan,
    perintah, bahkan ancaman (konsekuensi negatif dari ketidaksetujuan).
  \end{itemize}
\item
  \textbf{Kekuatan (\emph{Power}):} \emph{Power} adalah kemampuan untuk
  mengontrol apa yang terjadi---menciptakan hal yang diinginkan dan
  menghalangi hal yang tidak diinginkan. Gagal menggunakan \emph{power}
  dapat menyebabkan rasa tidak berdaya dan frustrasi.
\item
  \textbf{Meminta Maaf dengan Tepat:} Permintaan maaf adalah ungkapan
  penyesalan yang sungguh-sungguh atas kata-kata atau perbuatan yang
  tidak pantas.

  \begin{itemize}
  \tightlist
  \item
    \textbf{Tujuan:} Menyembuhkan luka dan/atau memperbaiki kerusakan.
    Jika merusak properti, perbaiki. Jika kerusakan emosional, tanyakan:
    ``Apa yang bisa saya lakukan untuk menebusnya?''.
  \item
    \textbf{Timing:} Hal kecil dilakukan saat itu juga. Hal besar
    memerlukan waktu untuk mencerna kerusakan dan mengumpulkan kata-kata
    yang layak.
  \item
    \textbf{Saluran:} Tatap muka dianggap lebih baik untuk perbaikan
    emosional dibandingkan saluran impersonal (surat, email).
  \end{itemize}
\end{enumerate}

\subsection{D. Meningkatkan Keterampilan
Percakapan}\label{d.-meningkatkan-keterampilan-percakapan}

\begin{enumerate}
\def\labelenumi{\arabic{enumi}.}
\tightlist
\item
  \textbf{Memulai Percakapan (\emph{Small Talk}):} Interaksi verbal yang
  dimulai dengan bincang-bincang kecil sebelum pindah ke berbagi yang
  lebih mendalam. \emph{Small talk} adalah pertukaran informasi pada
  tingkat permukaan (biografi, pekerjaan, hobi).
\item
  \textbf{Menjaga Percakapan:} Gunakan pertanyaan sebagai instrumen yang
  kuat untuk membangun percakapan, mendorong orang membuka diri,
  mengendalikan arah pembicaraan, dan mendapatkan informasi
  baru/klarifikasi.
\item
  \textbf{Keterampilan Mendengarkan:} Pendengar yang baik lebih efektif
  berbicara dengan orang asing.

  \begin{itemize}
  \tightlist
  \item
    \textbf{Memparafrasekan:} Mengucapkan ulang ide pembicara dengan
    bahasa kita sendiri.
  \item
    \textbf{Menggunakan Nama:} Ulangi nama orang tersebut saat
    diperkenalkan dan terus gunakan selama percakapan agar fokus
    perhatian Anda terpusat padanya.
  \end{itemize}
\item
  \textbf{Memberikan Arahan (\emph{Direction}):} Arahan (instruksi,
  petunjuk tempat) harus jelas.

  \begin{itemize}
  \tightlist
  \item
    Berikan rincian spesifik dan sesuaikan pesan dengan tingkat
    pengetahuan pendengar (pendidikan, pengalaman, linguistik).
  \item
    Petunjuk paling mudah diikuti jika diurutkan secara
    \textbf{kronologis} (urutan waktu) atau \textbf{tata ruang} (urutan
    tempat).
  \end{itemize}
\end{enumerate}

\subsection{E. Strategi Praktis untuk Komunikasi Berisiko
Tinggi}\label{e.-strategi-praktis-untuk-komunikasi-berisiko-tinggi}

\begin{enumerate}
\def\labelenumi{\arabic{enumi}.}
\tightlist
\item
  \textbf{Menyelidik (\emph{Probing}):} Tujuannya adalah meminta
  informasi yang diperlukan untuk memastikan pesan jelas dan bebas dari
  \emph{noise}. Kebingungan biasanya memerlukan:

  \begin{itemize}
  \tightlist
  \item
    \textbf{\emph{Restatement}}: Mengucapkan ulang untuk memahami orang
    lain (misalnya, menanyakan kembali langkah pertama dan kedua).
  \item
    \textbf{Definisi:} Untuk menghindari masalah kosakata yang samar
    atau sulit (misalnya, menanyakan arti istilah medis).
  \item
    \textbf{Klarifikasi:} Memahami arti pesan yang tidak cukup jelas,
    seringkali dengan meminta contoh, ilustrasi, atau analogi.
  \end{itemize}
\item
  \textbf{Membuat Permintaan Informasi:} Permintaan harus menyertakan
  bahasa tertentu yang menjelaskan bagaimana permintaan harus dijawab,
  bentuk jawaban yang diharapkan, dan \emph{deadline} (waktu dan
  tanggal) untuk menerima informasi.
\item
  \textbf{Menyampaikan Berita Buruk:} Jika dikomunikasikan dengan buruk
  dapat mengakibatkan kebingungan, stres, bahkan kemarahan.

  \begin{itemize}
  \tightlist
  \item
    Jaga kebutuhan mendesak (\emph{immediate needs}) orang tersebut.
  \item
    Dengarkan dengan \textbf{empati}, jangan memberikan saran.
  \item
    Jangan mengecilkan kesedihan atau kehilangan dengan pernyataan
    klise.
  \item
    Dorong dia untuk menampilkan emosi dan bantu terhubung dengan
    kelompok dukungan atau profesional.
  \end{itemize}
\end{enumerate}

\subsection{F. Membangkitkan Kreativitas
Pribadi}\label{f.-membangkitkan-kreativitas-pribadi}

Kreativitas adalah kemampuan menciptakan hal-hal baru dan berpikir
dengan cara yang tidak biasa.

\begin{enumerate}
\def\labelenumi{\arabic{enumi}.}
\tightlist
\item
  \textbf{Manfaat:} Penting untuk memecahkan masalah yang tidak dapat
  dipecahkan dengan cara konvensional dan membuat kita keluar dari
  rutinitas yang membosankan.
\item
  \textbf{Hambatan:} Penghalang pribadi (ketakutan akan kegagalan,
  kebiasaan lama) dan penghalang budaya (menghargai logika tinggi, harus
  praktis).
\item
  \textbf{Teknik Kreativitas:}

  \begin{itemize}
  \tightlist
  \item
    \textbf{Pendekatan Analitis (\emph{Analytical Decomposition}):}
    Menguraikan hal yang rutin (misalnya rapat) ke dalam
    komponen-komponennya (Siapa, Kapan, Di mana). Kemudian, carikan
    alternatif untuk setiap komponen (misalnya, tempat rapat tidak harus
    di ruang kantor, bisa di taman atau kafe). Kombinasikan
    komponen-komponen baru tersebut.
  \item
    \textbf{Manipulasi Detail:} Mengubah bentuk suatu hal dengan cara
    tertentu, misalnya membuat lebih besar, lebih kecil, memutar,
    mengangkat, atau membagi. (Contoh: evolusi ukuran komputer dari
    gedung menjadi kecil).
  \end{itemize}
\end{enumerate}

\begin{center}\rule{0.5\linewidth}{0.5pt}\end{center}

\section{III. Poin-Poin Slides
Pendukung}\label{iii.-poin-poin-slides-pendukung}

\begin{longtable}[]{@{}
  >{\raggedright\arraybackslash}p{(\linewidth - 4\tabcolsep) * \real{0.3333}}
  >{\raggedright\arraybackslash}p{(\linewidth - 4\tabcolsep) * \real{0.3333}}
  >{\raggedright\arraybackslash}p{(\linewidth - 4\tabcolsep) * \real{0.3333}}@{}}
\toprule\noalign{}
\begin{minipage}[b]{\linewidth}\raggedright
Slide
\end{minipage} & \begin{minipage}[b]{\linewidth}\raggedright
Topik Utama
\end{minipage} & \begin{minipage}[b]{\linewidth}\raggedright
Poin-Poin Kunci (Dikutip dari Sumber)
\end{minipage} \\
\midrule\noalign{}
\endhead
\bottomrule\noalign{}
\endlastfoot
\textbf{1-3} & \textbf{Pengantar \& Ekspresi Diri} & Ekspresi Diri
adalah pesan dari diri kita tentang perasaan, pikiran, opini. Sukai dan
terima diri sendiri (Afirmasi). Gunakan pembicaraan positif: ``Saya
adalah orang yang baik''. Pesan harus ekspresi otentik diri kita. \\
\textbf{4-5} & \textbf{Approval Seeking} & Kebanyakan orang ingin
didukung, diterima---tapi mencari persetujuan bisa berakibat buruk. Cara
menghindari: Jangan mendominasi atau didominasi. Persiapan Mental: Ikuti
tujuan pribadi; Terima bahwa orang lain tidak akan selalu memahami Anda.
Ekspresi: \emph{Speak up}, jangan mudah meminta maaf atas pendapatmu. \\
\textbf{6-7} & \textbf{Mengelola Perbedaan I: Kepatuhan \& Power} &
Strategi Kepatuhan bersifat transaksional (\emph{take and give}). Contoh
Strategi: \emph{Pregiving}, \emph{Promising}, \emph{Menyanjung}
(Penghargaan), \emph{Otoritas} (Kuasa/Ancaman). \emph{Power} adalah
kemampuan mengontrol apa yang terjadi dan memblokir hal yang tidak
diinginkan. \\
\textbf{8-9} & \textbf{Mengelola Perbedaan II: Meminta Maaf} &
Permintaan maaf adalah ungkapan penyesalan yang sungguh-sungguh. Cara
efektif: Sembuhkan luka emosional atau perbaiki kerusakan fisik.
Perbaiki kerusakan emosional dengan bertanya: ``Apa yang bisa saya
lakukan untuk menebusnya?''. \\
\textbf{10-12} & \textbf{Keterampilan Percakapan} & Percakapan dimulai
dengan \emph{small talk} (pertukaran informasi permukaan: biografi,
hobi). Gunakan pertanyaan untuk: mendorong orang membuka diri,
mengendalikan arah pembicaraan, dan mendapatkan klarifikasi.
\textbf{Mendengarkan:} Memparafrasekan ide pembicara. Ulangi dan gunakan
nama orang tersebut agar mereka merasa diperhatikan. \\
\textbf{13-14} & \textbf{Memberikan Arahan \& Probing} & Arahan paling
mudah diikuti bila diberikan dalam urutan \textbf{kronologis} atau
\textbf{tata ruang}. \emph{Probing} (Menyelidik) memastikan pesan jelas.
Tiga alat \emph{probing}: \emph{Restatement} (mengucapkan ulang),
\emph{Definisi} (menghindari kosakata samar), dan \emph{Klarifikasi}
(meminta contoh/analogi). \\
\textbf{15-16} & \textbf{Menyampaikan Berita Buruk} & Gagal
menyampaikannya dapat mengakibatkan kebingungan/kemarahan. Strategi
efektif: Jaga kebutuhan mendesak orang tersebut. Dengarkan empati,
jangan berikan saran. Jangan gunakan pernyataan klise. \\
\textbf{17-18} & \textbf{Kreativitas} & Manfaat: Memecahkan masalah dan
keluar dari rutinitas yang membosankan. Hambatan: Takut akan kegagalan
(takut dianggap tidak cerdas atau tidak praktis). Teknik Analitis:
Uraikan komponen (Kapan, Di mana, Siapa), lalu cari alternatif baru
(Contoh Kasus Rapat Staf). \\
\end{longtable}

\begin{center}\rule{0.5\linewidth}{0.5pt}\end{center}

\section{IV. 10 Soal Kuis}\label{iv.-10-soal-kuis}

\subsection{Kuis 1 (Soal 1 - 5)}\label{kuis-1-soal-1---5}

\begin{enumerate}
\def\labelenumi{\arabic{enumi}.}
\tightlist
\item
  Menurut sumber, apa fungsi utama dari afirmasi dalam ekspresi diri?
\item
  Pengungkapan diri (\emph{Self-disclosure}) harus \emph{appropriate}
  (sesuai). Apa dua faktor utama yang menentukan seberapa jauh kita
  harus membiarkan diri terekspos dalam pengungkapan diri?
\item
  Mengapa perilaku mencari persetujuan (\emph{approval seeking}) dapat
  berakibat buruk pada relasi, meskipun orang wajar menginginkan
  penerimaan?
\item
  Jelaskan dan berikan contoh dua strategi perolehan kepatuhan
  (\emph{compliance strategies}) dalam pendekatan \emph{take and give}
  yang bersifat transaksional.
\item
  Dalam konteks permintaan maaf yang efektif, jelaskan perbedaan langkah
  yang harus dilakukan jika terjadi kerusakan properti fisik versus
  kerusakan emosional.
\end{enumerate}

\subsection{Kuis 2 (Soal 6 - 10)}\label{kuis-2-soal-6---10}

\begin{enumerate}
\def\labelenumi{\arabic{enumi}.}
\setcounter{enumi}{5}
\tightlist
\item
  Keterampilan bertanya merupakan instrumen yang kuat dalam percakapan.
  Sebutkan tiga manfaat yang dapat diperoleh komunikator melalui
  pertanyaan.
\item
  Sebutkan dua keterampilan mendengarkan yang penting yang membantu
  seseorang mengingat nama orang lain dan membuatnya merasa
  diperhatikan.
\item
  Dalam konteks memberikan arahan, bagaimana cara menata petunjuk agar
  paling mudah diikuti oleh pendengar?
\item
  Jelaskan perbedaan antara \emph{restatement} dan \emph{klarifikasi}
  sebagai alat \emph{probing} saat Anda dihadapkan pada pesan yang tidak
  jelas.
\item
  Untuk mengatasi rutinitas membosankan, salah satu teknik kreativitas
  yang disarankan adalah Pendekatan Analitis. Bagaimana langkah-langkah
  kerja teknik ini?
\end{enumerate}

Beri nama file Activitas\_6.xlsx menjadi Activitas\_6\_NIM.xlsx, simpan
di cloud lalu submit link file tersebut pada form Kuiz 6

\bookmarksetup{startatroot}

\chapter{Kuliah 7 Mendengar}\label{kuliah-7-mendengar}

Kuiz: \url{https://forms.office.com/r/CBThnS750K}

Video:
\url{https://youtube.com/playlist?list=PL_m-BplfO92H6W1WV5WOH24XO0YjAlTXa&si=YQyjnRvKShsCldP5}

Berikut ini Rencana Pembelajaran Semester (RPS) untuk Minggu ke-7,
lengkap dengan pokok bahasan, agenda kuliah 2 jam yang menarik dan
menginspirasi, catatan kuliah, bahan \emph{slides} pendukung, serta 10
pertanyaan kuis beserta jawabannya, dengan mengacu pada materi sumber
yang tersedia.

\begin{center}\rule{0.5\linewidth}{0.5pt}\end{center}

\section{Topik Minggu ke-7 Berdasarkan
RPS}\label{topik-minggu-ke-7-berdasarkan-rps}

Berdasarkan Rencana Pembelajaran Semester (RPS) untuk mata kuliah
Komunikasi Interpersonal dan Publik, \textbf{Topik Minggu ke-7} adalah:

\textbf{Topik Utama: Mendengarkan (Listening)}

\textbf{Kompetensi yang Hendak Dicapai:} Menajamkan kepekaan serta
membesarkan \textbf{kekuatan mendengar secara aktif}, sehingga mahasiswa
dapat memahami apa yang sebenarnya terjadi, memahami diri sendiri,
keluarga, rekan kerja, pelanggan, dan siapapun yang ia temui, serta
lingkungan hidup, sebagai bekal untuk \textbf{berimprovisasi dan
mengembangkan kontribusi yang kreatif}.

\section{Pokok Bahasan dan Pelajaran (Minggu
ke-7)}\label{pokok-bahasan-dan-pelajaran-minggu-ke-7}

Pokok bahasan dan pelajaran utama yang akan disampaikan meliputi:

\begin{enumerate}
\def\labelenumi{\arabic{enumi}.}
\tightlist
\item
  \textbf{Sifat Mendengarkan (The Nature of Listening):} Mendefinisikan
  mendengarkan (Listening) dan membedakannya dari mendengar (Hearing).
\item
  \textbf{Gaya-Gaya Mendengarkan (Listening Styles):} Memahami empat
  gaya mendengarkan utama---\emph{task-oriented}, \emph{relational},
  \emph{analytical}, dan \emph{critical}, serta gaya-gaya lain seperti
  \emph{people-oriented} dan \emph{action-oriented}.
\item
  \textbf{Hambatan Umum Mendengarkan Efektif (Barriers to Effective
  Listening):} Mengidentifikasi kebiasaan mendengarkan yang buruk
  seperti \emph{pseudolistening} (pura-pura mendengarkan) dan
  \emph{selective attention} (perhatian selektif).
\item
  \textbf{Proses Mendengarkan Aktif (\emph{Mindful Listening}):}
  Mempelajari lima komponen proses mendengarkan: \emph{Hearing},
  \emph{Attending}, \emph{Understanding}, \emph{Remembering}, dan
  \emph{Responding} (Model HURIER).
\item
  \textbf{Teknik Respons Mendengarkan Efektif:} Menguasai spektrum
  respons, dari yang reflektif (seperti \emph{Silent Listening} dan
  \textbf{Paraphrasing}) hingga yang direktif (seperti \emph{Analyzing}
  dan \emph{Advising}).
\end{enumerate}

\begin{center}\rule{0.5\linewidth}{0.5pt}\end{center}

\section{Agenda dan Isi Kuliah 2 Jam (Menarik, Captivating, dan
Inspiring)}\label{agenda-dan-isi-kuliah-2-jam-menarik-captivating-dan-inspiring}

Kuliah ini dirancang untuk durasi 120 menit, menggunakan teknik
\emph{storytelling}, kontras, dan \emph{call-to-action} untuk mencapai
\emph{resonance}.

\begin{longtable}[]{@{}
  >{\centering\arraybackslash}p{(\linewidth - 6\tabcolsep) * \real{0.2941}}
  >{\raggedright\arraybackslash}p{(\linewidth - 6\tabcolsep) * \real{0.2353}}
  >{\raggedright\arraybackslash}p{(\linewidth - 6\tabcolsep) * \real{0.2353}}
  >{\raggedright\arraybackslash}p{(\linewidth - 6\tabcolsep) * \real{0.2353}}@{}}
\toprule\noalign{}
\begin{minipage}[b]{\linewidth}\centering
Waktu
\end{minipage} & \begin{minipage}[b]{\linewidth}\raggedright
Agenda Kuliah
\end{minipage} & \begin{minipage}[b]{\linewidth}\raggedright
Isi dan Aktivitas (Menarik \& Inspiratif)
\end{minipage} & \begin{minipage}[b]{\linewidth}\raggedright
Kompetensi/Pelajaran
\end{minipage} \\
\midrule\noalign{}
\endhead
\bottomrule\noalign{}
\endlastfoot
\textbf{0--10 Menit} & \textbf{Pembukaan: ``The Cost of Not Listening''}
& Mulai dengan \textbf{sebuah cerita/anekdot \emph{captivating}}
mengenai kegagalan komunikasi serius (misalnya, salah paham di tempat
kerja/keluarga yang berakibat fatal atau memalukan, atau kisah nyata di
mana kegagalan mendengarkan menyebabkan kerugian signifikan, seperti
kesalahan medis atau kecelakaan). Tekankan bahwa mendengarkan adalah
keterampilan yang \textbf{mengubah kehidupan}. & Pendahuluan:
Mendengarkan sebagai keterampilan yang dipelajari. \\
\textbf{10--30 Menit} & \textbf{Jembatan Konseptual: Dari Kebisingan ke
Pemahaman} & \textbf{A. Mendengarkan vs.~Mendengar:} Bedakan
\emph{Hearing} (proses pasif/fisiologis) dari \emph{Listening} (proses
aktif, sadar, dan membuat makna). Tinjau kembali konsep \emph{Noise}
(Semantic, Organizational, Cultural, Psychological) yang mengganggu
pendengaran. Bahas \textbf{Mindful Listening} (perhatian penuh dan
tanggapan yang bijaksana) vs.~\textbf{Mindless Listening} (reaksi
otomatis). & Definisi dan Sifat Mendengarkan. \\
\textbf{30--50 Menit} & \textbf{Aktivitas: Kenali Gaya Anda} &
\textbf{B. Gaya Mendengarkan:} Perkenalkan empat/enam gaya mendengarkan.
Minta mahasiswa melakukan \textbf{Self-Assessment Cepat} (misalnya,
memilih skenario yang paling menggambarkan reaksi mereka). Diskusikan
bagaimana gaya-gaya ini dipengaruhi oleh budaya (misalnya, budaya
individualistik menghargai efisiensi; budaya kolektivis menghargai
harmoni/nonverbal). & Gaya-Gaya Mendengarkan. \\
\textbf{50--70 Menit} & \textbf{Tembok Penghalang Komunikasi} &
\textbf{C. Hambatan:} Fokus pada hambatan kognitif dan perilaku.
Jelaskan \emph{pseudolistening} (pura-pura mendengarkan) dan
\emph{rebuttal tendency} (kecenderungan langsung membantah).
\emph{Refleksi Pathos}: Tanyakan, ``Bagaimana perasaan Anda ketika Anda
menyadari seseorang \emph{pseudolistening}?''. & Hambatan Umum
Mendengarkan Efektif. \\
\textbf{70--90 Menit} & \textbf{Simulasi: Mendengarkan Aktif
(Paraphrasing Power)} & \textbf{D. Respon Mendengarkan Reflektif (Fokus
Paraphrasing):} Jelaskan bahwa \emph{Paraphrasing} (menyatakan kembali
pesan pembicara dengan kata-kata Anda sendiri) adalah kunci untuk
memverifikasi pemahaman dan memotong spiral defensif.
\textbf{Aktivitas:} \emph{Role-playing} berpasangan. Skenario: A
menyampaikan masalah emosional/instruksi ambigu, B memparafrasekan untuk
mengklarifikasi fakta dan perasaan. & Proses dan Teknik Mendengarkan
Aktif (Paraphrasing). \\
\textbf{90--110 Menit} & \textbf{Senjata Rahasia Pendengar yang
Kompeten} & \textbf{E. Spektrum Respons \& Empati:} Tinjau spektrum
respons (dari \emph{Silent Listening} hingga \emph{Advising}). Soroti
\textbf{Empathizing} sebagai respons suportif yang menunjukkan
kepedulian. Bahas \textbf{Counterfeit Questions} (pertanyaan palsu)
sebagai bentuk komunikasi strategis yang harus dihindari jika ingin
membangun iklim suportif. & Teknik Respons Mendengarkan (Empati,
Analisis, Nasihat). \\
\textbf{110--120 Menit} & \textbf{Penutup Inspiratif \& Kuis} &
\textbf{Inspirasi dan \emph{Call to Action}}: Mendorong mahasiswa untuk
menjadikan mendengarkan sebagai alat untuk \textbf{kontribusi kreatif}.
Ingatkan: Hanya dengan benar-benar memahami realitas orang lain (melalui
mendengarkan), kita dapat memberikan solusi yang transformatif. Kuis 10
pertanyaan dimulai. & Aplikasi Keterampilan \& Penutup. \\
\end{longtable}

\begin{center}\rule{0.5\linewidth}{0.5pt}\end{center}

\section{Catatan Kuliah (Lecture
Notes)}\label{catatan-kuliah-lecture-notes}

\subsection{1. Mendengarkan: Sebuah Kompetensi yang
Diremehkan}\label{mendengarkan-sebuah-kompetensi-yang-diremehkan}

\textbf{Mendengarkan (\emph{Listening})} adalah proses aktif membuat
makna dari pesan lisan orang lain. Ini bukan sekadar fungsi fisiologis
pasif seperti \textbf{Mendengar (\emph{Hearing})}.

\begin{itemize}
\tightlist
\item
  \textbf{Mindful Listening:} Memberikan perhatian yang hati-hati dan
  penuh pemikiran terhadap pesan, yang sangat penting ketika pesan
  tersebut vital bagi Anda atau orang yang Anda pedulikan.
\end{itemize}

\subsection{2. Lima Komponen Mendengarkan (Model HURIER / Proses
Mendengarkan
Interpersonal)}\label{lima-komponen-mendengarkan-model-hurier-proses-mendengarkan-interpersonal}

Meskipun tidak semua sumber menyebut model HURIER, komponen kuncinya
dijelaskan:

\begin{enumerate}
\def\labelenumi{\arabic{enumi}.}
\tightlist
\item
  \textbf{Mendengar (\emph{Hearing}/Sensing):} Proses di mana gelombang
  suara mencapai telinga dan otak.
\item
  \textbf{Memperhatikan (\emph{Attending}):} Proses penyaringan pesan
  (seleksi) agar menjadi fokus.
\item
  \textbf{Memahami (\emph{Understanding}):} Menerapkan filter persepsi
  (budaya, pengalaman, nilai) untuk menafsirkan pesan.
\item
  \textbf{Mengingat (\emph{Remembering}):} Menyimpan informasi untuk
  penggunaan di masa depan. Proses \emph{chunking} atau \emph{ordering}
  dapat membantu ingatan.
\item
  \textbf{Merespons (\emph{Responding}):} Memberikan \emph{feedback}
  verbal dan nonverbal kepada pengirim.
\end{enumerate}

\subsection{3. Gaya Mendengarkan Kunci}\label{gaya-mendengarkan-kunci}

\begin{itemize}
\tightlist
\item
  \textbf{Task-Oriented Listening:} Paling peduli dengan efisiensi dan
  menyelesaikan pekerjaan. Pendengar ini menyukai presentasi yang
  ringkas, rapi, dan bebas kesalahan (\emph{Action-oriented}).
\item
  \textbf{Relational Listening:} Berfokus pada pembangunan hubungan dan
  emosi orang lain (\emph{People-oriented}).
\item
  \textbf{Critical Listening:} Mengevaluasi konten pesan; cenderung
  berfokus pada logika, bukti, dan kesalahan.
\item
  \textbf{Analytical Listening:} Menganalisis pesan dari berbagai sudut
  pandang dan memahami secara detail sebelum membuat penilaian.
\end{itemize}

\subsection{4. Menghindari Kebiasaan Mendengarkan yang
Buruk}\label{menghindari-kebiasaan-mendengarkan-yang-buruk}

\begin{itemize}
\tightlist
\item
  \textbf{Pseudolistening (Pura-pura Mendengarkan):} Bertindak seperti
  mendengarkan tetapi tidak menyerap pesan.
\item
  \textbf{Selective Attention (Perhatian Selektif):} Hanya mendengarkan
  bagian pesan yang ingin didengar.
\item
  \textbf{Rebuttal Tendency:} Kecenderungan untuk mempersiapkan respons
  atau argumen balasan daripada mendengarkan secara penuh.
\end{itemize}

\subsection{5. Kekuatan Paraphrasing
(Parafrasa)}\label{kekuatan-paraphrasing-parafrasa}

\textbf{Paraphrasing} adalah keterampilan mendengarkan yang paling
penting, yaitu menyatakan kembali pesan pembicara dengan kata-kata Anda
sendiri.

\begin{itemize}
\tightlist
\item
  \textbf{Tujuan:} Untuk memverifikasi pemahaman pesan, baik informasi
  faktual maupun perasaan/pikiran yang mendasari.
\item
  \textbf{Penting:} Jika Anda tidak dapat memparafrasekan, Anda mungkin
  gagal memahami pesan tersebut, dan ini adalah sinyal untuk meminta
  klarifikasi.
\end{itemize}

\subsection{6. Respons Emosi dan
Pertanyaan}\label{respons-emosi-dan-pertanyaan}

\begin{itemize}
\tightlist
\item
  \textbf{Empathizing (Berempati):} Menyampaikan kepedulian terhadap apa
  yang orang lain rasakan dan alami.
\item
  \textbf{Sincere Questions (Pertanyaan Tulus):} Bertujuan untuk
  mengklarifikasi makna dan mengumpulkan fakta. Ini membantu pendengar
  global/pembelajar karena memberikan contoh untuk mengkonkretkan ide
  abstrak.
\item
  \textbf{Counterfeit Questions (Pertanyaan Palsu):} Disamarkan sebagai
  pertanyaan tetapi sebenarnya upaya untuk mengirim pesan atau
  mengontrol, dan harus dihindari dalam komunikasi suportif (contoh:
  pertanyaan yang memerangkap, pertanyaan yang membuat pernyataan,
  pertanyaan dengan agenda tersembunyi).
\end{itemize}

\begin{center}\rule{0.5\linewidth}{0.5pt}\end{center}

\section{Bahan Slides Pendukung (Konsep
Visual)}\label{bahan-slides-pendukung-konsep-visual}

Slides ringkas, fokus pada visual dan metafora untuk meningkatkan
pemahaman dan daya ingat.

\begin{longtable}[]{@{}
  >{\centering\arraybackslash}p{(\linewidth - 6\tabcolsep) * \real{0.2941}}
  >{\raggedright\arraybackslash}p{(\linewidth - 6\tabcolsep) * \real{0.2353}}
  >{\raggedright\arraybackslash}p{(\linewidth - 6\tabcolsep) * \real{0.2353}}
  >{\raggedright\arraybackslash}p{(\linewidth - 6\tabcolsep) * \real{0.2353}}@{}}
\toprule\noalign{}
\begin{minipage}[b]{\linewidth}\centering
Slide
\end{minipage} & \begin{minipage}[b]{\linewidth}\raggedright
Judul (Focus)
\end{minipage} & \begin{minipage}[b]{\linewidth}\raggedright
Isi Visual/Poin Kunci
\end{minipage} & \begin{minipage}[b]{\linewidth}\raggedright
Fungsi
\end{minipage} \\
\midrule\noalign{}
\endhead
\bottomrule\noalign{}
\endlastfoot
\textbf{1} & \textbf{MENDENGARKAN: Keterampilan Paling Berharga} &
Gambar wajah yang terlihat fokus dan tenang, kontras dengan ikon
\emph{noise} atau gangguan. & Hook \& Tujuan. \\
\textbf{2} & \textbf{Hearing vs.~Listening} & Diagram Sederhana: Telinga
-\textgreater{} Otak (Hearing, Pasif) vs.~Telinga -\textgreater{} Filter
Kognitif -\textgreater{} Respons (Listening, Aktif). & Kontras
Konseptual. \\
\textbf{3} & \textbf{4 Gaya Mendengarkan} & 4 Ikon: 1. Stop watch
(\emph{Task}); 2. Hati (\emph{Relational/People}); 3. Kaca Pembesar
(\emph{Analytical}); 4. Tanda Tanya/Timbangan (\emph{Critical}). &
Visualisasi Gaya. \\
\textbf{4} & \textbf{Jebakan ``Pura-Pura Mendengarkan''} & Karakter
kartun dengan mata terbuka lebar dan \emph{speech bubble} kosong di
pikiran mereka (\emph{Pseudolistening}). & Hambatan Mendengarkan
(Humor). \\
\textbf{5} & \textbf{Paraphrasing Power} & Rumus: \textbf{Pesan Anda
\(\rightarrow\) Filter Saya \(\rightarrow\) Ulangi (Kata-Kata Saya)
\(\rightarrow\) Konfirmasi.} Gambar dua orang berjabat tangan setelah
\emph{paraphrasing} berhasil. & Teknik Kunci \& Klarifikasi. \\
\textbf{6} & \textbf{Spektrum Respons} & Grafik Spektrum: Dari
\textbf{Reflective} (Silent, Paraphrasing) di kiri, hingga
\textbf{Directive} (Evaluating, Advising) di kanan. & Peta Alat
Respons. \\
\textbf{7} & \textbf{The Empathy Bridge} & Jembatan yang menghubungkan
dua orang di atas jurang (kesalahpahaman). Tulisan: \emph{Empathy:
Connecting to their feelings}. & Pesan Inspiratif (Pathos/Ethos). \\
\textbf{8} & \textbf{Call to Action: Be a Creative Contributor} &
Kutipan inspiratif (misalnya, tentang inovasi lahir dari pemahaman
mendalam). Hashtag: \#ikompetenkomunikasiantarpribadi. & Penutup \&
\emph{Action}. \\
\end{longtable}

\begin{center}\rule{0.5\linewidth}{0.5pt}\end{center}

\section{10 Pertanyaan Kuis}\label{pertanyaan-kuis}

Berikut adalah 10 pertanyaan kuis untuk menilai pemahaman tentang topik
Mendengarkan, dengan fokus pada terminologi dan konsep utama.

\begin{longtable}[]{@{}
  >{\centering\arraybackslash}p{(\linewidth - 2\tabcolsep) * \real{0.1571}}
  >{\raggedright\arraybackslash}p{(\linewidth - 2\tabcolsep) * \real{0.8429}}@{}}
\toprule\noalign{}
\begin{minipage}[b]{\linewidth}\centering
No.
\end{minipage} & \begin{minipage}[b]{\linewidth}\raggedright
Pertanyaan Kuis
\end{minipage} \\
\midrule\noalign{}
\endhead
\bottomrule\noalign{}
\endlastfoot
\textbf{1} & Apa perbedaan mendasar antara \emph{Hearing} (Mendengar)
dan \emph{Listening} (Mendengarkan)? \\
\textbf{2} & Gaya mendengarkan manakah yang paling menekankan pada
organisasi, presisi, dan pesan yang bebas kesalahan? \\
\textbf{3} & Ketika Anda hanya mendengarkan bagian dari pesan yang
sesuai dengan minat Anda dan mengabaikan sisanya, kebiasaan buruk ini
disebut apa? \\
\textbf{4} & Ketika seorang pendengar menyiapkan argumen balasan
alih-alih berfokus penuh pada pesan pembicara, ini dikenal sebagai
apa? \\
\textbf{5} & Apa tujuan utama dari teknik \emph{Paraphrasing}
(Parafrasa) dalam Mendengarkan Aktif? \\
\textbf{6} & Berikan satu contoh \emph{Reflective Listening Response}
(Respons Mendengarkan Reflektif). \\
\textbf{7} & Pertanyaan yang ditanyakan bukan untuk mendapatkan
informasi tetapi untuk mengirim pesan atau mengontrol penerima
(misalnya, ``Bukankah menurutmu sebaiknya kita berlibur sekarang?''),
disebut sebagai apa? \\
\textbf{8} & Proses aktif dan penuh pemikiran dalam memberikan perhatian
dan respons yang cermat terhadap pesan yang kita terima disebut apa? \\
\textbf{9} & Dalam konteks mendengarkan, respons yang mengungkapkan
kepedulian terhadap apa yang dirasakan dan dialami orang lain disebut
apa? \\
\textbf{10} & Mengapa penting untuk memparafrasekan (bukan hanya
mengulang) instruksi atau keputusan sebelum bertindak dalam lingkungan
profesional? \\
\end{longtable}

\bookmarksetup{startatroot}

\chapter{Ujian Tengah Semester Kelas
01}\label{ujian-tengah-semester-kelas-01}

UTS-5 bagian akhir dari rangaian UTS ini adalah melkaukan review dari
hasil submisi. UTS-5 Berisikan telahan pesan personal UTS-1 s/d UTS-4
berdasarkan rubrik masing-masing. Anda diminta melakukan Self Assesment
dan Peer Assessment menggunakan rubrik yang ada. Self Assessment
menggunakan AI, sedangkan Peer Assessment dilakukan manual

\section{Instruksi}\label{instruksi}

\begin{enumerate}
\def\labelenumi{\arabic{enumi}.}
\tightlist
\item
  Bila belum punya, download file excel
  \href{./asesmen/UTS-5_Skor.xlsx}{Lembar Skor}. File ini disimpan di
  repositori anda dengan di sarankan di folder UTS-5, lalu link kan ke
  portal anda pada bagian UTS-5
\item
  Bila belum, lakukan Self-Assessment menggunakan AI (Skor 2) dan catat
  skor nya pada baris teratas Lembar Skor
\item
  Lihat daftar penugasan Peer 1, Peer 2, Peer 3. Bila Anda menemukan
  submisi Peer tidak bisa dinilai (baik karena belum submit, link
  bermasalah, atau tidak ada isi yang bisa di nilai), beri skor 0.
\item
  Lakukan Peer-Assessment 1 Tanpa AI pada Peer 1 (Skor 4) dan catat
  skornya pada baris kedua Lembar Skor
\item
  Lakukan Peer-Assessment 2 Tanpa AI pada Peer 2 (Skor 4) dan catat
  skornya pada baris ketiga Lembar Skor
\item
  Lakukan Peer-Assessment 3 Tanpa AI Pada Peer 3(Skor 4, bonus) dan
  catat skornya pada baris keempat Lembar Skor
\item
  Pastikan Skor Penilaian diisi ke dalam file excel
  \href{./asesmen/UTS-5_Skor.xlsx}{Lembar Skor}. Paastikan skor keempat
  asesmen ini ada dalam file yang sama, masing-masing menggunakan satu
  baris, dengan urutan di atas. Berarti dimulai dengan baris berisi data
  Anda. Lalu update portal Anda, LALU LAPORKAN LINK FILE TERSEBUT KE MS
  FORM UTS \url{https://forms.cloud.microsoft/r/tF4pm04jUT}
\end{enumerate}

\section{Penugasaan Peer 1, Peer 2 dan Peer
3}\label{penugasaan-peer-1-peer-2-dan-peer-3}

Daftar Penugasan Kelas 01 TBD

\section{Submisi Kelas 01}\label{submisi-kelas-01}

\section{Laporan Submisi Tugas UTS Kelas
01}\label{laporan-submisi-tugas-uts-kelas-01}

\begin{longtable}[]{@{}
  >{\raggedright\arraybackslash}p{(\linewidth - 6\tabcolsep) * \real{0.1053}}
  >{\raggedright\arraybackslash}p{(\linewidth - 6\tabcolsep) * \real{0.1579}}
  >{\raggedright\arraybackslash}p{(\linewidth - 6\tabcolsep) * \real{0.1316}}
  >{\raggedright\arraybackslash}p{(\linewidth - 6\tabcolsep) * \real{0.6053}}@{}}
\toprule\noalign{}
\begin{minipage}[b]{\linewidth}\raggedright
No
\end{minipage} & \begin{minipage}[b]{\linewidth}\raggedright
Nama
\end{minipage} & \begin{minipage}[b]{\linewidth}\raggedright
NIM
\end{minipage} & \begin{minipage}[b]{\linewidth}\raggedright
Link URL Hasil Tugas
\end{minipage} \\
\midrule\noalign{}
\endhead
\bottomrule\noalign{}
\endlastfoot
1 & Ulivia Embun Tresna Wardani & 10322015 & Belum Submit \\
2 & Daffari Adiyatma & 18222003 &
\url{https://daffari.github.io/all-about-me} \\
3 & Daffa Ramadhan Elengi & 18222009 & Belum Submit \\
4 & Jonathan Wiguna & 18222019 &
\url{https://jonathanw33.github.io/all-about-me/} \\
5 & Kezia Caren Cahyadi & 18222041 & Belum Submit \\
6 & Dama Dhananjaya Daliman & 18222047 &
\url{https://runningpie.github.io/II2100_All-About-Me/} \\
7 & Christopher Richard Chandra & 18222057 & Belum Submit \\
8 & Winata Tristan & 18222061 &
\url{https://stdfreesince03.github.io/18222061-AsesmenII2100} \\
9 & Muhammad Faiz Atharrahman & 18222063 &
\url{https://all-about-me.faiz.at/} \\
10 & Muhammad Rafi Dhiyaulhaq & 18222069 &
\url{https://rafidhiyaulh.github.io/all-about-me/} \\
11 & Yoga Putra Pratama & 18222073 &
\url{https://yogaputrap.github.io/all-about-me/} \\
12 & Satria Wisnu Wibowo & 18222087 &
\url{https://peabnj.github.io/all-about-me/} \\
13 & Muhammad Adli Arindra & 18222089 &
\url{https://adli-arindra.github.io/all-about-me/} \\
14 & Wisyendra Lunarmalam & 18222095 &
\url{https://wisye.github.io/all-about-me/} \\
15 & Fawwaz Aydin Mustofa & 18222109 &
\url{https://fawwazay11.github.io/all-about-me/} \\
16 & Samuel Franciscus Togar H & 18222131 &
\url{https://sfrans21.github.io/ii2100_all-about-me/} \\
17 & Hanan Fitra Salam & 18222133 &
\url{https://hananfits.github.io/all-about-me/} \\
18 & Felissha Dunell Damanik & 18223099 & Belum Submit \\
19 & Muhammad Dhafin Faidhulhaq & 18223103 & Belum Submit \\
20 & Nisrina Zakiyah & 18224001 &
\url{https://bellechillguy.github.io/kominter-18224001/} \\
21 & Keira Amelie Hanayesha & 18224003 & Belum Submit \\
22 & Mishael Gilland & 18224005 &
\url{https://soyuzn.github.io/all-about-me/} \\
23 & Faizal Ali & 18224007 &
\href{https://github.com/zal1zal/all-about-me\%20&\%20https://github.io/zal1zal/all-about-me}{https://github.com/zal1zal/all-about-me
\& https://github.io/zal1zal/all-about-me} \\
24 & Keisha Daffa Aryani & 18224009 &
\url{https://keisharyaa.github.io/II2100_Keisha-Daffa/} \\
25 & Satria Guna Darma & 18224011 &
\url{https://astebern.github.io/all-about-me/} \\
26 & Salma Az Zahra & 18224015 &
\url{https://salmaazzra.github.io/all-about-me/} \\
27 & Naomi Azzahra & 18224017 &
\url{https://nanapossum.github.io/all-about-me/} \\
28 & Evan Nathanael Tanuri & 18224019 &
\url{https://evantanuri.github.io/all-about-me/} \\
29 & Muhammad Okto Huzainy & 18224021 &
\url{https://takkko-o.github.io/all-about-me/} \\
30 & Ahmad Aditya Ar Rasyid & 18224023 &
\url{https://kenpa1945.github.io/18224023-AsesmenII2100/} \\
31 & Muhammad Faris Daffa & 18224025 & Belum Submit \\
32 & Nafhan Shafy Aulia & 18224027 &
\url{https://nafhansa.github.io/all-about-me/} \\
33 & Mineva Azzahra & 18224029 &
\url{https://min646.github.io/all-about-me/} \\
34 & Daniel Arrigo Manurung & 18224031 &
\url{https://1920xl080.github.io/all-about-me/} \\
35 & Rezky Muhammad Hafiz Batubara & 18224033 &
\url{https://skyx59.github.io/18224033-AsesmenII2100/All_About_me/} \\
36 & Stella Cometta Febriana & 18224035 &
\url{https://github.com/18224035-dotcom/KOMUNIKASI-INTERPERSONAL-/tree/main/all-about-me-main} \\
37 & Indah Ramadhani & 18224037 &
\url{https://indahr118.github.io/all-about-me/} \\
38 & Luthfi Al Pasha & 18224039 & Belum Submit \\
39 & Irghi Satya Priangga & 18224041 &
\url{https://hippochld8.github.io/all-about-me-irghisp/} \\
40 & Aldyto Rafif Abhinaya & 18224043 &
\url{https://aldytorafif05.github.io/18224043_Aldyto-Rafif_UTS-Kominter/} \\
41 & Muthia Ariesta Anggraeni & 18224045 &
\url{https://muthiariesta.github.io/Kominter/} \\
42 & Daniel Benaya Toar Supaath & 18224047 &
\url{https://danielbnaya.github.io/all-about-me} \\
43 & Nizham Rafa Lazuardi & 18224049 &
\url{https://nizreal.github.io/all-about-me/} \\
44 & Belva Charissa Putri Have & 18224051 &
\url{https://belvacharissa.github.io/UTS-Komunikasi-Interpesonal-dan-Publik-/} \\
45 & Nathan Pasha Athallah & 18224053 & Belum Submit \\
46 & Adrian Akhdan Assyauqi & 18224055 &
\url{https://ztackholder.github.io/all-about-me/} \\
47 & Levina Nathania Bunardi & 18224057 &
\url{https://levinanthania.github.io/} \\
48 & Tiara Kusuma Wardhani & 18224059 & \url{tiaraksmwr.github.io} \\
49 & Riantama Putra & 18224061 &
\url{https://veinsan.github.io/all-about-me/} \\
50 & Yumna Fathonah Kautsar & 18224063 &
\url{https://yummypop.github.io/all-about-me/} \\
51 & Benedicta Sherin Chyntia Putri & 18224065 &
\url{https://benedictasherinn.github.io/II-2100/} \\
52 & Bryant Azraqi Mohammad & 18224067 &
\url{https://azraqii.github.io/all-about-me/index.html} \\
53 & Fikrifalah Muslich & 18224069 &
\url{https://fikrifalah.github.io/18224069/All_About_me/} \\
54 & Ellaine Juvina & 18224071 &
\url{https://aucloire.github.io/all-about-me/} \\
55 & Abdullah Ahmad Yusuf & 18224073 &
\url{https://yoeztzy.github.io/all-about-me/} \\
56 & Sherry Eunike & 18224075 &
\url{https://gasukabawang.github.io/all-about-me/} \\
57 & Naufalziyadh Alif Bintang S & 18224077 &
\url{https://github.com/ItsOxiris/all-about-me} \\
58 & Rafi Putra Nugraha & 18224079 & \url{https://duskoid.github.io/} \\
59 & Illona Nasywa Hannum & 18224081 &
\url{https://buahkol.github.io/all-about-me/} \\
60 & Antania Hanjani Yustika Putri & 18224083 &
\url{https://hantaniaa.github.io/all-about-me/} \\
61 & Ahmad Rizal Fahmi & 18224085 &
\url{https://lntermezzzo.github.io/about-me/} \\
62 & Abhinaya Rajendra Fargaz & 18224087 &
\url{https://about-abhinaya.vercel.app} \\
63 & Muhammad Azikra Wira Pratama & 18224089 &
\url{https://awp-confirm.github.io/all-about-me/} \\
64 & Made Krisna Kusuma Wijaya & 18224091 &
\url{https://yaisnaa.github.io/II2100_Krisna-Kusuma/} \\
65 & Kenneth Moses Saragih & 18224093 &
\url{https://kaem-nyoba.github.io/II2100_Kenneth-Moses/} \\
66 & Sirojul Firdaus & 18224095 &
\url{https://sirojulfirdaus.github.io/kipp-all-about-me/} \\
67 & Lukman Hakim Syah Ardhana & 18224097 &
\url{https://lukmanhakimsa.github.io/all-about-me/} \\
68 & Muhammad Afif Habiburrahman & 18224099 &
\url{https://github.com/fifiyy/all-about-me} \\
69 & Jacko Luciano & 18224101 &
\url{jackoluciano.github.io/all-about-me} \\
70 & Haidarrozan Pramasony & 18224103 &
\url{https://oxanovijk.github.io/18224103-Kominter/} \\
71 & Kyan Saktya Diraya & 18224105 &
\url{https://kyanureeves.github.io/all-about-me/} \\
72 & Laurensius Dani Rendragraha & 18224107 &
\url{https://laurensiusdani.github.io/all-about-me/} \\
73 & Adriel Nathanael Simatupang & 18224109 &
\url{https://mimumscar.github.io/all-about-me/} \\
\end{longtable}

\bookmarksetup{startatroot}

\chapter{Ujian Tengah Semester Kelas
03}\label{ujian-tengah-semester-kelas-03}

UTS-5 bagian akhir dari rangaian UTS ini adalah melkaukan review dari
hasil submisi. UTS-5 Berisikan telahan pesan personal UTS-1 s/d UTS-4
berdasarkan rubrik masing-masing. Anda diminta melakukan Self Assesment
dan Peer Assessment menggunakan rubrik yang ada. Self Assessment
menggunakan AI, sedangkan Peer Assessment dilakukan manual

\section{Instruksi}\label{instruksi-1}

\begin{enumerate}
\def\labelenumi{\arabic{enumi}.}
\tightlist
\item
  Bila belum punya, download file excel
  \href{./asesmen/UTS-5_Skor.xlsx}{Lembar Skor}. File ini disimpan di
  repositori anda dengan di sarankan di folder UTS-5, lalu link kan ke
  portal anda pada bagian UTS-5
\item
  Bila belum, lakukan Self-Assessment menggunakan AI (Skor 2) dan catat
  skor nya pada baris teratas Lembar Skor
\item
  Lihat daftar penugasan Peer 1, Peer 2, Peer 3. Bila Anda menemukan
  submisi Peer tidak bisa dinilai (baik karena belum submit, link
  bermasalah, atau tidak ada isi yang bisa di nilai), beri skor 0.
\item
  Lakukan Peer-Assessment 1 Tanpa AI pada Peer 1 (Skor 4) dan catat
  skornya pada baris kedua Lembar Skor
\item
  Lakukan Peer-Assessment 2 Tanpa AI pada Peer 2 (Skor 4) dan catat
  skornya pada baris ketiga Lembar Skor
\item
  Lakukan Peer-Assessment 3 Tanpa AI Pada Peer 3(Skor 4, bonus) dan
  catat skornya pada baris keempat Lembar Skor
\item
  Pastikan Skor Penilaian diisi ke dalam file excel
  \href{./asesmen/UTS-5_Skor.xlsx}{Lembar Skor}. Paastikan skor keempat
  asesmen ini ada dalam file yang sama, masing-masing menggunakan satu
  baris, dengan urutan di atas. Berarti dimulai dengan baris berisi data
  Anda. Lalu update portal Anda, LALU LAPORKAN LINK FILE TERSEBUT KE MS
  FORM UTS \url{https://forms.cloud.microsoft/r/tF4pm04jUT}
\end{enumerate}

\section{Penugasaan Peer 1, Peer 2 dan Peer
3}\label{penugasaan-peer-1-peer-2-dan-peer-3-1}

\section{Submisi Kelas 03}\label{submisi-kelas-03}

\section{Laporan Submisi Tugas UTS Kelas
03}\label{laporan-submisi-tugas-uts-kelas-03}

\begin{longtable}[]{@{}
  >{\raggedright\arraybackslash}p{(\linewidth - 6\tabcolsep) * \real{0.1053}}
  >{\raggedright\arraybackslash}p{(\linewidth - 6\tabcolsep) * \real{0.1579}}
  >{\raggedright\arraybackslash}p{(\linewidth - 6\tabcolsep) * \real{0.1316}}
  >{\raggedright\arraybackslash}p{(\linewidth - 6\tabcolsep) * \real{0.6053}}@{}}
\toprule\noalign{}
\begin{minipage}[b]{\linewidth}\raggedright
No
\end{minipage} & \begin{minipage}[b]{\linewidth}\raggedright
Nama
\end{minipage} & \begin{minipage}[b]{\linewidth}\raggedright
NIM
\end{minipage} & \begin{minipage}[b]{\linewidth}\raggedright
Link URL Hasil Tugas
\end{minipage} \\
\midrule\noalign{}
\endhead
\bottomrule\noalign{}
\endlastfoot
1 & Rian Albar Insani & 11422020 & Belum Submit \\
2 & Rifki Fariz Farabi & 11422030 & Belum Submit \\
3 & Rhio Bimo Prakoso Sugiyanto & 13523123 & Belum Submit \\
4 & Muhammad Raihaan Perdana & 13523124 & Belum Submit \\
5 & Dita Maheswari & 13523125 & Belum Submit \\
6 & Andi Farhan Hidayat & 13523128 & Belum Submit \\
7 & Ivant Samuel Silaban & 13523129 & Belum Submit \\
8 & Rafa Abdussalam Danadyaksa & 13523133 & Belum Submit \\
9 & Sebastian Enrico Nathanael & 13523134 & Belum Submit \\
10 & Ahmad Syafiq & 13523135 &
\url{https://iammadsfq.github.io/all-about-me/} \\
11 & Danendra Shafi Athallah & 13523136 & Belum Submit \\
12 & Jonathan Kenan Budianto & 13523139 & Belum Submit \\
13 & Mahesa Fadhillah Andre & 13523140 & Belum Submit \\
14 & Andrew Tedjapratama & 13523148 & Belum Submit \\
15 & Naufarrel Zhafif Abhista & 13523149 & Belum Submit \\
16 & Muhammad Kinan Arkansyaddad & 13523152 & Belum Submit \\
17 & Muhammad Farrel Wibowo & 13523153 & Belum Submit \\
18 & M. Abizzar Gamadrian & 13523155 &
\url{https://abizzarg.github.io/all-about-me/All_About_me/} \\
19 & Arlow Emmanuel Hergara & 13523161 & Belum Submit \\
20 & Fachriza Ahmad Setiyono & 13523162 & Belum Submit \\
21 & Filbert Engyo & 13523163 & Belum Submit \\
22 & Muhammad Rizain Firdaus & 13523164 & Belum Submit \\
23 & Jasmine Oryza Keinina Bangun & 14324005 & Belum Submit \\
24 & Ayudia Nisrina Tsabitah & 14324016 & Belum Submit \\
25 & Nashwa Kamila & 14324017 & Belum Submit \\
26 & Andreas Saputra Tambun & 18224110 & Belum Submit \\
27 & Raditya Nanda Dhevara & 18224111 & Belum Submit \\
28 & Muhammad Reyna Athallah Agoes & 18224112 & Belum Submit \\
29 & Herlambang Setiaji Prabowo & 18224113 & Belum Submit \\
30 & Daniel Wicaksana Godjali & 18224114 & Belum Submit \\
31 & Muhammad Arkan Dhaifullah & 18224116 & Belum Submit \\
32 & Stephanie Mae & 18224117 & Belum Submit \\
33 & Brandon Zeko Alexander & 18224118 & Belum Submit \\
34 & Firanti Naulia Fasya & 18224119 & Belum Submit \\
35 & Ibrahim Ferizarizqi Permana & 18224120 & Belum Submit \\
36 & Aisyah Az Zahra & 18224121 & Belum Submit \\
37 & Tyara Penelope Lumban Gaol & 18224122 & Belum Submit \\
38 & Muchammad Rafif Azis Syahlevi & 18224123 & Belum Submit \\
39 & Geodipa Afatha Ryu M.F.Z. & 18224124 & Belum Submit \\
40 & Rachel Elnora Mannuelle S. & 18224125 & Belum Submit \\
41 & Muhamad Radhitya Alamsyah & 18224126 & Belum Submit \\
\end{longtable}

\bookmarksetup{startatroot}

\chapter{Kuliah 8}\label{kuliah-8}

\emph{TBD}

\bookmarksetup{startatroot}

\chapter{Kuliah 9}\label{kuliah-9}

\emph{TBD}

\bookmarksetup{startatroot}

\chapter{Kuliah 10}\label{kuliah-10}

\emph{TBD}

\bookmarksetup{startatroot}

\chapter{Kuliah 11}\label{kuliah-11}

\emph{TBD}

\bookmarksetup{startatroot}

\chapter{Kuliah 12}\label{kuliah-12}

\emph{TBD}

\bookmarksetup{startatroot}

\chapter{Kuliah 13}\label{kuliah-13}

\emph{TBD}

\bookmarksetup{startatroot}

\chapter{Summary}\label{summary}

In summary, this book has no content whatsoever.

\bookmarksetup{startatroot}

\chapter*{References}\label{references}
\addcontentsline{toc}{chapter}{References}

\markboth{References}{References}

\phantomsection\label{refs}




\end{document}
